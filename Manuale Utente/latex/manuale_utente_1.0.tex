\documentclass[a4paper,11pt]{article}
	\input{../../template.tex}
	\newcommand{\code}[1]{\texttt{#1}}

	\title{\textbf{{\fontsize{8mm}{5mm}\selectfont QUIZZIPEDIA}}}
	\date{}
	\author{}	


\begin{document}
	\maketitle
	\thispagestyle{empty}
	\begin{center}	
	\includegraphics{../team_not_found.jpg}\\
	\fontsize{5mm}{3mm}\url{team404swe@gmail.com}\\
	
	\vspace{50mm}
	\textbf{Manuale Utente 1.0}
	\end{center}
	\introtab{Manuale Utente}	%1 nome documento
			{2.0} 							%2 versione
			{Esterno} 						%3 Uso
			{16 agosto 2016} 				%4 Data cre
			{\today} 						%5 Data mod
			{Luca Alessio}		%6 Redazione
			{Martin V. Mbouenda} 			%7 Verifica
			{Davide Bortot} 				%8 Approvazione
	\newpage
	\thispagestyle{empty}
	\null  

	\newpage
	\newpage
	\fancyhead[R]{REGISTRO DELLE MODIFICHE}
	\fancyfoot[R]{\thepage}
	
	\hspace{30 mm}
	\section*{Registro delle modifiche}
	
	\beginregistro
	
	\rigaregistro{\textbf{Versione}}{\textbf{Autore}}{\textbf{Data}}		 {\hspace{5 mm}}
	\rigaregistro{0.0.7}{Luca Alessio (Verificatore)}{18/06/2016}{Stesura sezione "Scelta di un questionario"}
	\rigaregistro{0.0.6}{Luca Alessio (Verificatore)}{18/06/2016}{Stesura sezioni "Registrazione" e "Autenticazione"}
	\rigaregistro{0.0.5}{Luca Alessio (Verificatore)}{18/06/2016}{Stesura sezione "L'applicazione a prima vista"}
	\rigaregistro{0.0.4}{Andrea Multineddu (Programmatore)}{17/06/2016}{Inserimento immagini}
	\rigaregistro{0.0.3}{Luca Alessio (Verificatore)}{16/06/2016}{Stesurea sezione "Requisiti di sistema"}
	\rigaregistro{0.0.2}{Luca Alessio (Verificatore)}{16/06/2016}{Stesura sezione "Introduzione"}
	\rigaregistro{0.0.1}{Luca Alessio (Verificatore)}{16/06/2016}{Creazione del documento}
	
	\fineregistro

	\newpage
	\fancyhead[R]{\leftmark}
	\tableofcontents
	\pagenumbering{Roman}
	\newpage
	\listoffigures
	\listoftables
	
	\newpage
	\pagenumbering{arabic}
	
	\section*{Sommario}
	Questo documento rappresenta il Manuale Utente per il software Quizzipedia sviluppato dal Team404 per conto di Zucchetti S.p.A. ed ha lo scopo di illustrare all'utente le modalità di utilizzo e i servizi offerti dell'applicazione.
	
	\newpage
	\section{Introduzione}
	\subsection{Scopo del documento}
	Il documento ha lo scopo di definire nel dettaglio la struttura del sistema Quizzipedia, approfondendone la descrizione già definita nel documento di Specifica Tecnica. Per ogni package del sistema verrà data una descrizione estensiva delle sue classi. Per poter sviluppare al meglio il prodotto, i programmatori dovranno attenersi alle specifiche definite in questo documento.
	
	\subsection{Scopo del prodotto}
	Il progetto \textbf{Quizzipedia} ha come obiettivo lo sviluppo di un sistema software basato su tecnologie Web (Javascript\addglos, Node.js\addglos, HTML5\addglos, CSS3\addglos) che permetta la creazione, gestione e fruizione di questionari. Il sistema dovrà quindi poter archiviare i questionari suddivisi per argomento, le cui domande dovranno essere raccolte attraverso uno specifico linguaggio di markup (Quiz Markup Language) d'ora in poi denominato QML\addglos. In un caso d'uso a titolo esemplificativo, un "esaminatore" dovrà poter costruire il proprio questionario scegliendo tra le domande archiviate, ed il questionario così composto sarà presentato e fruibile all' "esaminando", traducendo l'oggetto QML in una pagina HTML\addglos, tramite un'apposita interfaccia web. Il sistema presentato dovrà inoltre poter proporre questionari preconfezionati e valutare le risposte fornite dall'utente finale.
	\\
	Per un'analisi più precisa ed approfondita del progetto si rimanda al documento\\ "\textit{analisi\_dei\_requisiti\_4.0.pdf}".
	\subsection{Glossario}
	Viene allegato un glossario nel file ``\textit{glossario\_4.0.pdf}'' nel quale viene data una definizione a tutti i termini che in questo documento appaiono con il simbolo '\addglos' a pedice.
	\subsection{Riferimenti}
		\subsubsection{Normativi}

		\begin{itemize}
			\item Capitolato d'appalto Quizzipedia:\\
			\url{http://www.math.unipd.it/~tullio/IS-1/2015/Progetto/C5.pdf}
			\item Norme di Progetto: "\textit{norme\_di\_progetto\_4.0.pdf}"
		\end{itemize}
		\subsubsection{Informativi}
		\begin{itemize}
			\item Corso di Ingegneria del Software anno 2015/2016:\\
			\url{http://www.math.unipd.it/~tullio/IS-1/2015/}
			\item Regole del progetto didattico:\\
			\url{http://www.math.unipd.it/~tullio/IS-1/2015/Dispense/PD01.pdf}\\
			\url{http://www.math.unipd.it/~tullio/IS-1/2015/Progetto/}
			\item Specifica Tecnica: "\textit{specifica\_tecnica\_3.0.pdf}"
			\item Framework Meteor:\\
			\url{https://www.meteor.com/}
			\item Framework AngularJs:\\
			\url{https://www.angularjs.org/}
		\end{itemize}
	\pagebreak
	\newpage
\section{Requisiti di Sistema}
Quizzipedia è un'applicazione web, pertanto è mandatorio che il dispositivo utilizzato sia connesso ad internet. Per quanto riguarda l'hardware del dispositivo, non sono invece richieste particolari specifiche.
\subsection{Browser supportati}
Quizzipedia supporta i seguenti browser:
\begin{itemize}
	\item Chrome
	\item Firefox 7+
	\item Internet Explorer 8+
	\item Safari 4+
	\item Android stock web browser
	\item IOS stock web browser
	\item Opera Desktop e Mobile
\end{itemize}
	\newpage
	\section{L'applicazione a prima vista}
	All'accesso dell'utente l'applicazione si presenta come segue:\\
	\begin{figure}[h!]
	\begin{center}
	\includegraphics[scale=0.5]{../images/screen_home.png}
	\end{center}
	\end{figure}
	L'utente ha la possibilità partire direttamente con la risoluzione di un questionario o di autenticarsi nel sistema per usufruire delle funzionalità di gestione di domande e questionari che vedremo nel dettaglio nei seguenti paragrafi.\\ \\
	Per la home page si è scelto di adottare un layout il più possibile minimale per renderlo semplice e accessibile ad ogni tipologia di utenza.\\
	Da ogni pagina del sito è possibile usufruire del comodo menù di navigazione laterale che elenca le funzionalità disponibili del sistema (alcune come ad esempio la creazione della domanda potrebbero essere bloccate qualora l'utente non abbia ancora effettuato l'accesso).
	\newpage
	\section{Registrazione}
	Per poter usufruire delle funzionalità più avanzate di Quizzipedia, l'utente deve necessariamente creare il proprio account nel sistema.\\
	L'utente può trovare il modulo di registrazione in alto a sinistra alla voce "Register", il modulo si presenta come segue:\\
	\begin{figure}[h!]
	\begin{center}
	\includegraphics[scale=0.85]{../images/form.png}
	\end{center}
	\end{figure}	
	\\
	E' necessario inserire nei corrispondenti campi un indirizzo email valido e una password che va ripetuta per evitare errori di battitura.\\ \\
	Se i dati inseriti sono corretti, alla pressione del tasto "Create Account" il nuovo profilo dell'utente verrà effettivamente creato e l'utente potrà accedere a tutte le funzionalità di Quizzipedia, in caso contrario verrà notificato da un messaggio d'errore appropriato.
	\section{Autenticazione}
	L'autenticazione nel sistema segue lo stesso principio della registrazione: si accede al form dalla voce nel menù Sign In, si inseriscono le proprie credenziali e si logga così nel sistema.\\ Se i dati inseriti non sono corretti ancora una volta l'utente viene notificato da un messaggio d'errore. 
	\newpage
	\section{Scelta di un questionario}
	La scelta del questionario si svolge su due livelli, prima avviene la scelta della categoria e poi avviene la scelta del singolo questionario presente tra i vari possibili questionari della stessa categoria.
	\begin{figure}[h!]
	\begin{center}
	\includegraphics[scale=0.5]{../images/screen_category.png}
	\end{center}
	\end{figure}
	\\
	Ad ogni questionario è abbinata una breve descrizione generale sugli argomenti che tratta.\\ E' possibile cominciare il questionario semplicemente cliccando sul tasto Svolgi.
	\newpage
	\section{Svolgimento di un questionario}
	spiegazione cosa fa ogni pulsante
	\newpage
	\section{Profilo Utente}
	c'è?
	\newpage
	\section{Creazione di una domanda}
	E' possibile creare una domanda in due modi differenti, tramite l'inserimento di codice QML puro oppure interagendo con un form semplificato da cui è possibile selezionare la categoria della domanda ed inserire il testo di domande e risposte. \\
	immagine qui delle due possibilità
	\newpage
	\section{Creazione di un questionario}
	foto con spiegazione pulsanti
	\newpage
	\section{Sintassi QML}
	
\end{document}