%!TEX encoding = UTF-8 Unicode	
	\documentclass[a4paper,11pt]{article}
	%INCLUDE DEL TEMPLATE
	\input{../../template.tex}
	
	\DeclareUnicodeCharacter{00A0}{ }
	
	\title{\textbf{{\fontsize{8mm}{5mm}\selectfont QUIZZIPEDIA}}}
	\date{}
	\author{}	
		
	\begin{document}
	\pagenumbering{Roman}
	\maketitle
	\thispagestyle{empty}
	\begin{center}
	\includegraphics{../../team_not_found.jpg}\\
	\fontsize{5mm}{3mm}\url{team404swe@gmail.com}\\
	
	\vspace{50mm}
	\textbf{Norme di Progetto 3.0}	
	\end{center}
	
	%qui
	\introtab{Norme di Progetto}			%1 nome documento
			{3.0} 							%2 versione
			{Interno} 						%3 Uso
			{21 dicembre 2015} 				%4 Data cre
			{\today} 						%5 Data mod
			{Martin Vadice Mbouenda}		%6 Redazione1
			{Alex Beccaro} 			%7 Verifica
			{Luca Alessio} 				%8 Approvazione
	%qui
	\newpage
	\null
	\thispagestyle{empty}
	
	\newpage	
	\newpage
	\fancyhead[R]{REGISTRO DELLE MODIFICHE}
	\fancyfoot[R]{\thepage}
	
	\hspace{30 mm}
	\section*{Registro delle modifiche}
	%\begin{longtable}{{p{0.10\textwidth}p{0.15\textwidth}p{0.12\textwidth}p{0.50\textwidth}}}
		%\begin{longtable}{{|p{0.10\textwidth}|p{0.15\textwidth}|p{0.15\textwidth}|p{0.50\textwidth}|}} 
		\beginregistro
		
			\rigaregistro{\textbf{Versione}}{\textbf{Autore}}{\textbf{Data}}{\hspace{5 mm} \textbf{Descrizione}}
		%REVIZIONE DI QUALIFICA
			\rigaregistro{3.0}{M. Crivellaro (Responsabile)}{05/06/2016}{Approvazione del documento alla versione 3.0}
			\rigaregistro{2.2}{M.Mbouenda  (Verificatore)}{04/06/2016}{Verifica del documento}
			\rigaregistro{2.1.2}{A. Beccaro  (Amministratore)}{02/06/2016}{Completato il dettaglio sugli strumenti}			
			\rigaregistro{2.1.1}{A. Beccaro  (Amministratore)}{02/06/2016}{Ristrutturazione secondo ISO 12207: processo di gestione, processo di allestimento. inizio miglioramento attività di progettazione}
			\rigaregistro{2.1}{M.Mbouenda  (Verificatore)}{28/05/2016}{Revisione del documento}
			\rigaregistro{2.0.2}{M. Mbouenda   (Responsabile)}{27/05/2016}{Ristrutturazione secondo ISO 12207: Processo di sviluppo, sistemazione attività di analisi dei requisiti }			
			\rigaregistro{2.0.1}{M. Mbouenda   (Responsabile)}{26/05/2016}{Ristrutturazione secondo ISO 12207: Processi primari}
			
		%REVIZIONE DI PROGETTO
			\rigaregistro{2.0}{A. Beccaro  (Verificatore)}{11/05/2016}{Approvazione documento.}		
			\rigaregistro{1.1}{A. Multineddu  (Amministratore)}{10/05/2016}{Verifica del documento.}
			\rigaregistro{1.0.4}{M. Mbouenda  (Amministratore)}{09/05/2016}{Aggiunta sottosezione §2.4.3.}
			\rigaregistro{1.0.3}{M. Mbouenda  (Amministratore)}{07/05/2016}{Modifica sottosezione Versionamento e Git Modo d'uso.}
			\rigaregistro{1.0.2}{M. Mbouenda  (Amministratore)}{05/05/2016}{Modifica sottosezione Analisi dei requisiti.}
			\rigaregistro{1.0.1}{M. Mbouenda  (Amministratore)}{04/05/2016}{Modifica sottosezione Ticketing.}
		%REVIZIONE DEI REQUISITI			
			\rigaregistro{1.0}{Davide Bortot (Responsabile)}{16/03/2016}{Approvazione del documento.}
			\rigaregistro{0.2}{A. Multineddu (Verificatore)}{15/03/2016}{Verifica completa del documento.}
			\rigaregistro{0.1.7}{D. Bortot (Amministratore)}{11/03/2016}{Modifica sottosezione "Componenti grafiche".}
	 		\rigaregistro{0.1.6}{M. Mbouenda (Amministratore)}{11/03/2016}{Modifica sottosezione "Test".}
	 		\rigaregistro{0.1.5}{M. Mbouenda (Amministratore)}{10/03/2016}{Modifica sezione "Processo di Sviluppo".}
	 		\rigaregistro{0.1.4}{M. Mbouenda (Amministratore)}{18/01/2016}{Stesura sottosezione "Produzione dei Documenti".}
	 		\rigaregistro{0.1.3}{M. Mbouenda (Amministratore)}{14/01/2016}{Stesura sottosezione "Strumenti di Progetto".}
	 		\rigaregistro{0.1.2}{M. Mbouenda (Amministratore)}{12/01/2016}{Modifica della sottosezione "Processo di verifica".}
			\rigaregistro{0.1.1}{M. Mbouenda (Amministratore)}{10/01/2016}{Modifica sottosezione "Comunicazione".}
			\rigaregistro{0.1.0}{A. Multineddu (Verificatore)}{10/01/2016}{Prima verifica. Segnalati errori nella sottosezione "Comunicazione".}
			\rigaregistro{0.0.5}{M. Mbouenda (Amministratore)}{08/01/2016}{Inizio stesura della sezione "Processo Organizzativo".}			
			\rigaregistro{0.0.4}{M. Mbouenda (Amministratore)}{05/01/2016}{Inizio stesura della sezione "Processo di Supporto"}
			\rigaregistro{0.0.3}{M. Mbouenda (Amministratore)}{04/01/2016}{Stesura sottosezione "Progettazione" e "Codifica".}
			\rigaregistro{0.0.2}{M. Mbouenda (Amministratore)}{27/12/2015}{Inizio stesura della sezione "Processo di Sviluppo"}
			\rigaregistro{0.0.1}{M. Mbouenda (Amministratore)}{21/12/2015}{Prima stesura del documento. Redazione della sezione "Introduzione" e strutturazione delle sezioni "Processo di Sviluppo", "Processo di supporto", "Processo organizzativo".}
			\caption{Versionamento del documento} 
		\fineregistro
	\newpage
	\fancyhead[R]{\leftmark}
	\tableofcontents
	\newpage
	\listoftables
	
	\listoffigures
	
	\newpage
	\section*{Sommario}
		In questo documento sono illustrate le "Norme di Progetto" del gruppo \textbf{Team404} relativo al capitolato \textbf{Quizzipedia}, commissionato da \textbf{Zucchetti S.p.A.} \\
		Lo scopo del documento è di descrivere le regole e procedure adottate dal gruppo per la realizzazione del capitolato. 
	\newpage
	\pagenumbering{arabic}
	\section{Introduzione}
	
		\subsection{Scopo del documento}
			Questo documento riporta le regole e convenzioni per il coordinamento dei rapporti interni ed esterni del gruppo. Definisce anche gli strumenti da utilizzare e le modalit\`a del lavoro al quale ogni membro del gruppo si dovr\`a sottoporre per garantire una migliore collaborazione e coerenza nello svolgimento del lavoro.\\
		Ogni modifica del presente documento verr\`a tempestivamente notificata ad ogni componente del team per via dei canali adottati.
		\subsection{Scopo del Prodotto}
			Il progetto \textbf{Quizzipedia} ha come obiettivo lo sviluppo di un sistema software basato su tecnologie Web (\inglos{Javascript}, \inglos{NodeJS}, \inglos{HTML5}, \inglos{CSS3}) che permetta la creazione, gestione e fruizione di questionari. Il sistema dovrà quindi poter archiviare i questionari suddivisi per argomento, le cui domande dovranno essere raccolte attraverso uno specifico linguaggio di markup che verrà chiamato "Quiz Markup Language" e d'ora in poi denominato \inglos{QML}. In un caso d'uso a titolo esemplificativo, un "esaminatore" dovrà poter costruire il proprio questionario scegliendo tra le domande archiviate, ed il questionario così composto sarà presentato e fruibile all' "esaminando", traducendo l'oggetto \inglos{QML} in una pagina \inglos{HTML}, tramite un'apposita interfaccia web. Il sistema presentato dovrà inoltre poter proporre questionari preconfezionati e valutare le risposte fornite dall'utente finale.	\\
		\subsection{Glossario}
			Viene allegato un glossario nel file "\textit{glossario\_3.0.pdf}" nel quale viene data una definizione a tutti i termini che in questo documento appaiono con il simbolo \textbf{'\addglos'}  a pedice.
		\subsection{Riferimenti}
			\subsubsection{Normativi}
				\begin{itemize}
					\item \textbf{Capitolato d'appalto Quizzipedia:}\\
					\url{http://www.math.unipd.it/~tullio/IS-1/2015/Progetto/C5.pdf}
					\item \textbf{Analisi dei Requisiti:} "\textit{analisi\_dei\_requisiti\_3.0.pdf}"
					\item \textbf{Piano di Progetto:} "\textit{piano\_di\_progetto\_3.0.pdf}"
					\item \textbf{Piano di Qualifica:} "\textit{piano\_di\_qualifica\_3.0.pdf}"
					\item \textbf{Studio di Fattibilità:} "\textit{studio\_di\_fattibilità\_3.0.pdf}"
					\item \textbf{Introduzione all'uso di git:} \\
					\url{http://git-scm.com/book/it} 
				\end{itemize}
			\subsubsection{Informativi}
				\begin{itemize}
					\item Corso di Ingegneria del Software anno 2015/2016:\\
					\url{http://www.math.unipd.it/~tullio/IS-1/2015/}
					\item Regole del progetto didattico:\\
					\url{http://www.math.unipd.it/~tullio/IS-1/2015/Dispense/PD01.pdf}
					\url{http://www.math.unipd.it/~tullio/IS-1/2015/Progetto/}\\
					\url{http://www.math.unipd.it/~tullio/IS-1/2015/Progetto/PD01b.html}
				\end{itemize}
	
	\newpage
	\section{Processi Primari}
	\subsection{Processo di fornitura}
		\subsubsection{Studio di fattibilità}	
	Vedi il documento "Studio\_di\_fattibilita\_1.0.pdf" consegnato durante le precedenti revisioni.
	
		\subsection{Processo di sviluppo}	
		Questo processo ha come scopo la produzione di un elemento di un sistema implementato come prodotto software. Trasforma specifici comportamenti, interfacce, e vincoli di implementazione in azioni che creano un sistema implementato come prodotto software. Il processo produce un software che soddisfa i requisiti architetturali e li verifica attraverso la	verifica e la validazione.	
		Il processo consiste delle seguente attività:
			\begin{enumerate}
				\item Analisi dei requisiti
				\item Progettazione
				\item Codifica
				\item Verifica e validazione
			\end{enumerate}
		\subsubsection{Analisi dei requisiti}
		Sarà compito degli analisti estrarre i requisiti del progetto dal capitolato e/o da eventuali incontri col proponente. Questa attività ha per obiettivo la stesura di un documento affidabile e consistente che descrive le richieste ed esigenze del proponente.

			\subsubsubsection{ Casi d'uso}
			Ogni caso d'uso dovrà essere descritto nel seguente modo
			\begin{center}
					%\begin{Large}
						\textbf{ UC[codice identificativo][gerarchia]}		
					%\end{Large}										
					
			\end{center}
			 presentare i seguenti campi obbligatori:
			\begin{itemize}
			\item \textbf{Codice identificativo:} codice univoco che rappresenterà il caso d'uso. 
			\item \textbf{Titolo:} sarà il nome del caso d'uso, scelto a secondo del caso.
			\item \textbf{Attori primari:} identifica i protagonisti coinvolti direttamente.
			\item \textbf{Descrizione:} minimale e precisa che descrive il caso d'uso.
			\item \textbf{Precondizione:} condizioni verificate prima del caso d'uso.
			\item \textbf{Postcondizione:} condizioni verificate alla fine del caso d'uso.
			\item \textbf{Scenario:} descrizione della situazione e del contesto di attuazione del caso d'uso.
			\item \textbf{Diagramma UML :} sarà una rappresentazione grafica del caso.
			\end{itemize}
			E a secondo del caso, i seguenti campi saranno specificati dove presenti:
			\begin{itemize}
			\item \textbf{Attori secondari:} identifica i protagonisti coinvolti indirettamente.
			\item \textbf{Scenari alternativi:} descrizione delle situazioni che non appartengono al flusso principale di esecuzione.
			\item \textbf{Estensioni:} per segnare le estensione del caso. 
			\item \textbf{Inclusioni}  per segnare le inclusioni del caso.
			\end{itemize}
							
			\subsubsubsection{ Requisiti}
			Ogni requisito dovrà avere i seguenti campi:
			\begin{itemize}
			\item \textbf{Codice identificativo:} 
			Ogni requisito sarà rappresentato da una lettera maiuscola iniziale seguita da une serie di numeri separati da un punto che identificherà la relazione gerarchica tra i requisiti dello stesso tipo. Un esempio di questa rappresentazione è la seguente:
					\begin{center}
					\begin{Large} X\end{Large}  \begin{normalsize}1.2.*.n	\end{normalsize}					
					\end{center}
					dove la lettera iniziale indicherà il tipo di requisito individuato e i successivi numeri rappresenteranno una gerarchia.					
			
			\item \textbf{Descrizione:} dettaglio della necessità del requisito.
			\item \textbf{Priorità:} che potrà essere obbligatoria o facoltativa.
			\end{itemize}
			Nel documento di Analisi dei requisiti 3.0
			I requisiti dovranno essere dettagliati  e suddivisi come segue:
			\begin{itemize}
			\item[-] \textbf{Requisiti Funzionali}
			\item[-] \textbf{Requisiti di qualità}
			\item[-] \textbf{Requisiti QML}
			\item[-] \textbf{Requisiti di sistema}
			\end{itemize}

	
			\subsubsubsection{ Strumenti per il tracciamento dei requisiti}
			Per il tracciamento dei requisiti verrà usato lo strumento \inglos{Astah}  nella sua versione gratuita per studenti e per l'utilizzo del quale ogni membro si dovrà registrare usando la mail rilasciata dall'Università degli studi di Padova per ottenere una licenza.  
			
		\subsubsection{ Progettazione}
			\subsubsubsection{ Scopo del processo}
			Tra gli obiettivi di questo processo abbiamo la stesura di documenti, quali la Specifica Tecnica e la Definizione di Prodotto, e soprattutto una vasta produzione di diagrammi \inglos{UML} che modellino i requisiti individuati in fase d'analisi. Di seguito vengono elencate le norme a carico dei progettisti.
			\subsubsubsection{ Descrizione}
			La progettazione deve in modo dimostrabile rispettare tutti i requisiti che il gruppo ha concordato con il committente. In particolare i componenti progettati devono essere tracciabili rispetto al requisito che soddisfano.
			\subsubsubsection{ Diagrammi UML} \label{subsec:UML}
			UML2.0 rimane il linguaggio da usare per i seguenti diagrammi:
			\begin{itemize}
			\item\textbf{Diagrammi dei package:} dovranno essere presenti sia per l'architettura generale che di dettaglio, sarà fondamentale per definire i moduli all'interno del framework \inglos{NodeJS} richiesto dal capitolato;
			\item\textbf{Diagrammi delle classi:} qualora il progetto utilizzasse delle classi, i diagrammi delle classi dovranno essere presenti sia per l'architettura generale che di dettaglio. Nell'ambiente \inglos{NodeJS} a prima vista sembra che siano poco utilizzate, a favore dei package.
			\item\textbf{Diagrammi di sequenza:} utilizzati per definire le interazione tra l'utente e il sistema;
			\item\textbf{Diagrammi di attività:} per evitare di compiere errori verrà utilizzato per facilitare la verifica dei passi dell'algoritmo da implementare ;
			\end{itemize}
			Per la realizazzione dei diagrammi verrà utilizzato \inglos{StarUML}, che è un software di modelazzione open source compatibile con la norme UML2.0. Per maggiori dettagli sul prodotto si farà riferimento alla guida ufficiale disponibile all'indirizzo \url{http://docs.staruml.io/en/latest/}.
			\subsubsubsection{ Stile di progettazione}
				\begin{itemize}
				\item La progettazione dovrà usare quanto più possibile \inglos{design pattern} globalmente affermati, la loro scelta dovrà essere giustificata;
				\item Suddividere il progetto in moduli, in accordo con lo stile di progettazione dell'ambiente \inglos{NodeJS};
				\item Non utilizzare codice sincrono per operazioni di input e di output.
				
				\end{itemize}
		\subsubsection{Specifica tecnica - ST}
		In fase di progettazione architetturale i progettisti, con il supporto degli analisti, hanno il compito di definire nel documento \textbf{ST}  l'architettura ad alto livello dell'intero sistema ma anche dei singoli componenti. Questo viene fatto principalmente utilizzando i diagrammi adottati ed elencati in §\ref{subsec:UML}. Viene inoltre fornito un tracciamento per le componenti che associa ad ognuno di loro almeno un requisito. 
		\subsubsection{Definizione di progetto - DP}	
		In fase di progettazione di dettaglio, i progettisti hanno il compito di redarre il documento \textbf{DP} che amplia quanto riportato in \textbf{ST} definendo nel dettaglio le singole componenti del sistema sempre usando i diagrammi elencati in §\ref{subsec:UML}. Oltre al tracciamento delle componenti, vengono progettati in questa fase anche i test di unità da descrivere nel documento \textbf{PQ}. A seguito della stesura di questo documento, potrà iniziare la fase di codifica.
		\subsubsection{Codifica}
			\subsubsubsection{Scopo del processo}
			Il processo di codifica ha come obiettivo la trasposizione in codice sorgente del software progettato in Specifica Tecnica e Definizione di Prodotto. Inoltre è necessaria la stesura di documentazione sul codice per assicurarne un buon grado di leggibilità e manutenibilità.
			\subsubsubsection{Descrizione}		
				Questo processo porta a un software stabile, affidabile, funzionale e aderente con le richieste del proponente. La codifica verrà eseguita utilizzando gli strumenti in §\ref{s:strum}. 
			
			\subsubsubsection{Intestazione dei file}		
			A seconda del linguaggio di programmazione usato, ogni file di codice dovrà avere, in forma di commento, la seguente intestazione:		
			\lstinputlisting[language=C]{../files/esempio.cpp}
			\captionof{figure}{Esempio di intestazione di file}
		\begin{itemize}
		\item \textbf{FileName:} nome completo del file ('nome.ext');
		\item \textbf{FilePath:} percorso del file con radice la cartella del progetto;
		\item \textbf{Version:} versione corrente del file;
		\item \textbf{History:} registro delle modifiche del file, composto come segue 
			\begin{itemize}
			\item[-] \textbf{Author:} autore della modifica con il ruolo tra parentesi;
			\item[-] \textbf{Date:} data della modifica;
			\item[-] \textbf{Version:} versione della modifica;
			\item[-] \textbf{Description:} descrizione della modifica.
			\end{itemize}	
		\end{itemize}
		
		\subsubsubsection{Formattazione}
		La formattazione del codice sorgente deve essere definita in modo rigoroso e consistente, così
che tutto il codice sia formattato in maniera chiara e attendibile. A questo scopo si è scelto di usare come riferimento per i file \inglos{Javascript}  la linea guida della \textbf{Jquery Foundation} reperibile all'indirizzo \url{http://contribute.jquery.org/style-guide/js/}.
	
	\subsubsection{ Tecnologia}
	Per l'implementazione del prodotto abbiamo scelto il \inglos{framework} \inglos{MeteorJS}- che integra \inglos{NodeJS} (come richiesto dal committente) insieme al il database \inglos{MongoDB} - che si occuperà della gestione delle comunicazioni tra client e server, mentre il \inglos{frontend} verrà gestito con il \inglos{framework} \inglos{materializeCss} per l'aspetto grafico e in mezzo avremmo  \inglos{AngularJS} che ci permetterà di legare il model con la view .
	
	\subsubsection{Verifica e validazione}
	Per la \textbf{verifica} di un documento viene scelto dal responsabile di progetto uno tra i membri del gruppo nel ruolo di verificatore, senza suscitare nessun conflitto di interesse, ossia senza che colui designato a verificare un determinato documento abbia partecipato alla sua stesura.
	 Le verifiche automatizzate sui documenti, dove previste, difficilmente sono esaustive e possono tralasciare delle anomalie. Per eseguire la verifica è necessario controllare che il documento rispetti tutte le norme descritte nelle Norme di Progetto. \\
	La \textbf{validazione} del documento consiste nel controllare che il documento abbia il giusto contenuto, è un compito che va oltre la semplice verifica delle norme. Richiede una conoscenza a priori del contenuto e degli scopi del documento.
\subsubsubsection{Strumenti di verifica} 

Per l'analisi statica del codice \inglos{JavaScript} prodotto, si intende usare lo strumento software \textbf{\inglos{Complexity-report}}, fornito dalla piattaforma \inglos{GitHub}.
Esso fornisce una valutazione delle metriche sull'intero progetto analizzando ogni singolo file e fornendo un report in formato testuale.
Per una descrizione dettagliata delle metriche utilizzate per l'analisi statica del codice si rimanda al documento \textit{piano\_di\_qualifica\_3.0.pdf}.
		
	\section{Processi di supporto}
		\subsection{Processo di documentazione}
		
			\subsubsection{Scopo del processo}
			Questo processo ha lo scopo di definire degli strumenti per rendere coerente e uniforme l'insieme della documentazione prodotta nel ciclo di vita del software.
			
			\subsubsection{Procedure}
			Di seguito verranno presentate le regole e procedure che ogni componente di \textbf{\emph{Team404}} dovrà seguire nella redazione e la manutenzione  della documentazione di progetto. Il team ha scelto il linguaggio di markup \textbf{\LaTeX} per la  stesura dei suoi documenti per vari motivi:
			\begin{itemize}
			\item Permette una facile gestione degli indici e dei glossari.
			\item Dispone di  un sistema di "impaginazione" di alta resa tipografica.
			\item Permette di personalizzare comandi da usare in seguito.
			\item Dispone di numerose librerie con nuovi comandi.
	\end{itemize}		 
			\subsubsection{Template}
			Sarà creato un file \textit{template.tex}  che servirà di struttura iniziale a tutti i documenti in modo da garantire uniformità tra di loro.
			\subsubsection{Struttura dei documenti}
			\subsubsubsection{Comandi personalizzati}
			All'interno del file \textit{'template.tex'} verranno creati alcuni comandi generici da usare in modo da mantenere la stessa formattazione, quali:
			\begin{itemize}
			\item \textbf{\textbackslash addglos:} per indicare che la parola è nel glossario aggiungendo una 'G' a pedice alla parola.
			\item \textbf{\textbackslash beginglos:} per iniziare una sezione di termini del glossario.			
			\item \textbf{\textbackslash beginregistro:} per inserire una riga nel registro delle modifiche.  
			\item \textbf{\textbackslash fineglos:} per chiudere una sezione di termini del glossario. 
			\item \textbf{\textbackslash fineregistro:} per chiudere la tabella del registro delle modifiche.
			\item \textbf{\textbackslash introtab:}  per inserire la tabella di informazioni sul documento.
			\item \textbf{\textbackslash itemglos:} per inserire un termine e la sua descrizione in una sezione del glossario.
			\item \textbf{\textbackslash rigaregistro:} per inserire una riga nel registro delle modifiche. Ammette 4 parametri elencati in sezione.
			\item \textbf{\textbackslash subsubsubsection:}  per rendere disponibile la possibilità di usare una sotto sezione di terza livello.
			%\item \textbf{\textbackslash } 
			\end{itemize}
			\subsubsubsection{Frontespizio}
				Nella prima pagina verrà inserito il frontespizio che sarà costruito nel secondo ordine: 
				\begin{enumerate}
					\item \textbf{ Nome del progetto:} centrato e con una dimensione pari a 32 pt; 
					\item \textbf{ Logo del Gruppo:}  Il logo del gruppo è nel file "team\_not\_found.jpg" che si trova nella cartella principale dei documenti.
					\item \textbf{ Indirizzo email:} centrato e con una dimensione pari a 16 pt;
					\item \textbf{ Nome del documento:} centrato, in maiuscoletto, corredato dalla versione e con una dimensione pari a 24 pt; 
					\item \textbf{ Una tabella con informazioni generali del documento: }
					\begin{itemize}
						\item Nome del documento;
						\item Versione del documento;
						\item Data di redazione: deve essere indicata secondo il formato [ISO 8601];
						\item Redazione: elenco in ordine alfabetico dei redattori del documento;
						\item Verifica: elenco in ordine alfabetico dei Verificatori del documento;
						\item Approvazione: viene indicato il soggetto responsabile di aver approvato il documento;
						\item Uso: interno o esterno;
						\item Distribuzione: elenco in ordine alfabetico dei soggetti a cui verrà distribuito il documento in oggetto. 
					\end{itemize}
					\item \textbf{ Sommario:} brevissimo riassunto indicante lo scopo del documento.
				\end{enumerate}
				\subsubsubsection{Registro delle modifiche}
				Nella pagina successiva al frontespizio va inserito il registro delle modifiche per tener traccia di tutte le modifiche che verranno introdotte nel documento. Questo verrà fatto con una tabella formata come segue:
				\begin{itemize}
					\item \textbf{Versione:} la versione del documento dopo la modifica;
					\item \textbf{Autore:} l'autore della modifica con il ruolo tra parentesi;
					%\item \textbf{Ruolo:} il ruolo dell'autore nel gruppo;
					\item \textbf{Data:} la data della modifica;
					\item \textbf{Descrizione:} una breve indicazione della modifica effettuata.
				\end{itemize}
				\subsubsubsection{Indice}
				Per agevolare la consultazione e permettere una lettura ipertestuale e non necessariamente sequenziale, verrà creato una tabella dei contenuti in modo da facilitare la navigazione all'interno del documento.
				
				\subsubsubsection{Intestazione}
				Ogni pagina del documento ad eccezione della prima avrà un intestazione formata nel modo seguente:
				\begin{itemize}
				\item Logo del  gruppo a sinistra;
				\item Nome del progetto affianco al logo;
				\item Numero e titolo della sezione corrente a destra.
	\end{itemize}			
				\subsubsubsection{Piè di pagina}
				 Come per l'intestazione ogni pagina ad eccezione della prima avrà un piè di pagina formato nel modo seguente:
				 \begin{itemize}
				 \item L'e-mail del gruppo a sinistra;
				 \item Numero progressivo di pagina a destra.
	\end{itemize}			  
			\subsubsection{Versionamento}
			Ogni documento prodotto deve essere versionato in modo che chiunque lo utilizzi possa avere una cronologia delle sue modifiche. Ad ogni versione corrisponde una riga nel registro delle modifiche. Per numerare le versioni verrà adottata la forma: \
			\begin{center} \textbf{v.X.Y.Z} \end{center}
			
			dove \textbf{Z}:
			\begin{itemize}
			\item Parte da 0 e non è limitato superiormente;
			\item Viene incrementata ad ogni modifica del documento.
			\end{itemize}
			dove \textbf{Y}:
			\begin{itemize}
			\item Parte da 0 e non è limitato superiormente;
			\item Viene incrementato dal verificatore ad ogni verifica;
			\item Quando viene incrementato, \textbf{Z} viene azzerato.
			\end{itemize}
			dove \textbf{X}:
			\begin{itemize}
			\item Parte da 1 ed è limitato superiormente dal numero di revisioni;
			\item Viene incrementato dal Responsabile del gruppo quando approva il documento;
			\item Quanto viene incrementato, \textbf{Y} e \textbf{Z} vengono azzerati.
			\end{itemize}
			Le righe di registro dovranno apparire in ordine decrescente dall'ultima modifica fatta.
			
			\subsubsection{Norme tipografiche}
				\subsubsubsection{Formattazione del testo}
				\begin{itemize}
				\item \textbf{Grassetto:} da usare per i titoli, per gli elementi di un elenco. Verrà usato anche per le parole significative;
				\item \textbf{Glossario:} viene aggiunta una lettera '\textbf{G}' a pedice a tutte le parole che hanno una corrispondenza nel glossario;
				\end{itemize}
				\subsubsubsection{Punteggiatura}
				\begin{itemize}
				\item \textbf{Maiuscolo:} la lettera iniziale maiuscola va utilizzata solo per i nomi propri o dopo i segni di punteggiatura punto, punto interrogativo e punto esclamativo;
				\item \textbf{Numeri:} per rappresentare i numeri verrà usato lo standard del sistema internazionale SI/ISO 31 che usa lo spazio come separatore delle migliaia e la virgola come separatore decimale (ad esempio 1 234 567,89); 
				\end{itemize}
				\subsubsubsection{Composizione del testo}
				\begin{itemize}
				\item \textbf{Elenco puntato:}
					\begin{itemize}
					\item[-] La prima parola va sempre con la lettera maiuscola;
					\item[-] La prima parola va in grassetto se è seguita da una descrizione della parola stessa;
					\item[-] Ogni punto dell'elenco deve terminare con un punto e virgola, tranne l'ultimo che deve terminare con un punto.
					\end{itemize}
				\end{itemize}
					
			\subsubsubsection{Formati}
			\begin{itemize}
				\item \textbf{Orario}: sarà usato lo standard ISO 8601 per rappresentare l'orario nel formato HH:MM dove:
					\begin{itemize}
					\item[-]HH: indica le ore, da scrivere sempre con 2 cifre;
					\item[-]MM: indica le minuti, da scrivere sempre con 2 cifre.
					\end{itemize}
				\item \textbf{Data:}  per agevolare la leggibilità, per la data verrà usato il formatto GG/MM/AAAA inverso allo standard dove:
				\begin{itemize}
				\item[-]GG: indica il giorno;
				\item[-]MM: indica il mese;
				\item[-]AAAA: indica l'anno.
				\end{itemize}
				\item \textbf{Link:} i link ipertestuali dovranno essere di colore blu.
			\end{itemize}
			\subsubsection{Sigle}
			Sono stare adottate le seguente sigle:
			\begin{itemize}
			\item \textbf{AR} -> Analisi dei Requisiti;
			\item \textbf{PP} -> Piano di progetto;
			\item \textbf{NP} -> Norme di progetto;
			\item \textbf{SF} -> Studio di fattibilità;
			\item \textbf{PQ} -> Piano di qualifica;
			\item \textbf{ST} -> Specifica tecnica;
			\item \textbf{MU} -> Manuale utente;
			\item \textbf{DP} -> Definizione di prodotto;
			\item \textbf{RR} -> Revisione dei requisiti;
			\item \textbf{RP} -> Revisione di progettazione;
			\item \textbf{RQ} -> Revisione di qualifica;
			\item \textbf{RA} -> Revisione di accettazione.
			\end{itemize}
			\subsubsection{Componenti grafiche}
			\subsubsubsection{Tabelle}
				Ogni tabella nei documenti dovrà avere sotto di essa un numero identificativo e una didascalia. Inoltre dopo l'indice di ogni documento dovrà apparire un elenco delle tabelle inserite.
				\subsubsubsection{Immagini}
					Le immagini da inserire in un documento dovranno essere raggruppate in una sotto cartella della cartella contenente il documento stesso; devono essere ben distanziate dai paragrafi che la precedono e da quelli che la seguono ed essere centrate orizzontalmente \\ \\ 
					 Inoltre dopo l'indice di ogni documento dovrà figurare un elenco delle figure e delle immagini presenti. Gli strumenti scelti per la produzione dei documenti sono elencati in §\ref{doc:prodoc}
			\subsubsection{Elenco dei documenti}
			Per la realizzazione del progetto Quizzipedia viene  prodotta una serie di documenti tali:
			 
			\begin{itemize}
			
				\item  \textbf{Studio di Fattibilità:} riporta lo studio e le motivazioni che hanno portato alla scelta del capitolato \inglos{Quizzipedia}
			 
				\item  \textbf{Norme di Progetto:}	descrivere le regole e procedure adottate dal gruppo per la realizzazione del capitolato. 	
			
				\item  \textbf{Piano di Progetto:}	definisce le scelte progettuali significative adottate, le problematiche d'interesse e un adeguata offerta tecnico-economica relativa al progetto da presentare al committente.
							
				\item  \textbf{Piano di Qualifica:}	definisce le strategie di verifica qualitative dei processi che coinvolgono l'attuazione di modelli standard e l'utilizzo di metriche  e misure necessarie alla valutazione oggettiva del processo o prodotto in analisi.
				
				\item  \textbf{Analisi dei requisiti:} dallo studio del capitolato d'appalto, individua e definisce chiaramente i requisiti che faranno cardine alla software \inglos{Quizzipedia}.
				
				\item  \textbf{Specifica Tecnica:} descrive l'architettura software sulla quale verrà sviluppata \inglos{Quizzipedia}.									
			
				\item  \textbf{Definizione di Progetto:} descrive nel dettaglio la struttura del sistema Quizzipedia.
			
				\item  \textbf{Glossario:} contiene la descrizione estesa dei termini utilizzati nei vari documenti.
			
				\item  \textbf{Manuale Utente:} destinata all'utente finale, fornisce una guida dettagliata delle funzionalità di \inglos{Quizzipedia}
			
				\item  \textbf{Verbali:} contiene un resoconto delle incontri e riunioni del gruppo o del gruppo con il committente. 
			
			\end{itemize}
			
						
			\subsection{Processo di verifica}
				\subsubsection{Scopo del processo}
					Questo processo ha per scopo di confermare che ogni componente software sviluppato sia conforme ai requisiti.		
				\subsubsection{Risultati osservabili}
				Il processo di verifica produce i seguenti risultati:
				\begin{itemize}
					\item l'individuazione e l'applicazione di una strategia di verifica;
					\item i criteri per la verifica di quanto prodotto dal software sono identificati;
					\item le attività di verifica sono state svolte;
					\item i difetti sono stati individuati e catalogati;
					\item i risultati della verifica sono resi disponibili ai proponenti.
				\end{itemize}
				
				\subsubsection{Tecniche di analisi}
				\subsubsubsection{Analisi statica}
				Tale tecnica è attuabile sia alla documentazione sia al codice, consiste nell'individuazione di errori ed anomalie ad esempio effettuando una lettura critica del testo a largo spettro oppure più mirata. Il Verificatore controllerà i documenti e il codice utilizzando le seguenti tecniche:
				\begin{itemize}
				\item[-]\textbf{\inglos{Walkthrough}:} Consiste in una lettura del documento/codice cercando errori ed anomalie a largo spettro senza un'idea precisa di quali tipi errori sarà possibile trovare. Ogni difetto rilevato sarà discusso con gli autori allo scopo di evitare incomprensioni e concordare sulle modifiche necessarie. L'utilizzo è previsto principalmente durante lo sviluppo iniziale, quando non si possiede ancora una chiara visione sui possibili errori. Un uso ripetuto di questa tecnica rende possibile la stesura di una lista di controlla. 		
				
				\item[-]\textbf{\inglos{Inspection}:} si basa sulla lettura mirata dei documenti/codice. Durante tale lettura si cercano gli errori segnalati nella lista di controllo. Progressivamente, con l'acquisizione di esperienza e grazie al precedente uso della tecnica di \inglos{walkthrough}, la lista di controllo viene estesa e specializzata, rendendo l'\inglos{inspection}  sempre più efficace;
				\end{itemize}
				
				\subsubsubsection{Analisi dinamica}
				L'analisi dinamica si applica solamente al prodotto software e viene svolta durante l'esecuzione del codice mediante l'uso di appositi test per verificarne il funzionamento e rilevare possibili difetti d'implementazione. Affinché i test siano utili, devono essere ripetibili e devono trovare errori: infatti ogni test ha un costo, e un test che non aiuti a trovare dei difetti non ha motivo di esistere. I test sono ripetibili se, dati gli stessi dati in input, lo stesso ordine di esecuzione, lo stesso hardware e lo stesso software, ottengo in uscita gli stessi risultati;
				
				\subsubsection{Test}
								
				\subsubsubsection{Test di unità}
				Il test di unità verifica che ogni singola unità di prodotto software funzioni correttamente.
Attraverso questo test si verifica la correttezza di tutti i moduli base che compongono i software andando
a limitare gli errori di implementazione.
				\subsubsubsection{Test di integrazione}
				Il test di integrazione verifica che due o più moduli precedentemente verificati (con test di unità), una volta assemblati, funzionino ed interagiscano come previsto. Aiuta inoltre a rilevare eventuali difetti residui dei moduli, non individuati nella precedente fase di test. Questo test verifica inoltre l'interazione dei moduli prodotti con componenti software esterne come framework o librerie. È inoltre possibile che siano utilizzate delle componenti fittizie (driver e stub ), impiegate al posto di determinati moduli (già pronti o ancora in costruzione) che abbiano un comportamento "sempre corretto" rispetto al contratto dei moduli che sostituiscono: questo permette di effettuare test sul corretto funzionamento dell'intero sottosistema e delle singole parti sviluppate.
				\subsubsubsection{Test di sistema}
				I test di sistema consistono nella validazione dei prodotti software. Questo test viene eseguito quando si ritiene il prodotto giunto ad una versione definitiva. Viene quindi verificata la completa copertura dei requisiti da parte del prodotto.
				\subsubsubsection{Test di regressione}
				Questo test viene eseguito subito dopo che una componente viene modificata. Consiste nel rieseguire tutti i test per verificare che dopo le modifiche il resto dei moduli continuino a funzionare in modo corretto. Il tracciamento aiuta a capire quali sono i test da ripetere (di ogni tipo) poiché potenzialmente a rischio in caso di modifica.
				\subsubsubsection{Test di accettazione}
				Coincide con il collaudo del software in presenza del proponente. Se tale test ha esito positivo, il prodotto sarà considerato sufficientemente maturo da permetterne il rilascio.
		\pagebreak
	\section{Processi Organizzativi}
		\subsection{Scopo del processo}
		Questo processo a come scopo di strutturare e facilitare l'organizzazione interna del gruppo stabilendo norme e modalità di comunicazione e di interazione.
		
		\subsection{Processo di gestione}
		il processo di gestione di uno o più processi ed include le
attività ed i tasks che vanno dalla definizione alla
implementazione di un processo, dalla pianificazione alla
chiusura
		\subsubsection{Ruoli di progetto}
				\subsubsubsection{Definizione dei ruoli}
		Durante lo sviluppo di questo progetto vi sono diversi ruoli che  definiscono le responsabilità e i compiti da effettuare. Ogni membro del gruppo è tenuto a ricoprire tutti i ruoli almeno una volta. Vediamo nel dettaglio i vari ruoli:
		
		\subsubsubsection*{Responsabile: }È il responsabile ultimo, per conto del suo gruppo, dei risultati del progetto.
Elabora ed emana piani e scadenze, ed approva l'emissione di documenti. 
Coordina le attività del gruppo, relazionandosi con il controllo di qualità interno al progetto. 
Redige Organigramma e Piano di Progetto, ed approva inoltre l'Offerta ed i relativi allegati.
			\subsubsubsection*{Amministratore: }È responsabile dell'efficienza e dell'operatività dell'ambiente di sviluppo, della redazione e attuazione di piani e procedure di gestione per la qualità.
Controlla versioni e configurazioni del prodotto e gestisce l'archivio della documentazione di progetto. 
Collabora alla redazione del Piano di Progetto e nel contempo redige le Norme di Progetto per conto del Responsabile.
			\subsubsubsection*{Analista: }È responsabile delle attività di analisi. 
Redige lo Studio di Fattibilità (documento interno al gruppo) e l'Analisi dei Requisiti.
			\subsubsubsection*{Progettista: }È responsabile delle attività di progettazione. 
Redige Specifica Tecnica, Definizione di Prodotto e la parte programmatica del Piano di Qualifica.
			\subsubsubsection*{Programmatore: }È responsabile delle attività di codifica miranti alla realizzazione del prodotto e delle componenti di ausilio necessarie per l'esecuzione delle prove di verifica e validazione.
			\subsubsubsection*{Verificatore: }È responsabile delle attività di verifica.
Redige la parte retrospettiva del Piano di Qualifica che illustra l'esito e la completezza delle verifiche e delle prove effettuate secondo il piano.
		
				
		
		\subsubsection{Comunicazione}
			\subsubsubsection{Interna}			
			Per le comunicazioni ad uso interno del gruppo è stato creato un gruppo sull'\inglos{App} mobile \inglos{Whatsapp} per una rapida interazione tra i membri. Inoltre verrà usato il web \inglos{tool} \inglos{Trello}, il cui uso è spiegato nella sezione §\ref{subsec:ticketing}.
			\subsubsubsection{Esterna}
			Il Responsabile di progetto sarà l'unico che potrà mantenere i contatti con individui esterni al gruppo, per farlo è stata creata una casella di posta elettronica: \url{team404swe@gmail.com}. Tutte le comunicazioni con il committente dovranno essere effettuate con questo indirizzo e-mail. Sarà suo dovere informare successivamente, gli altri componenti del gruppo di tutte le informazioni scambiate con il committente.
		\subsubsection{Incontri}
		\subsubsubsection{Interni}
		In caso di necessità, uno o più componenti del gruppo possono fare richiesta di una riunione al responsabile del progetto. Sarà compito di quest'ultimo autorizzarla o meno. 
		\subsubsubsection{Esterni}
		Sarà compito del Responsabile di progetto fissare incontri con il proponente o committente e informare tutti i componenti del gruppo. Ogni riunione comprenderà la stesura di un verbale ufficiale contenete le seguenti informazioni:
\begin{itemize}
\item Data e ora;
\item Luogo;
\item Partecipanti esterni;
\item Partecipanti interni;
\item Domande e risposte.
\end{itemize}
		
		\subsubsection{Ticketing} 	\label{subsec:ticketing}
		Per la gestione dei \inglos{task} relativi al progetto, il gruppo ha scelto il \inglos{tool} di gestione di progetti \inglos{Trello} con il quale è possibile creare, assegnare, seguire e commentare delle liste composte
da una o più etichette personalizzabili a secondo delle necessità e suddividendole anche in categoria. Così abbiamo creato una categoria \inglos{Opens Tickets} per i tickets aperti e \inglos{Closed tickets} per quelli chiusi.Ogni tickets sarà rappresentato da un'etichetta che sarà parametrizzata nel modo seguente:

	
		\begin{enumerate}
		\item \textbf{Titolo:} riassume in modo conciso il \inglos{task};
		\item \textbf{Membri:} sono i membri ai quali viene assegnato il task;
		\item \textbf{Label:} identifica il tipo di \inglos{task} e gli assegna un colore;
		\item \textbf{Data di scadenza:} indica la data entro la quale il task deve essere completato;
		\item \textbf{Descrizione:} è una descrizione più ampia che può essere inserita per spiegare il compito da svolgere;
		\item \textbf{Allegato:} permette di allegare un file al task;
		\item \textbf{Stato:} rappresenta lo stato del task che può essere successivamente 'Accettato', 'Eseguito' e poi 'Verificato' ed include anche una barra di progressione da 0 a 100\%; 
		\item \textbf{Commenti:} è una funzionalità che permette ai membri di scambiare parere e informazioni riguardanti il task.
		
	
		\end{enumerate}
				% da sistemare perché 
		\begin{figure}[htbp]
		\centering
		\includegraphics[scale=0.50]{../images/tuto2.png}
		\caption{Esempio di configurazione di un task su Trello}			
			
		\end{figure}
			Inoltre, ogni componente del team è tenuto a installare la versione mobile di \inglos{Trello} per ricevere in tempo reale notifiche relative al progetto in modo da poter agir di conseguenza.


	
		\subsection{Processo di Allestimento}
		Il processo di allestimento e tenuta in esercizio di tutte le
infrastrutture e servizi necessari allo sviluppo di uno o più
processi
		\subsubsection{Strumenti di progetto} \label{s:strum} 	
	
		\subsubsubsection{Sistema operativo}
		La scelta del sistema operativo da utilizzare sarà a discrezione di ogni componente del gruppo, dato che non influirà sul funzionamento dell'applicazione.
		\subsubsubsection{Codifica dei caratteri}
		Per assicurarsi la corretta visualizzazione dei caratteri accentati tutti i file testuali presenti nel \inglos{repository} (sia codice sorgente che documentazione) devono essere memorizzati con la codifica\textbf{ \inglos{UTF-8}.}
		\subsubsubsection{Versionamento}
		Per il versionamento dei documenti e del codice sorgente è stato creato un account per il gruppo sul servizio di hosting \inglos{GitHub} con nome utente \textbf{team404swe} con tutti i membri del gruppo come contributori. Nell'account sono stati creati le \inglos{repository} \textbf{"team404swe/Documenti"} per i documenti e \textbf{"team404swe/Quizzipedia"} per il codice sorgente.Quest'ultimi verranno aggiornati in tutta sicurezza attraverso  il software \textbf{\inglos{GitHub Desktop}} dai sistemi compatibili (\inglos{Windows} e \inglos{Mac OSX})per la gestione delle \inglos{repository} \inglos{GitHub}. 
				
		\subsubsubsection{Git - Modo d'uso} \label{git_modo} 
		Qualche comandi utili per prendere in mano \inglos{Git}.
		\begin{enumerate}			
			\item Per creare una copia di un repository locale
			\begin{enumerate} \item[] \texttt{\$ git clone /percorso/del/repository}\end{enumerate}
			\item Per creare una copia di un repository locale usando invece un server remoto 
			\begin{enumerate} \item[] \texttt{\$ git clone nomeutente@host:/percorso/del/repository} \end{enumerate}
			\item Per proporre modifiche al file <nomedelfile>		
			\begin{enumerate} \item[] \texttt{\$ git add <nomedelfile> } \end{enumerate}
			\item Per proporre modifiche a tutti i file
			\begin{enumerate} \item[] \texttt{\$ git add * }\end{enumerate}
			\item Per validare queste modifiche proposte 
			\begin{enumerate} \item[] \texttt{\$ git commit -m "Messaggio per la commit"}  \end{enumerate}
			\item Per inviare queste modifiche al repository remoto
			\begin{enumerate} \item[] \texttt{\$ git push origin master }\end{enumerate}
			\item Per aggiornare il tuo repository locale alla commit più recente	
			\begin{enumerate} \item[] \texttt{\$ git pull } \end{enumerate}
		\end{enumerate}
			\subsubsubsection{GitHub - Modo d'uso}
			GitHub Desktop essendo molto intuitivo, si farà riferimento direttamente alla guida ufficiale disponibile all'indirizzo: https://guides.github.com/introduction/getting-your-project-on-github/  o direttamente tramite riga di comando come in §\ref{git_modo} 
			\subsubsection{Openshift online} \label{s:openshift}
			Per ospitare il nostro software online abbiamo scelte la piattaforma \textbf{\inglos{OpenShift} Online } che mette a disposizione, grazie all'account creato con la mail del gruppo, una vasta gamma di linguaggi e servizi distribuiti in applicazioni che possono anche essere \inglos{framework} tra i quali \inglos{MeteorJS} che useremmo per il nostro progetto.\\
			Questa piattaforma ci permette inoltre di lavorare direttamente in locale, salvando di volta in volta il lavoro in una apposita \inglos{repository} direttamente da riga di comando attraverso l'applicazione di supporto di \inglos{RedHat} \inglos{rhc} che da windows richiede l'installazione dei software \inglos{Ruby} versione 1.9.3 o 2.0.0,   e \inglos{Git}(ultima versione disponibile)  per funzionare correttamente.\\
			Per la procedura ufficiale e dettagliata per windows, seguire le istruzioni al seguente link \url{https://developers.openshift.com/getting-started/windows.html} \\
			
			
			\subsubsubsection{MeteorJS} \label{s:meteor}
			La piattaforma \inglos{NodeJS} Questo \inglos{framework} web basato su \inglos{NodeJS} che produce codice  multi piattaforma. Si integra con \inglos{MongoDB} e utilizza lo stile architetturale \inglos{Publish/Subscribe}. E' compatibile con con qualsiasi libreria  di interfaccia utente e per il nostro progetto viene utilizzato il \inglos{framework} \inglos{AngularJS}.\\
			
			\subsubsubsection{AngularJS}\label{s:angular}
			Il \inglos{framework} \inglos{AngularJS} viene usato per esaltare e potenziare l'approccio dichiarativo dell'\inglos{HTML} nella definizione dell'interfaccia grafica sfruttando la \inglos{Single Page Application} in modo da velocizzare il caricamento delle nostre pagine. Per il nostro progetto, \inglos{AngularJS} è stato installato insieme a \inglos{MeteorJS} sul nostro server presso \inglos{OpenShift}. Una documentazione ricca di esempi e dettagli viene fornita dagli sviluppatori del progetto ed è disponibile a seguente link: \url{}
			
			\subsubsubsection{MaterializeCss - da valutare} 
			Con l'obbiettivo di migliorare l'esperienza dell'utente finale del nostro software, viene utilizzato il framework \inglos{MaterializeCss} creato da Google e basato sul \inglos{Material Design} e compatibile con i maggiori browser. \inglos{MaterializeCss} è un semplice \inglos{framework} di interfaccia utente \inglos{frontend} per la costruzione per la realizzazione di progetti web \inglos{responsive} e \inglos{user-friendly}. \\
			
			
			Per altri riferimenti specifici o per un ampia documentazione, si farà riferimento alla guida ufficiale disponibile all'indirizzo: \url{http://materializecss.com/}
			
			\subsubsection{ Installazione Meteor-Angular-Materialize}
			Per installare \inglos{MeteorJS} sul proprio computer, ogni membro dovrà seguire le seguente istruzioni:
			
			\lstinputlisting[language=C]{../files/Meteor_Angular_Materialize.cpp}
			\captionof{figure}{Istruzioni Meteor-Angular-Materialize}
			
			
			
	\subsubsection{Produzione dei documenti} \label{doc:prodoc}
	
	\subsubsubsection{Dropbox}
	Per la condivisione di file e documenti non soggetti a controllo di versione tra i membri del gruppo. A questo scopo è stata creata una cartella che contiene anche i vari  manuali degli strumenti di interesse per il progetto.
	\subsubsubsection{Editor}
	Come editor per i documenti viene scelto \inglos{TexMaker}, un programma open source per scrivere documenti in \LaTeX   che offre multiple funzionalità quali il supporto dell'unicode, la colorazione sintattica e la correzione autografica nonché la possibilità di esportare il documento in \inglos{HTML} o in formato \inglos{OpenDocument}.
	\subsubsubsection{Correttore ortografico}
	Come correttore ortografico dei documenti prodotti verrà utilizzato il software \textbf{\inglos{MySpell}} integrato nell'editor TexMaker.
	\newpage
	
		
	\end{document}
