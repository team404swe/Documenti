\documentclass[a4paper,11pt]{article}
\input{../../../template.tex}

\graphicspath{ {./Images/} }
\title{\textbf{{\fontsize{10mm}{6mm}\selectfont QUIZZIPEDIA}}}

\makeindex

\begin{document}
	\maketitle
	
	\begin{center}

	\includegraphics{../../../team_not_found.jpg}\\	
	\fontsize{5mm}{3mm}\url{team404swe@gmail.com}\\
	\vspace{40mm}
	\textbf{ Verbale Interno 2015/12/18 }
	\end{center}
	\thispagestyle{empty}	% per togliere il numero in fondo pagina
	\introtab{Verbale Interno 2015/12/18}			%1 nome documento
			{1.0} 							%2 versione
			{Interno} 						%3 Uso
			{8 gennaio 2016} 				%4 Data cre
			{\today} 						%5 Data mod
			{Davide Bortot}					%6 Redazione1
			%{} 							%7 Redazione2
			{Andrea Multineddu} 			%8 Verifica
			{Davide Bortot} 				%9 Approvazione
	
	\newpage
	\section{Sommario}
	Questo documento ha lo scopo di formalizzare l'incontro tra i membri del gruppo \textbf{Team404} avvenuto il 30/05/2015. Scopo dell'incontro era di fare il punto considerando giudizio dei docenti dopo la RP-min e accordarsi sul proseguimento delle attività del progetto. 
	\subsection{Generalità}
	\begin{itemize}
	\item\textbf{Data}: 30 Maggio 2016.
	\item\textbf{Luogo}: Padova, Via Trieste 63, Dipartimento di Matematica Pura ed Applicata dell'Università di Padova, Aula 1C/150.
	\item\textbf{Ora d'inizio}: 10:00.
	\item\textbf{Durata}: 40 minuti.
	\item\textbf{Partecipanti}: Davide Bortot, Luca D'Alessio, Andrea Multineddu e Alex Beccaro.
	\end{itemize}

	\newpage
	\section{Argomenti discussi}
		\subsection{Valutazione del giudizio}
		
		
		
		\subsection{Mezzi di comunicazione e di gestione}
		
		
		
		\subsection{Documentazione}
		
		
		\subsection{Ruoli}
		Si è deciso che durante il primo periodo del progetto i ruoli verranno così suddivisi:
		\begin{table}[h!]			
		\begin{center}
			\begin{tabular}{l c}
			\textbf{Componente} & \textbf{Ruolo}\\
			\midrule
			Davide Bortot & Responsabile\\
			Martin Vadice Mbouenda & Amministratore\\
			Marco Crivellaro & Analista\\
			Alex Beccaro & Analista\\
			Luca Alessio & Analista/Verificatore\\
			Andrea Multineddu & Analista/Verificatore\\
			\midrule
			\end{tabular}
		\end{center}
		\end{table}
		\newline
		Gli ultimi due membri citati dovranno accordarsi su quale ruolo ricoprire; preferibilmente la figura dell'analista dovrebbe essere disponibile a trovarsi fisicamente con gli altri due analisti.
\end{document}