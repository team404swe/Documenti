\documentclass[a4paper,11pt]{article}
\input{../../../template.tex}

\graphicspath{ {./Images/} }
\title{\textbf{{\fontsize{10mm}{6mm}\selectfont QUIZZIPEDIA}}}

\makeindex

\begin{document}
	\maketitle
	
	\begin{center}

	\includegraphics{../../../team_not_found.jpg}\\	
	\fontsize{5mm}{3mm}\url{team404swe@gmail.com}\\
	\vspace{40mm}
	\textbf{ Verbale Interno 2016/06/20 }
	\end{center}
	\thispagestyle{empty}	% per togliere il numero in fondo pagina
	\introtab{Verbale Interno 2016/06/20}			%1 nome documento
			{1.0} 							%2 versione
			{Interno} 						%3 Uso
			{20 giugno 2016} 				%4 Data cre
			{\today} 						%5 Data mod
			{Martin Mbouenda}				%6 Redazione1
			%{} 							%7 Redazione2
			{Maco Crivellaro} 			%8 Verifica
			{Alex Beccaro} 				%9 Approvazione
	 
	\newpage
	\section{Sommario}
	Questo documento ha lo scopo di formalizzare l'incontro tra i membri del gruppo \textbf{Team404} avvenuto il 20/06/2016. Scopo dell'incontro era di fare il punto della situazione considerando la grande coordinazione necessaria alla scrittura del codice, e di effettuare una seduta di lavoro intenso e collaborativo. 
	\subsection{Generalità}
	\begin{itemize}
	\item\textbf{Data}: 20 Giugno 2016.
	\item\textbf{Luogo}: Padova, Via Trieste 63, Dipartimento di Matematica Pura ed Applicata dell'Università di Padova, Aula 1C/150.
	\item\textbf{Ora d'inizio}: 14:00.
	\item\textbf{Durata}: 4 ore.
	\item\textbf{Partecipanti}: Davide Bortot, Alex Beccaro, Andrea Multineddu, Martin  Mbouenda e Marco Crivellaro.
	\end{itemize}

	\newpage
	\section{Argomenti discussi}
		\subsection{Discussione sull'avanzamento del progetto}
		In una discussione aperta, ogni membro ha potuto dare una sua opinione sull'andamento e fare proposte che sono state discusse tra i membri.
		Si è concordato di:
		\begin{itemize}
			\item iniziare la codifica immediatamente dopo la stesura della Definizione di Prodotto, o parallelamente ove possibile.
			\item lavorare su una repository Meteor, in modo che l'applicazione sia testabile in locale. Il caricamento su server dell'applicazione verrà effettuato solo quando necessario.
			\item utilizzare quanto più possibile funzionalità built-in dei framework Meteor e AngularJs per evitare codifica inutile
		\end{itemize}
		\subsection{Riassegnazione dei compiti}
		Sono stati riassegnati i compiti e il lavoro in modo tale da permettere una codifica il più possibile in parallelo. E' stato istituito un sistema di assegnazione del lavoro e ticketing apposito sulla piattaforma Trello, in modo da concordare task e responsabilità univoche.
								
\end{document}