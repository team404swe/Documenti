\documentclass[a4paper,11pt]{article}
\input{../../../template.tex}

\graphicspath{ {./Images/} }
\title{\textbf{{\fontsize{10mm}{6mm}\selectfont QUIZZIPEDIA}}}

\makeindex

\begin{document}
	\maketitle
	
	\begin{center}

	\includegraphics{../../../team_not_found.jpg}\\	
	\fontsize{5mm}{3mm}\url{team404swe@gmail.com}\\
	\vspace{40mm}
	\textbf{ Verbale Interno 2015/12/18 }
	\end{center}
	\thispagestyle{empty}	% per togliere il numero in fondo pagina
	\introtab{Verbale Interno 2015/12/18}			%1 nome documento
			{1.0} 							%2 versione
			{Interno} 						%3 Uso
			{20 giugno 2016} 				%4 Data cre
			{\today} 						%5 Data mod
			{Marco Crivellaro}				%6 Redazione1
			%{} 							%7 Redazione2
			{Martin Mbouenda} 			%8 Verifica
			{Alex Beccaro} 				%9 Approvazione
	 
	\newpage
	\section{Sommario}
	Questo documento ha lo scopo di formalizzare l'incontro tra i membri del gruppo \textbf{Team404} avvenuto il 20/06/2016. Scopo dell'incontro era di fare il punto considerando la grande coordinazione da necessaria nella scrittura del codice e più di una riunione, è una seduta di lavoro intenso e collaborativo. 
	\subsection{Generalità}
	\begin{itemize}
	\item\textbf{Data}: 20 Giugno 2016.
	\item\textbf{Luogo}: Padova, Via Trieste 63, Dipartimento di Matematica Pura ed Applicata dell'Università di Padova, Aula 1C/150.
	\item\textbf{Ora d'inizio}: 14:00.
	\item\textbf{Durata}: 4 ore.
	\item\textbf{Partecipanti}: Davide Bortot, Alex Beccaro, Andrea Multineddu, Martin  Mbouenda e Marco Crivellaro.
	\end{itemize}

	\newpage
	\section{Argomenti discussi}
		\subsection{Discussione sull'avanzamento del progetto}
		In una discussione aperta, ogni membro ha potuto dare una sua opinione sull'andamento e fare prosposte che sono state discusse tra i membri. 
		\subsection{Riassegnazione dei dei compiti}
		Viene fatta una riassegnazione dei compiti dopo una valutazione e discussione  e poi sono stati riassegnati i compiti in modo tale che ognuno avesse da in carico le parti di codice con un certo legame. 
		\subsection{Lavoro collaborativo}
		Alle fine della discussione, il gruppo si è messo al lavoro.
		
		
		
		
\end{document}