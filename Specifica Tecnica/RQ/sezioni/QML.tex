\section{Linguaggio QML}
Il linguaggio QMl è lo strumento scelto per definire le domande del sistema Quizzipedia.
Ogni domanda avrà sempre:
\begin{itemize}
\item un tipo 
\item un teso della domanda
\item un insieme di risposte, alcune giuste, alcune errate
\end{itemize}
Per rappresentare queste informazioni in modo semplice e intuitivo si è deciso di utilizzare la seguente sintassi:
\begin{itemize}
\item  \textit{Definizione della domanda}: \code{<question> *Contenuto* <fine>}\\
i marcatori \code{<question>} e \code{<fine>} definiscono i confini della definizione di una domanda
\item \textit{Definizione del tipo:} \code{ \{TIPO\} }\\
la sintassi \code{ \{TIPO\} }, inserita all inizio della definizione della domanda, ne identifica univocamente il tipo. La dicitura \code{ "TIPO" } sarà una coppia di lettere che identifica il tipo della domanda, ad esempio:
	\begin{itemize}
		\item \textbf{VF}: domanda Vero o Falso
		\item \textbf{MU}: domanda a Risposta Multipla con unica risposta esatta
		\item \textbf{MX}: domanda a Risposta Multipla con più risposte esatte
	\end{itemize}
\item \textit{Definizione del testo}: \code{=> *Testo della domanda*}\\
la sintassi \code{=>} posta dopo la definizione del tipo, delimita la definizione del testo della domanda.
\item\textit{Definizione delle risposte}: cambia da domanda a domanda\\
Le risposte saranno una lista di elementi del tipo 

\code{\newline\{\} Risposta1\newline\{\} Risposta2\newline\{X\} Risposta3\newline\{\} Risposta4} 
\newline

dove l'elemento \code{\{X\}} contiene la risposta esatta, nel caso delle domande a risposta multipla.
\newline

Si potrà avere ad esempio una lista per domande Vero o Falso

\code{\newline\{V\} Risposta1\newline\{F\} Risposta2 }
\end{itemize} 
\pagebreak

Viene riportato un esempio completo di domanda a risposta multipla con più risposte complete, che utilizza tutti gli elementi di cui sopra:
\newline

\noindent
\code{
<question> \{MX\} \newline
=>scegli i frutti. \newline
\{\}  mais \newline
\{X\} mango \newline
\{\}  fagioli \newline
\{X\}  pera \newline
<fine>
}
\newpage