\documentclass[a4paper,11pt]{article}
	\input{../../template.tex}

	\title{\textbf{{\fontsize{8mm}{5mm}\selectfont QUIZZIPEDIA}}}
	\date{}
	\author{}	


	\begin{document}
	\maketitle
	\thispagestyle{empty}
	\begin{center}	
	\includegraphics{../team_not_found.jpg}\\
	\fontsize{5mm}{3mm}\url{team404swe@gmail.com}\\
	
	\vspace{50mm}
	\textbf{Specifica Tecnica 1.0}
	\end{center}
	\introtab{Specifica Tecnica 1.0}			%1 nome documento
			{1.0} 							%2 versione
			{Esterno} 						%3 Uso
			{22 aprile 2016} 				%4 Data cre
			{\today} 						%5 Data mod
			{M. Crivellaro - L. Alessio - D. Bortot}	%6 Redazione
			{Alex Beccaro } 			%7 Verifica
			{Martin V. Mbouenda} 				%8 Approvazione
	\newpage
	\thispagestyle{empty}
	\null  

	\newpage
	\newpage
	\fancyhead[R]{REGISTRO DELLE MODIFICHE}
	\fancyfoot[R]{\thepage}
	
	\hspace{30 mm}
	\section*{Registro delle modifiche}
	
	\beginregistro
	
	\rigaregistro{\textbf{Versione}}{\textbf{Autore}}{\textbf{Data}}		 {\hspace{5 mm} 
	\textbf{Descrizione}}

	\rigaregistro{0.0.12}{Andrea Multineddu (Progettista)}{06/05/2016}{Aggiunte immagini sezione "Diagrammi delle attività".}
	\rigaregistro{0.0.11}{Davide Bortot (Progettista)}{05/05/2016}{Completata la parte descrittiva della sezione "Diagrammi delle classi del Presenter". Riadattamento allo stile del documento di §4.3. Inseriti grafici UML dei Package del Model e della View.}
	\rigaregistro{0.0.10}{Luca Alessio (Progettista)}{04/05/2016}{Stesura sezione "Diagrammi delle attività" e correzioni varie su altre sezioni.}
	\rigaregistro{0.0.9}{Davide Bortot (Progettista)}{03/05/2016}{Stesura sezione "Package della componente Presenter", inserito rispettivo grafico UML, e prima stesura di "Diagrammi delle classi del Presenter".}
	\rigaregistro{0.0.8}{Andrea Multineddu (Progettista)}{01/05/2016}{Prima stesura sezioni "Package della componente Model" e "Diagrammi delle classi del Model".}
	\rigaregistro{0.0.7}{Luca Alessio (Progettista)}{30/04/2016}{Prima stesura sezioni "Package della componente View" e "Diagrammi delle classi del View".}	
	\rigaregistro{0.0.6}{Davide Bortot (Progettista)}{28/04/2016}{Stesura delle sezioni §2.2, §2.3, §2.4 con la descrizione d'alto livello delle componenti MVP.}
	\rigaregistro{0.0.5}{Luca Alessio (Progettista)}{27/04/2016}{Stesura parziale sezione "Tecnologie e strumenti utilizzati"}
	\rigaregistro{0.0.4}{Davide Bortot (Progettista)}{25/04/2016}{Ampliata la sezione "Architettura generale del sistema".}
	\rigaregistro{0.0.3}{Luca Alessio (Progettista)}{24/04/2016}{Stesura parziale sezione "Architettura generale del sistema"}
	\rigaregistro{0.0.2}{Davide Bortot (Progettista)}{23/04/2016}{Strutturazione iniziale del documento e stesura sezione introduttiva}
	\rigaregistro{0.0.1}{Marco Crivellaro (Progettista)}{22/04/2016}{Creazione documento.}
	\fineregistro

	\newpage
	\fancyhead[R]{\leftmark}
	\tableofcontents
	\pagenumbering{Roman}
	\newpage
	\listoffigures
	\listoftables
	
	\newpage
	\pagenumbering{arabic}
	
	\section*{Sommario}
	
	
	\newpage
	\section{Introduzione}
	\subsection{Scopo del documento}
	
	
	\subsection{Scopo del prodotto}
	Il progetto \textbf{Quizzipedia} ha come obiettivo lo sviluppo di un sistema software basato su tecnologie Web (Javascript\addglos, Node.js\addglos, HTML5\addglos, CSS3\addglos) che permetta la creazione, gestione e fruizione di questionari. Il sistema dovrà quindi poter archiviare i questionari suddivisi per argomento, le cui domande dovranno essere raccolte attraverso uno specifico linguaggio di markup (Quiz Markup Language) d'ora in poi denominato QML\addglos. In un caso d'uso a titolo esemplificativo, un "esaminatore" dovrà poter costruire il proprio questionario scegliendo tra le domande archiviate, ed il questionario così composto sarà presentato e fruibile all' "esaminando", traducendo l'oggetto QML in una pagina HTML\addglos, tramite un'apposita interfaccia web. Il sistema presentato dovrà inoltre poter proporre questionari preconfezionati e valutare le risposte fornite dall'utente finale.
	\\
	Per un'analisi più precisa ed approfondita del progetto si rimanda al documento\\ "\textit{analisi\_dei\_requisiti\_2.0.pdf}".
	\subsection{Glossario}
	Viene allegato un glossario nel file ``\textit{glossario\_2.0.pdf}'' nel quale viene data una definizione a tutti i termini che in questo documento appaiono con il simbolo '\addglos' a pedice.
	\subsection{Riferimenti}
		\subsubsection{Normativi}
		\begin{itemize}
			\item Capitolato d'appalto Quizzipedia:\\
			\url{http://www.math.unipd.it/~tullio/IS-1/2015/Progetto/C5.pdf}
			\item Norme di Progetto: "\textit{norme\_di\_progetto\_2.0.pdf}"
		\end{itemize}
		\subsubsection{Informativi}
		\begin{itemize}
			\item Corso di Ingegneria del Software anno 2015/2016:\\
			\url{http://www.math.unipd.it/~tullio/IS-1/2015/}
			\item Regole del progetto didattico:\\
			\url{http://www.math.unipd.it/~tullio/IS-1/2015/Dispense/PD01.pdf}
			\url{http://www.math.unipd.it/~tullio/IS-1/2015/Progetto/}\\
			\end{itemize}
	\pagebreak
	\newpage
	\section{Specifiche del prodotto}
	I contenuti della specifica saranno presentati seguendo l'approccio top-down: dalla descrizione macroscopica del sistema si scenderà sempre più in dettaglio passando alla descrizione delle singole componenti. Verranno inoltre descritti i design pattern utilizzati e come essi sono stati applicati. Per agevolare la comprensione si è scelto di utilizzare i diagrammi dei package, delle classi, di attività e di sequenza, descritti attraverso lo standard UML 2.0.
	
	\subsection{Architettura generale del sistema}
	Il sistema Quizzipedia è di tipo \emph{client-server}; il \emph{client} fornisce all'utente un'interfaccia web su browser per la creazione e fruizione di questionari, mentre il lato \emph{server} si occupa di gestire e salvare i dati su DBMS PostgreSQL. La base di dati raccoglie principalmente quesiti memorizzati in QML, che su richiesta verranno elaborati da un interprete.
Il suddetto interprete processa l'input QML, precedentemente validato nella forma da un parser apposito, traducendolo in linguaggio HTML5 visualizzabile da browser.
Nella realizzazione del sistema Quizzipedia verrà adottato il design pattern \emph{Model View Presenter} nella sua variante \emph{Passive View}.
\begin{figure}[h!]
\begin{center}
	\includegraphics[scale=1.3]{../images/mvp.png}
	\caption{Il design pattern Model View Presenter (Passive View)}
\end{center}
\end{figure}
\begin {itemize}
\item\textbf{Model}: definisce l'organizzazione dei dati e ne specifica le modalità di accesso. Nel sistema Quizzipedia è situato nella parte server che opera sulla sottostante base di dati. Fa parte del Model anche la componente \emph{Parser} che opera sui dati prima che essi vengano salvati nel DBMS.
\item\textbf{View (Passive)}: rappresenta l'interfaccia grafica presentata all'utilizzatore, la quale visualizza i dati e cattura le interazioni dell'utente. L'aggettivo "Passive" indica che la View non è responsabile del proprio aggiornamento al variare del Model, compito che ricade sul Presenter.
\item\textbf{Presenter}: controlla la View e ne gestisce il comportamento in reazione alle interazioni dell'utente, interagisce di conseguenza col Model per ottenere i dati necessari.
Una volta ottenuti i dati si preoccupa di aggiornare la View. Nel sistema Quizzipedia implementa la parte logica, affiancata a quella grafica, del Client. Realizza un totale disaccoppiamento tra Model e View, controllandone i flussi di comunicazione.
	\end {itemize}
	Altre possibili architetture che sono state prese in considerazione sono definite dai design pattern \emph{Model View Controller (MVC)} con \emph{Front Controller}, \emph{Model View Presenter (MVP)} nelle varianti \emph{Presentation Model} e \emph{Supervising Controller}, e il design pattern \emph{Model View ViewModel (MVVM)}.
	\begin{itemize}
	\item La prima pone il Controller, componente simile al Presenter, nella parte server del sistema. Quest'opzione è stata scartata per evitare di aumentare troppo la complessità del lato server, ponendo invece il Presenter dal lato client, così da redistribuire responsabilità e carico di lavoro.
	\item Il \emph{Presentation Model} invece è affine al design pattern scelto, ma impone che sia la componente View ad aggiornarsi autonomamente al variare del Model. Proprio per questo motivo si è deciso di scartarla in favore del Passive View: per disaccoppiare completamente le componenti Model e View, e per alleggerire ulteriormente quest'ultima concentrandone tutta la parte logica nel Presenter.
	\item Il \emph{Supervising Controller} propone che la View si aggiorni autonomamente nel caso di piccole modifiche (tramite data-binding col Model), lasciando le manipolazioni più complicate al Presenter; è stato scartato in favore del disaccoppiamento totale tra View e Model.
	\item Il pattern \emph{MVVM} prevede di creare per ogni View un ViewModel, che rappresenta tutte le informazioni e i comportamenti della corrispondente View. La View si limita infatti, a visualizzare graficamente quanto esposto dal ViewModel, a riflettere in esso i suoi cambi di stato oppure ad attivarne dei comportamenti (tramite data-binding). Tale architettura è indicata per applicazioni particolarmente dinamiche in cui View e Model devono essere costantemente aggiornati. Non è questo il caso di Quizzipedia.
	\end{itemize}
	\subsection{Descrizione del componente Model}
	Il Model, situato nella parte server del sistema svolge le seguenti funzioni:
	\begin{itemize}
		\item Interagisce con un database PostgreSQL nel quale vengono salvati i dati del sistema (ad es. domande, statistiche, utenti). Fornisce quindi adeguate funzionalità di salvataggio e caricamento da database di tali dati.
		\item Al momento del salvataggio di una nuova domanda esegue il \emph{parsing} del codice QML tramite un componente chiamato \emph{Parser}. Se la domanda è sintatticamente corretta può essere salvata nel database.
		\item Offre un'interfaccia logica di accesso al Presenter attraverso la quale richiedere dati e operazioni su di essi. Quest'interfaccia sarà l'unico punto di accesso disponibile al Presenter. Per accentrare le funzionalità d'accesso si farà uso del design pattern \emph{Façade}.
	\end{itemize}
	\subsection{Descrizione del componente View}
	La componente View rappresenta l'interfaccia grafica che visualizza i dati del Model e inoltra i comandi dell'utente (o gli eventi da esso generati) al Presenter che si occuperà di gestire tali richieste sui dati interagendo col Model; la View si occupa quindi solamente della rappresentazione grafica dei dati e non ha alcun contatto diretto con essi. Essendo un'interfaccia web verrà realizzata tramite HTML5 e CSS3 per le parti statiche e con l'utilizzo di Javascript per le parti dinamiche. Per assicurare uno stile coerente tra le pagine web e migliorare l'adattabilità a piattaforme mobile verrà utilizzato il framework \emph{Materialize}.
	\subsection{Descrizione del componente Presenter}
	Il Presenter ricopre tre ruoli fondamentali: recepire ed elaborare gli input dell'utente,
comunicare col Model, ed aggiornare il View con i dati ottenuti. Per poterlo fare possiede le seguenti caratteristiche:
	\begin{itemize}
		\item Conosce i riferimenti alle altre due componenti. Il Presenter è l'unica componente che conosce entrambe le altre e facendo da singolo tramite tra Model e View permette il loro totale disaccoppiamento;
		\item E' in grado di elaborare gli input della View e tradurli in azioni sul Model. Viceversa ad ogni modifica del Model si preoccupa di aggiornare di conseguenza la View.  
		\item E' responsabile della traduzione delle domande dal formato QML a formato HTML visualizzabile da browser, tramite un componente \emph{Interpreter}.  Tale funzionalità viene richiesta ogni qualvolta il Presenter richiede e riceve dal Model una domanda in formato QML.
		\item Possiede dei gestori che possano modificare l'aspetto della View in reazione all'interazione dell'utente o ai dati ricevuti dal Model. Al suo interno il Presenter contiene delle classi che in risposta ad un evento, quale l'interazione dell'utente con la View o il ricevimento di una risposta dal Model, modificano l'aspetto della GUI presentata.
		\item Gestisce la somministrazione di un questionario ad un utente, domanda dopo domanda, fino alla consegna e valutazione.
	\end{itemize}
	\newpage

	\section{Tecnologie e strumenti utilizzati}
	\subsection{HTML5}
	Linguaggio di markup per la progettazione di pagine Web. Richiesto espressamente nel capitolato per la creazione dell'interfaccia utente (mi pare, devo controllare).
	\begin{itemize}
		\item\textbf{Utilizzo}: viene utilizzato per creare la GUI che permette all'utente di accedere al sistema mediante browser.
		\item\textbf{Vantaggi}: favorisce la portabilità su diversi dispositivi (desktop, smartphone, tablet...) e browser.
Sono punti a favore anche l'elevata compatibilità con tecnologie quali CSS3 e Javascript.
		\item\textbf{Svantaggi}: essendo HTML5 un linguaggio non ancora standard il rischio nel suo utilizzo è l'instabilità dei tag utilizzati. Un tag oggi accettato potrebbe essere modificato
in un futuro prossimo, rendendo la visualizzazione dell'interfaccia utente dipendente dal stabilità dei tag utilizzati. 
	\end{itemize}
	\subsection{CSS3}
	Principale linguaggio usato per la formattazione di pagine HTML.
	\begin{itemize}
		\item\textbf{Utilizzo}: Viene utilizzato per formattare il codice HTML,
ovvero creare fogli di stile che permettono all'utente di modificare alcuni aspetti grafici della
pagina Web
		\item\textbf{Vantaggi}: Richiede un minor sforzo di interpretazione da parte del browser ed è leggero da scaricare
		\item\textbf{Svantaggi}: Può presentare problemi di compatibilità con browser meno recenti
	\end{itemize}
	\subsection{Javascript}
	E' un linguaggio di scripting debolmente orientato agli oggetti, utilizzato nelle applicazioni
Web. Viene interpretato all'interno del browser. Permette di definire funzionalità simili a
quelle offerte da C++ e Java, quali cicli e strutture di controllo.
Viene solitamente affiancato a pagine statiche HTML per poter gestire i contenuti dinamicamente, offrendo funzionalità che il linguaggio di markup non può offrire.
	\begin{itemize}
		\item\textbf{Utilizzo}: nel sistema Quizzipedia Javascript rivestirà un ruolo importante, verrà infatti utilizzato nella realizzazione del parser/interprete/boh andrea dimmi te.
		\item\textbf{Vantaggi}: eseguito lato Client (sul browser) non sovraccarica il Server per l'esecuzione di richieste, anche se lo script è complesso.
		\item\textbf{Svantaggi}: per script sorgenti molto corposi, può risultare oneroso in termini di
tempo lo scaricamento dei contenuti. Deve, inoltre, fare affidamento ad un linguaggio
che possa fisicamente effettuare transazioni di dati quando lo script esegue operazioni
su oggetti remoti (eg: database).
E' altresì un linguaggio non tipizzato, quindi occorre porre attenzione ai tipi delle
variabili che non sono dichiarati, ma variano dinamicamente.
		\item\textbf{Variabili e oggetti}: le variabili se sono dichiarate all'interno di una funzione sono
visibili solo all'interno di essa; se sono invece esterne sono globali. Vengono dichiarate
con la keyword \emph{var} o semplicemente assegnando loro un valore.
Ogni elemento in JavaScript è un tipo primitivo o un oggetto.
Gli oggetti sono entità dotate di unicità (sono uguali solo a sé stessi) e identificabili
con vettori associativi, che associano nomi di proprietà a valori.
	\end{itemize}
	\subsection{Node.js}
	\begin{itemize}
		\item\textbf{Utilizzo}:
		\item\textbf{Vantaggi}:
		\item\textbf{Svantaggi}:
	\end{itemize}
	\subsection{PostgreSQL}
	DBMS ad oggetti open-source.
	\begin{itemize}
		\item\textbf{Utilizzo}: Base di dati con lo scopo di memorizzare domande e altri dati necessari al funzionamento del sistema.
		\item\textbf{Vantaggi}: Tecnologia più robusta, stabile e performante di MySQL (inizialmente preso in considerazione ma poi scartato in favore di PostgreSQL).
		\item\textbf{Svantaggi}: Complessità maggiore rispetto al classico MySQL.
	\end{itemize}
	\newpage
	\section{Diagrammi dei packages}
	\subsection{Package della componente Model}
	\begin{figure}[h!]
	\begin{center}
		\includegraphics[scale=0.65]{../images/ModelPackage.png}
		\caption{Package della componente Model}
	\end{center}
	\end{figure}
	Il package per il componente Model del pattern architetturale MVP contiene i seguenti sotto packages:
	\begin{itemize}
		\item\textbf{Database:} questo package si occupera' di gestire le richieste in arrivo dal Presenter e avvisera' quest'ultimo ogni qualvota si verifica un cambiamento dei dati salvati nel sistema tramite appositi eventi; Per fare ciò si avvale delle classi:
			\begin{itemize}
				\item\textit{Database}
				\item\textit{QuestionModifiedEvent}
				\item\textit{QuestionRemovedEvent}
				\item\textit{QuizModifiedEvent}
				\item\textit{QuizRemovedEvent}
			\end{itemize}
				e delle seguenti interfacce:
			\begin{itemize}
				\item\textit{tipologiaEvento}
			\end{itemize}
		\item\textbf{Parser:} questo package fornisce funzionalita' per il controllo sintattico rispetto a QML; Per fare ciò si avvale delle classi:
			\begin{itemize}
				\item\textit{Parser}
			\end{itemize}
		\item\textbf{Statistic:} questo package fornisce classi per il raccoglimento delle statistiche relative ai quesiti, questionari ed utenti; Per fare ciò si avvale delle classi:
			\begin{itemize}
				\item\textit{Statistiche}
			\end{itemize}
		\end{itemize}
		\newpage
	
	\subsection{Package della componente View}
	\begin{figure}[h!]
	\begin{center}
		\includegraphics[scale=0.6]{../images/ViewPackage.png}
		\caption{Package della componente View}
	\end{center}
	\end{figure}
	Il package per il componente View del pattern architetturale MVP contiene i seguenti sotto packages:
	\begin{itemize}
		\item\textbf{CurrentViewManager:} questo sotto package ha lo scopo di definire lo stato attuale del sistema Quizzipedia; per fare ciò si avvale delle classi:
			\begin{itemize}
				\item\textit{CurrentView}%mettere link alla sezione in cui è spiegata
				\item\textit{Sender}%mettere link alla sezione in cui è spiegata
			\end{itemize}
		\item\textbf{Pages:} contiene la classe astratta \textit{Page}, che rappresenta una specifica situazione del sistema (ovvero una pagina web del sito), più tutte le sue derivazioni concrete:
			\begin{itemize}
				\item\textit{MainPage}%mettere link alla sezione in cui è spiegata
				\item\textit{CategoryListPage}%mettere link alla sezione in cui è spiegata
				\item\textit{QuizListPage}%mettere link alla sezione in cui è spiegata
				\item\textit{QuizExecutionPage}%mettere link alla sezione in cui è spiegata
				\item\textit{QuizManagementPage}%mettere link alla sezione in cui è spiegata
				\item\textit{ViewTutorialPage}%mettere link alla sezione in cui è spiegata
			\end{itemize}
		\item\textbf{UserAuthentication:} contiene la classe \textit{User} e la sua derivata \textit{Admin} che raggruppano le funzionalità di autenticazione di un utente generico sulla piattaforma Quizzipedia; come \textit{Pages} anche \textit{UserAuthentication} è un package d'appoggio per la classe \textit{CurrentViewManager}::\textit{CurrentView}.\\
		\\
	\end{itemize}
	\subsection{Package della componente Presenter}
	\begin{center}
		\includegraphics[scale=0.6]{../images/PresenterPackage.png}
	\end{center}
	Gli elementi del package collaborano e interagiscono con l'obiettivo comune di trattare il flusso di dati tra View e Model e mantenere aggiornato lo stato di entrambi. \\
	Il package del Presenter contiene i seguenti sub-package:
	\begin{itemize}
	\item \textbf{UserInputManager}
	Il package \emph{UserInputManager} gestisce gli input degli utenti ricevuti dalla View, realizzando la logica dell'applicazione web. Per svolgere il proprio compito si interfaccia con i package \emph{ViewUpdater} e \emph{ModelUpdater} a seconda di quale componente deve aggiornare. Il package contiene al suo interno la classe:
	\begin{itemize}
		\item \textit{InputManager}
	\end{itemize}
	
	\item \textbf{ModelUpdater}
	Il package \emph{ModelUpdater} ha il compito di riflettere le modifiche della View sul Model. Riceve dalla View i dati, se necessario li elabora, e li passa al Model. Il package contiene al suo interno le seguenti classi:
	\begin{itemize}
		\item \textit{ViewDataReceiver}
		\item \textit{ToModelSender}
	\end{itemize}
	\item \textbf{ViewUpdater}
	Il package \emph{ViewUpdater} ha il compito di riflettere le modifiche del Model sulla View. Se i dati che riceve dal Model sono domande QML, queste vanno prima tradotte in linguaggio HTML per essere utilizzate nella View; per fare ciò si interfaccia con il package \emph{Interpreter}. Il package contiene al suo interno le seguenti classi:
	\begin{itemize}
		\item \textit{ModelDataReceiver}
		\item \textit{ToViewSender}
		\item \textit{Translator}
	\end{itemize}
	\newpage
	\item \textbf{Interpreter}
	Il package \emph{Interpreter} è responsabile della traduzione di testo QML in codice HTML5 visualizzabile da browser. Per permettere al package di essere estendibile in futuro con nuovi tipi di "Interpreter", le classi al suo interno sono organizzate seguendo il pattern \emph{Abstract Factory} . Il package contiene al suo interno le seguenti classi:
	\begin{itemize}
	\item \textit{Interpreter}
	\item \textit{InterpreterFactory}
	\item \textit{QMLInterpreterFactory}
	\item \textit{QMLInterpreter}
	\item \textit{QML2HTMLInterpreter}
	\end{itemize}
	
	\end{itemize}		
	\section{Diagrammi delle classi}
		\subsection{Diagrammi delle classi del Model}
			\subsubsection{Package Database}
			IMMAGINE
 			\subsubsubsection{Classe Database}
 			\begin{itemize}
		    	\item\textbf{Funzione del componente:} la classe permettera' l'inserimento, la lettura, la modifica e la rimozione dei dati all'interno del database
			\item\textbf{Relazioni d'uso di altri componenti:} interagisce con il Presenter, inviando o ricevendo dati sulla base delle richieste di quest'ultimo
			\item\textbf{Attivita' svolte e dati trattati:} ogni metodo della classe consente l'inserimento, la lettura, la modifica e la rimozione di dati dal database, in seguito alla quale si possono verificare eventi per notificare al Presenter
			\end{itemize}
			\subsubsubsection{Interfaccia tipologiaEvento}
			\begin{itemize}
		    	\item\textbf{Funzione del componente:} l'interfaccia identifichera' dei particolari eventi che richiedono l'intervento del Presenter
			\item\textbf{Relazioni d'uso di altri componenti:} questa interfaccia verra implementata da classi che andranno a specificare un particolare tipo di evento
			\end{itemize}
			\subsubsubsection{Classe QuestionModifiedEvent}
			\begin{itemize}
		    	\item\textbf{Funzione del componente:} la classe specifica la modifica di un quesito nel database
				\item\textbf{Relazioni d'uso di altri componenti:} implementa l'interfaccia tipologiaEvento
				\item\textbf{Attivita' svolte e dati trattati:} segnala al Presenter il verificarsi della modifica di un quesito
			\end{itemize}
			\subsubsubsection{Classe QuestionRemovedEvent}
			\begin{itemize}
		    	\item\textbf{Funzione del componente:} la classe specifica la rimozione di un quesito nel database
				\item\textbf{Relazioni d'uso di altri componenti:} implementa l'interfaccia tipologiaEvento
				\item\textbf{Attivita' svolte e dati trattati:} segnala al Presenter il verificarsi della rimozione di un quesito
			\end{itemize}
			\subsubsubsection{Classe QuizModifiedEvent}
			\begin{itemize}
		    	\item\textbf{Funzione del componente:} la classe specifica la modifica di un quiz nel database
				\item\textbf{Relazioni d'uso di altri componenti:} implementa l'interfaccia tipologiaEvento
				\item\textbf{Attivita' svolte e dati trattati:} segnala al Presenter il verificarsi della modifica di un quiz
			\end{itemize}
			\subsubsubsection{Classe QuizRemovedEvent}
			\begin{itemize}
		    	\item\textbf{Funzione del componente:} la classe specifica la rimozione di un quiz nel database
				\item\textbf{Relazioni d'uso di altri componenti:} implementa l'interfaccia tipologiaEvento
				\item\textbf{Attivita' svolte e dati trattati:} segnala al Presenter il verificarsi della rimozione di un quiz
			\end{itemize}
			
			\subsubsection{Package Parser}
			IMMAGINE
 			\subsubsubsection{Classe Parser}
 			\begin{itemize}
		    	\item\textbf{Funzione del componente:} controlla che il testo fornito risulti corretto secondo la sintassi QML
			\item\textbf{Attivita' svolte e dati trattati:} il Parser controlla che il testo fornito in input rispetta la sintassi QML e fornisce in caso di errore un messaggio avvertendo l'utente di dove si trova l'errore e la tipologia
			\end{itemize}
			
			\subsubsection{Package Statistiche}
			IMMAGINE
 			\subsubsubsection{Classe Statistiche}
 			\begin{itemize}
		    	\item\textbf{Funzione del componente:} questa classe fornisce funzionalita' per il raccoglimento delle statistiche all'interno di Quizzipedia
			\item\textbf{Attivita' svolte e dati trattati:} la classe aggiornera' le statistiche relative ai quesiti, questionari ed utenti.
			Per i quesiti verranno indicati il numero di volte che e' stato proposto e il numero di risposte corrette.
			Per i questionari verranno indicati le valutazioni medie ottenute dagli utenti e il numero di volte che e' stato proposto.
			Per gli utenti verranno indicati la valutazione migliore e la media dei tentativi eseguiti su singolo quiz
			\end{itemize}
			
			\subsection{Diagrammi delle classi del View}
		     \subsubsection{Package CurrentViewManager}
			IMMAGINE		    
		    	\subsubsubsection{Classe CurrentView}
		    	\begin{itemize}
		    \item\textbf{Funzione del componente:} lo scopo di questa classe è quello di rappresentare la vista del sistema attualmente disponibile all'utente e di gestirne eventuali cambiamenti 
			\item\textbf{Relazioni d'uso di altri componenti:} (da dire bene) toPresenter, fromPresenter e currentUser sono parametri di tipi appartenenti ad altri package percui c'è una 
relazione di dipendenza (?) verso i rispettivi packages Sender, Receiver ed UserAuthentication
			\item\textbf{Attività svolte e dati trattati:} sono presenti dei parametri per la memorizzazione dell'utente corrente, della pagina corrente e per l'invio e la ricezione di informazioni dal Presenter. La funzionalità principale offerta è enterLink (nome da rivedere) che, come suggerisce il nome, permette all'utente di spostarsi nell'applicazione cambiando pagina. Sono presenti altre funzionalità minori che gestiscono l'avvio e la chiusura della sessione e il ritorno alla homepage.
			\end{itemize}
			\subsubsubsection{Classe Sender}
			\begin{itemize}
			\item\textbf{Funzione del componente:} inoltra le richieste dell'utente al presenter
			\item\textbf{Relazioni d'uso di altri componenti:}  viene utilizzata all'interno di CurrentViewManager::CurrentView in risposta agli eventi che accadono sull'interfaccia web in seguito alle azioni compiute dall'utente; l'output dei metodi di questa classe viene poi processato dal presenter
			\item\textbf{Attività svolte e dati trattati:} ogni metodo di questa classe corrisponde ad una situazione che si verifica, attraverso un rispettivo metodo, in una classe derivata da Pages::Page; i dati restituiti da questi primi metodi verranno reinoltrati da questa classe al presenter
			\end{itemize}
			
			\subsubsection{Package Pages}
			IMMAGINE\\
			\subsubsubsection{Classe Page}
			\begin{itemize}
		    \item\textbf{Funzione del componente:} rappresenta una pagina web //meglio renderla interfaccia?
			\item\textbf{Relazioni d'uso di altri componenti:} classe astratta che viene concretizzata dalle sue classi derivate (MainPage, CategoryListPage, QuizListPage,QuizTutorialPage, QuizExecutionPage, QuizManagementPage)
			\item\textbf{Attività svolte e dati trattati:} le funzionalità offerte dalla classe sono compiti fondamentali che una pagina web deve svolgere ovvero ricevere gli input dell'utente (quali click su link o dati inseriti in un form che poi andranno passati al presenter) e aggiornarsi quando nuovi dati sono disponibili
			\end{itemize}
			\subsubsubsection{Classe MainPage}
			\begin{itemize}
		    \item\textbf{Funzione del componente:} mostra la pagina principale a cui l'utente arriva entrando nella piattaforma Quizzipedia 
			\item\textbf{Relazioni d'uso di altri componenti:} concretizza la classe astratta Page da cui è diretta discendente (e viene usata da CurrentView come currentPage)
			\item\textbf{Attività svolte e dati trattati:} permette l'autenticazione/registrazione dell'utente nel sistema, da una panoramica del sistema generale all'utente
			\end{itemize}
			\subsubsubsection{Classe CategoryListPage}
			\begin{itemize}
		    \item\textbf{Funzione del componente:} pagina che elenca gli argomenti tra i quali l'utente può scegliere e fornisce operazioni di ordinamento sulla vista 
			\item\textbf{Relazioni d'uso di altri componenti:} concretizza la classe astratta Page da cui è diretta discendente (e viene usata da CurrentView come currentPage)
			\item\textbf{Attività svolte e dati trattati:} le funzionalità di questa classe permettono la visualizzazione delle varie categorie, l'ordinamento per diversi criteri (alfabetico, categorie più visitate, ultimi questionari disponibili...) e la scelta di una tra di esse
			\end{itemize}
			\subsubsubsection{Classe QuizListPage}
			\begin{itemize}
		    \item\textbf{Funzione del componente:} pagina che elenca i questionari (su uno stesso argomento) tra i quali l'utente può scegliere e fornisce operazioni di ordinamento sulla vista
			\item\textbf{Relazioni d'uso di altri componenti:} concretizza la classe astratta Page da cui è diretta discendente (e viene usata da CurrentView come currentPage). L'argomento visualizzato è quello scelto nella precedente pagina CategoryListPage
			\item\textbf{Attività svolte e dati trattati:} le funzionalità di questa classe permettono la visualizzazione di informazioni sui vari questionari, l'ordinamento per diversi criteri (alfabetico, più visitato, novità...) e la scelta di uno tra essi
			\end{itemize}
			\subsubsubsection{Classe QuizExecutionPage}
			\begin{itemize}
		    \item\textbf{Funzione del componente:} questa classe rappresenta il punto focale del sistema Quizzipedia ovvero la parte in cui l'utente svolge i questionari scelti
			\item\textbf{Relazioni d'uso di altri componenti:} concretizza la classe astratta Page da cui è diretta discendente (e viene usata da CurrentView come currentPage). Il questionario visualizzato è quello scelto nella precedente pagina QuizListPage.
			\item\textbf{Attività svolte e dati trattati:} l'utente può navigare tra le domande del questionario nell'ordine che preferisce, dare le proprie risposte e, al termine del questionario, visualizzarne il risultato; per ognuna di queste funzionalità è presente un metodo della classe
			\end{itemize}
			\subsubsubsection{Classe QuizManagementPage}
			\begin{itemize}
		    \item\textbf{Funzione del componente:} gestisce creazione, modifica ed eliminazione di singole domande e interi questionari
			\item\textbf{Relazioni d'uso di altri componenti:} concretizza la classe astratta Page da cui è diretta discendente (e viene usata da CurrentView come currentPage)
			\item\textbf{Attività svolte e dati trattati:} le funzionalità offerte dalla classe consentono creazione, modifca ed eliminazione di singoli quesiti e di interi questionari
			\end{itemize}
			\subsubsubsection{Classe QuizTutorialPage}
			\begin{itemize}
		    \item\textbf{Funzione del componente:} questa pagina visualizza un breve manuale che spiega l'uso e la sintassi del linguaggio QML (vedi analisi\_dei\_requisiti\_2.0.pdf)
			\item\textbf{Relazioni d'uso di altri componenti:} concretizza la classe astratta Page da cui è diretta discendente (e viene usata da CurrentView come currentPage)
			\item\textbf{Attività svolte e dati trattati:} questa classe svolge solamente una semplice attività di visualizzazione di informazioni
			\end{itemize}
			\subsubsection{Package UserAuthentication}
			\subsubsubsection{Classe User}
			\begin{itemize}
\item\textbf{Funzione del componente:} classe che rappresenta il singolo utilizzatore attuale del sistema nella propria sessione
\item\textbf{Relazioni d'uso di altri componenti:} User viene utilizzato dalla classe CurrentViewManager::CurrentView
\item\textbf{Attività svolte e dati trattati:} questa classe prevede le funzionalità basilari per l'autenticazione dell'utente all'interno del sistema come registrazione, login e  logout ed altre operazioni secondarie come il cambio o il recupero della password smarrita.
			\end{itemize}
			\newpage
			
			\subsection{Diagrammi delle classi del Presenter}
			\subsubsection{Package UserInputManager}
			\subsubsubsection{InputManager}
			\begin{itemize}
				\item\textbf{Funzione del componente}: riceve gli input dalla View e li indirizza verso le classi ModelUpdater e ViewUpdater.
				\item\textbf{Relazione d'uso di altre componenti}: si interfaccia con i package ModelUpdater e ViewUpdater e con il package View dal quale riceve gli input.
				\item\textbf{Attività svolte e dati trattati}: questa classe possiede operazioni per ricevere e trattare gli input che gli arrivano dalla View. Possiede inoltre i riferimenti alle classi ViewDataReceiver e ToModelSender alle quali inoltra i rispettivi input d'interesse.
			\end{itemize}

			\subsubsection{Package ModelUpdater}
			\subsubsubsection{ViewDataReceiver}
			\begin{itemize}
				\item\textbf{Funzione del componente}: ricevere e strutturare i dati da mandare al Model.
				\item\textbf{Relazione d'uso di altre componenti}: riceve dalla classe InputManager gli input, richiede alla View dati.
				\item\textbf{Attività svolte e dati trattati}: A seconda dell'input richiede alla View i dati da trasmettere al Model e li inoltra alla componente ToModelSender.
			\end{itemize}
			\subsubsubsection{ToModelSender}
			\begin{itemize}
				\item\textbf{Funzione del componente}: inoltra al Model i dati da salvare su database.
				\item\textbf{Relazione d'uso di altre componenti}: collabora con la classe ViewDataReceiver e s'interfaccia con la classe Database del Model esterna al package Presenter.
				\item\textbf{Attività svolte e dati trattati}: possiede un riferimento al Model e, una volta ricevuti i dati da ViewDataReceiver, li trasmette alla classe Database per il salvataggio.
			\end{itemize}
			\subsubsection{Package ViewUpdater}
			\subsubsubsection{ModelDataReceiver}
			\begin{itemize}
				\item\textbf{Funzione del componente}: riceve dal Model nuovi dati  con i quali aggiornare la View.
				\item\textbf{Relazione d'uso di altre componenti}: s'interfaccia con la classe Database per la ricezione dei dati e con la classe ToViewSender per l'inoltro.
				\item\textbf{Attività svolte e dati trattati}: riceve dal database  eventi e i dati modificati da tali eventi e li inoltra alla classe ToViewSender. 
			\end{itemize}
			\subsubsubsection{ToViewSender}
			\begin{itemize}
				\item\textbf{Funzione del componente}: è responsabile dell'aggiornamento della View rispetto agli input utente e rispetto alle modifiche del Model.
				\item\textbf{Relazione d'uso di altre componenti}: s'interfaccia con le classi InputManager, ModelDataReceiver e Translator.
				\item\textbf{Attività svolte e dati trattati}: la componente può ricevere input che non necessitano di interazione col Model; in tal caso provvede direttamente all'aggiornamento della View. Se è richiesta interazione col Model essa riceve i dati dalla classe ModelDataReceiver e li prepara per aggiornare la view. Se i dati sono in formato QML e necessitano di traduzione in HTML la componente richiede i servizi della classe Translator.
			\end{itemize}
			\subsubsubsection{Translator}
			\begin{itemize}
				\item\textbf{Funzione del componente}: s'interfaccia con le componenti del package Interpreter per la traduzione di quesiti QML in HTML.
				\item\textbf{Relazione d'uso di altre componenti}: collabora con le interfacce Interpreter e InterpreterFactory. Interagisce con la classe ToViewSender.
				\item\textbf{Attività svolte e dati trattati}: riceve dalla classe ToViewSender le richieste di traduzione e il codice QML da tradurre. Attraverso la factory InterpreterFactory (una sua concretizzazione) costruisce un Interpreter concreto e lo utilizza per la traduzione del codice. L'esito della traduzione viene reso disponibile a ToViewSender.
			\end{itemize}
			\subsubsection{Package Interpreter}
			\subsubsubsection{Interpreter}
			\begin{itemize}
				\item\textbf{Funzione del componente}: interfaccia di base del tipo Interpreter.
				\item\textbf{Relazione d'uso di altre componenti}: può essere concretizzata in diversi tipi di Interpreter. Viene riferita dalla classe Translator.
				\item\textbf{Attività svolte e dati trattati}: definisce il contratto degli Interpreter, cioè le operazioni di traduzione che saranno definite in ogni concretizzazione.
			\end{itemize}
			\subsubsubsection{InterpreterFactory}
			\begin{itemize}
				\item\textbf{Funzione del componente}: interfaccia di base delle Factory di tipi Interpreter.
				\item\textbf{Relazione d'uso di altre componenti}: può essere concretizzata in diversi tipi di InterpreterFactory. Viene riferita dalla classe Translator.
				\item\textbf{Attività svolte e dati trattati}: definisce il contratto delle factory, cioè le operazioni di costruzione di Interpreter che saranno definite in ogni concretizzazione.
			\end{itemize}
			\subsubsubsection{QMLInterpreterFactory}
			La classe QMLInterpreterFactory è un \emph{singleton}.
			\begin{itemize}
				\item\textbf{Funzione del componente}: crea oggetti di tipo QMLInterpreter.
				\item\textbf{Relazione d'uso di altre componenti}: è concretizzazione della classe InterpreterFactory. Crea oggetti QMLInterpreter.
				\item\textbf{Attività svolte e dati trattati}: crea su richiesta oggetti di tipo QMLInterpreter.
			\end{itemize}
			\subsubsubsection{QMLInterpreter}
			\begin{itemize}
				\item\textbf{Funzione del componente}: classe astratta che rappresenta gli Interpreter che traducono codice QML in un altro formato.
				\item\textbf{Relazione d'uso di altre componenti}: è sottotipo di Interpreter. Può essere concretizzata in più tipi di QMLInterpreter.
				\item\textbf{Attività svolte e dati trattati}: definisce il contratto dei QMLInterpreter, cioè le operazioni di traduzione da QML verso altri linguaggi.
			\end{itemize}
			\subsubsubsection{QML2HTMLInterpreter}	
			\begin{itemize}
				\item\textbf{Funzione del componente}: traduce codice QML in codice HTML.
				\item\textbf{Relazione d'uso di altre componenti}: è concretizzazione di QMLInterpreter.
				\item\textbf{Attività svolte e dati trattati}: riceve in input domande in QML e le traduce in codice HTML visualizzabile da browser.
			\end{itemize}
			\newpage	
	\section{Diagrammi di attività}
	
\subsubsection{Creazione Questionario}
\begin{center}
	\centerline{\includegraphics[scale=0.1]{"../images/diagramma 1".png}}
\end{center}
\textbf{Precondizioni:} l'utente è autenticato nel sistema Quizzipedia e accede all'area del sito dedicata alla gestione (creazione, eliminazione e modifica) dei propri questionari.\\
\textbf{Postcondizioni:} l'utente ha creato con successo (o ha annullato la creazione di) un questionario su un argomento a piacere contenente domande già disponibili o create precedentemente da lui stesso. L'utente viene reindirizzato alla pagina principale. Il questionario creato può ora essere svolto da altri utenti del sistema Quizzipedia.\\ %da rivedere dove viene reindirizzato l'utente
\textbf{Descrizione:} l'utente inizialmente assegna un titolo al questionario, sceglie l'argomento trattato tra le categorie presenti nel sistema e assegna un tempo limite entro il quale il questionario deve essere completato. A questo punto l'utente può iniziare ad inserire domande nel questionario: di volta in volta può scegliere tra le domande già disponibili sull'argomento (domande precaricate dal team404 e domande create da altri utenti del sistema) o tra le domande create da lui stesso (vedi diagramma d'attività in sezione successiva). Quando l'utente ha terminato di inserire le domande nel proprio questionario ne verrà presentata un'anteprima. L'utente ha infine la possibilità di annullare la creazione del questionario (e quindi tornare alla schermata precedente perdendo però i dati immessi in precedenza) o di confermarla con conseguente pubblicazione nel sito dove sarà disponibile agli altri utenti.\\

\newpage
\subsubsection{Creazione Domanda}
\begin{center}
	\centerline{\includegraphics[scale=0.1]{"../images/diagramma 2".png}}
\end{center}
\textbf{Precondizioni:} l'utente è autenticato nel sistema Quizzipedia e accede all'area del sito dedicata alla gestione (creazione, eliminazione e modifica) dei propri singoli quesiti.\\
\textbf{Postcondizioni:} l'utente ha creato con successo una nuova domanda. La nuova domanda sarà disponibile a tutti gli utenti qualora essi decidano di creare un questionario appartente alla stessa categoria. L'utente viene reindirizzato alla pagina di gestione questionari.\\ %anche qua decidere redirect
\textbf{Descrizione:} l'utente inizialmente sceglie una categoria (argomento trattato) per la domanda, successivamente procede con la stesura del codice QML in un'area di testo apposita. Qualora l'utente abbia dubbi sull'uso della sintassi è presente nella pagina un link al tutorial interno al sistema. Quando l'utente ha terminato l'inserimento del codice può sottoporlo al sistema che lo valuterà. L'input viene parsato per controllare anomalie nella sintassi, se sono presenti errori ciò viene segnalato all'utente che ha la possibilità di correggere il codice errato, altrimenti la domanda viene considerata valida e il sistema procede alla sua memorizzazione nel sistema (con conseguente notifica positiva all'utente). Al termine del procedimento la domanda sarà disponibile a tutti gli utenti registrati durante la fase di creazione di questionari.\\
\newpage
\subsubsection{Compilazione Questionario}
\begin{center}
	\includegraphics[scale=0.1]{"../images/diagramma 3".png}
\end{center}
\textbf{Precondizioni:} durante le sue ultime interazioni con il sistema Quizzipedia, l'utente ha scelto un'argomento e un questionario inerente a quell'argomento che è intenzionato a svolgere. Alternativamente l'utente arriva dall'esterno seguendo un link ad un preciso questionario.\\
\textbf{Postcondizioni:} L'utente ha terminato lo svolgimento del questionario (ha risposto a tutte le domande oppure è scaduto il tempo limite). L'utente viene reindirizzato ad una pagina contenente i risultati della sua prestazione cognitiva.\\
\textbf{Descrizione:} Al momento dell'inizio del questionario (coincidente con il momento in cui l'utente accede alla pagina del questionario) il tempo limite inizia a scorrere, l'utente dovrà rispondere al maggior numero di domande possibili entro lo scadere del tempo (prefisatto dal creatore del questionario, vedi sezioni precedenti). Viene presentato un singolo quesito per volta all'utente. L'utente può dare una risposta oppure cambiare domanda. L'utente ha la libertà di scorrere le domande del questionario e di svolgerle in qualsiasi ordine. Se l'utente ha dato una risposta a tutte le domande può consegnare il questionario. Se il tempo limite scade il questionario verrà immediatamente consegnato anche se ancora incompleto. Al momento della consegna il questionario verrà valutato dal sistema che informerà poi l'utente sull'esito della sua performance mediante un redirect ad una apposita pagina contente statistiche e correzioni.\\
\newpage
	\section{Diagrammi di sequenza}
	\newpage
	\section{Tracciamento}
	\newpage
	\section{Analisi di fattibilità}


\end{document}
