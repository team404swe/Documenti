\documentclass[a4paper,11pt]{article}
	\input{../../template.tex}

	\title{\textbf{{\fontsize{8mm}{5mm}\selectfont QUIZZIPEDIA}}}
	\date{}
	\author{}	


	\begin{document}
	\maketitle
	\thispagestyle{empty}
	\begin{center}	
	\includegraphics{../team_not_found.jpg}\\
	\fontsize{5mm}{3mm}\url{team404swe@gmail.com}\\
	
	\vspace{50mm}
	\textbf{Specifica Tecnica 1.0}
	\end{center}
	\introtab{Specifica Tecnica 1.0}			%1 nome documento
			{1.0} 							%2 versione
			{Esterno} 						%3 Uso
			{22 aprile 2016} 				%4 Data cre
			{\today} 						%5 Data mod
			{A. Multineddu - L. Alessio - D. Bortot}	%6 Redazione
			{Alex Beccaro } 			%7 Verifica
			{Martin V. Mbouenda} 				%8 Approvazione
	\newpage
	\thispagestyle{empty}
	\null  

	\newpage
	\newpage
	\fancyhead[R]{REGISTRO DELLE MODIFICHE}
	\fancyfoot[R]{\thepage}
	
	\hspace{30 mm}
	\section*{Registro delle modifiche}
	
	\beginregistro
	
	\rigaregistro{\textbf{Versione}}{\textbf{Autore}}{\textbf{Data}}		 {\hspace{5 mm} 
	\textbf{Descrizione}}
	\rigaregistro{0.0.16}{Davide Bortot (Progettista)}{10/05/2016}{Stesura della sezione "Design pattern utilizzati".}
	\rigaregistro{0.0.15}{Davide Bortot (Progettista)}{08/05/2016}{Inseriti grafici dei Diagrammi di Sequenza. Aggiunta tecnologia "Materialize" alla sezione "Tecnologie e strumenti utilizzati".}
	\rigaregistro{0.0.14}{Luca Alessio (Progettista)}{07/05/2016}{Stesura sezione "Diagrammi di Sequenza" per la parte descrittiva.}
	\rigaregistro{0.0.13}{Davide Bortot (Progettista)}{06/05/2016}{Integrara la sezione "Diagrammi delle classi del Presenter" con l'aggiunta dei grafici UML.}
	\rigaregistro{0.0.12}{Andrea Multineddu (Progettista)}{06/05/2016}{Aggiunte immagini sezione "Diagrammi delle attività".}
	\rigaregistro{0.0.11}{Davide Bortot (Progettista)}{05/05/2016}{Completata la parte descrittiva della sezione "Diagrammi delle classi del Presenter". Riadattamento allo stile del documento di §4.3. Inseriti grafici UML dei Package del Model e della View.}
	\rigaregistro{0.0.10}{Luca Alessio (Progettista)}{04/05/2016}{Stesura sezione "Diagrammi delle attività" e correzioni varie su altre sezioni.}
	\rigaregistro{0.0.9}{Davide Bortot (Progettista)}{03/05/2016}{Stesura sezione "Package della componente Presenter", inserito rispettivo grafico UML, e prima stesura di "Diagrammi delle classi del Presenter".}
	\rigaregistro{0.0.8}{Andrea Multineddu (Progettista)}{01/05/2016}{Prima stesura sezioni "Package della componente Model" e "Diagrammi delle classi del Model".}
	\rigaregistro{0.0.7}{Luca Alessio (Progettista)}{30/04/2016}{Prima stesura sezioni "Package della componente View" e "Diagrammi delle classi del View".}	
	\rigaregistro{0.0.6}{Davide Bortot (Progettista)}{28/04/2016}{Stesura delle sezioni §2.2, §2.3, §2.4 con la descrizione d'alto livello delle componenti MVP.}
	\rigaregistro{0.0.5}{Luca Alessio (Progettista)}{27/04/2016}{Stesura parziale sezione "Tecnologie e strumenti utilizzati"}
	\rigaregistro{0.0.4}{Davide Bortot (Progettista)}{25/04/2016}{Ampliata la sezione "Architettura generale del sistema".}
	\rigaregistro{0.0.3}{Luca Alessio (Progettista)}{24/04/2016}{Stesura parziale sezione "Architettura generale del sistema"}
	\rigaregistro{0.0.2}{Davide Bortot (Progettista)}{23/04/2016}{Strutturazione iniziale del documento e stesura sezione introduttiva}
	\rigaregistro{0.0.1}{Marco Crivellaro (Progettista)}{22/04/2016}{Creazione documento.}
	\fineregistro

	\newpage
	\fancyhead[R]{\leftmark}
	\tableofcontents
	\pagenumbering{Roman}
	\newpage
	\listoffigures
	\listoftables
	
	\newpage
	\pagenumbering{arabic}
	
	\section*{Sommario}
	Questo documento, redatto dal gruppo \textbf{Team404}, contiene la descrizione dell'architettura software sulla quale verrà sviluppato il progetto Quizzipedia, commissionato da Zucchetti S.p.A..
Scopo del documento è quello di illustrare le scelte progettuali che il gruppo ha deciso di
seguire per realizzare il prodotto.
	
	\newpage
	\section{Introduzione}
	\subsection{Scopo del documento}
	Lo scopo della Specifica Tecnica è quello di illustrare le scelte progettuali che il gruppo ha deciso di seguire nella realizzazione del prodotto. Vengono presentati, pur restando ad alto
livello, la gerarchia dei package, le loro interazioni e le principali classi in essi contenute, specificandone lo scopo e l'utilizzo. Verranno presentati inoltre i design pattern utilizzati nell'architettura.
	
	\subsection{Scopo del prodotto}
	Il progetto \textbf{Quizzipedia} ha come obiettivo lo sviluppo di un sistema software basato su tecnologie Web (Javascript\addglos, Node.js\addglos, HTML5\addglos, CSS3\addglos) che permetta la creazione, gestione e fruizione di questionari. Il sistema dovrà quindi poter archiviare i questionari suddivisi per argomento, le cui domande dovranno essere raccolte attraverso uno specifico linguaggio di markup (Quiz Markup Language) d'ora in poi denominato QML\addglos. In un caso d'uso a titolo esemplificativo, un "esaminatore" dovrà poter costruire il proprio questionario scegliendo tra le domande archiviate, ed il questionario così composto sarà presentato e fruibile all' "esaminando", traducendo l'oggetto QML in una pagina HTML\addglos, tramite un'apposita interfaccia web. Il sistema presentato dovrà inoltre poter proporre questionari preconfezionati e valutare le risposte fornite dall'utente finale.
	\\
	Per un'analisi più precisa ed approfondita del progetto si rimanda al documento\\ "\textit{analisi\_dei\_requisiti\_2.0.pdf}".
	\subsection{Glossario}
	Viene allegato un glossario nel file ``\textit{glossario\_2.0.pdf}'' nel quale viene data una definizione a tutti i termini che in questo documento appaiono con il simbolo '\addglos' a pedice.
	\subsection{Riferimenti}
		\subsubsection{Normativi}

		\begin{itemize}
			\item Capitolato d'appalto Quizzipedia:\\
			\url{http://www.math.unipd.it/~tullio/IS-1/2015/Progetto/C5.pdf}
			\item Norme di Progetto: "\textit{norme\_di\_progetto\_2.0.pdf}"
		\end{itemize}
		\subsubsection{Informativi}
		\begin{itemize}
			\item Corso di Ingegneria del Software anno 2015/2016:\\
			\url{http://www.math.unipd.it/~tullio/IS-1/2015/}
			\item Regole del progetto didattico:\\
			\url{http://www.math.unipd.it/~tullio/IS-1/2015/Dispense/PD01.pdf}
			\url{http://www.math.unipd.it/~tullio/IS-1/2015/Progetto/}\\
			\end{itemize}
	\pagebreak
	\newpage
	\section{Specifiche del prodotto}
	I contenuti della specifica saranno presentati seguendo l'approccio top-down: dalla descrizione macroscopica del sistema si scenderà sempre più in dettaglio passando alla descrizione delle singole componenti. Verranno inoltre descritti i design pattern utilizzati e come essi sono stati applicati. Per agevolare la comprensione si è scelto di utilizzare i diagrammi dei package, delle classi, di attività e di sequenza, descritti attraverso lo standard UML 2.0.
	
	\subsection{Architettura generale del sistema}
	Il sistema Quizzipedia è di tipo \emph{client-server}; il \emph{client} fornisce all'utente un'interfaccia web su browser per la creazione e fruizione di questionari, mentre il lato \emph{server} si occupa di gestire e salvare i dati su DBMS PostgreSQL. La base di dati raccoglie principalmente quesiti memorizzati in QML, che su richiesta verranno elaborati da un interprete.
Il suddetto interprete processa l'input QML, precedentemente validato nella forma da un parser apposito, traducendolo in linguaggio HTML5 visualizzabile da browser.
Nella realizzazione del sistema Quizzipedia verrà adottato il design pattern \emph{Model View Presenter} nella sua variante \emph{Passive View}.
\begin{figure}[h!]
\begin{center}
	\includegraphics[scale=1.3]{../images/mvp.png}
	\caption{Il design pattern Model View Presenter (Passive View)}
\end{center}
\end{figure}
\begin {itemize}
\item\textbf{Model}: definisce l'organizzazione dei dati e ne specifica le modalità di accesso. Nel sistema Quizzipedia è situato nella parte server che opera sulla sottostante base di dati. Fa parte del Model anche la componente \emph{Parser} che opera sui dati prima che essi vengano salvati nel DBMS.
\item\textbf{View (Passive)}: rappresenta l'interfaccia grafica presentata all'utilizzatore, la quale visualizza i dati e cattura le interazioni dell'utente. L'aggettivo "Passive" indica che la View non è responsabile del proprio aggiornamento al variare del Model, compito che ricade sul Presenter.
\item\textbf{Presenter}: controlla la View e ne gestisce il comportamento in reazione alle interazioni dell'utente, interagisce di conseguenza col Model per ottenere i dati necessari.
Una volta ottenuti i dati si preoccupa di aggiornare la View. Nel sistema Quizzipedia implementa la parte logica, affiancata a quella grafica, del Client. Realizza un totale disaccoppiamento tra Model e View, controllandone i flussi di comunicazione.
	\end {itemize}
	Altre possibili architetture che sono state prese in considerazione sono definite dai design pattern \emph{Model View Controller (MVC)} con \emph{Front Controller}, \emph{Model View Presenter (MVP)} nelle varianti \emph{Presentation Model} e \emph{Supervising Controller}, e il design pattern \emph{Model View ViewModel (MVVM)}.
	\begin{itemize}
	\item La prima pone il Controller, componente simile al Presenter, nella parte server del sistema. Quest'opzione è stata scartata per evitare di aumentare troppo la complessità del lato server, ponendo invece il Presenter dal lato client, così da redistribuire responsabilità e carico di lavoro.
	\item Il \emph{Presentation Model} invece è affine al design pattern scelto, ma impone che sia la componente View ad aggiornarsi autonomamente al variare del Model. Proprio per questo motivo si è deciso di scartarla in favore del Passive View: per disaccoppiare completamente le componenti Model e View, e per alleggerire ulteriormente quest'ultima concentrandone tutta la parte logica nel Presenter.
	\item Il \emph{Supervising Controller} propone che la View si aggiorni autonomamente nel caso di piccole modifiche (tramite data-binding col Model), lasciando le manipolazioni più complicate al Presenter; è stato scartato in favore del disaccoppiamento totale tra View e Model.
	\item Il pattern \emph{MVVM} prevede di creare per ogni View un ViewModel, che rappresenta tutte le informazioni e i comportamenti della corrispondente View. La View si limita infatti, a visualizzare graficamente quanto esposto dal ViewModel, a riflettere in esso i suoi cambi di stato oppure ad attivarne dei comportamenti (tramite data-binding). Tale architettura è indicata per applicazioni particolarmente dinamiche in cui View e Model devono essere costantemente aggiornati. Non è questo il caso di Quizzipedia.
	\end{itemize}
	\subsection{Descrizione del componente Model}
	Il Model, situato nella parte server del sistema svolge le seguenti funzioni:
	\begin{itemize}
		\item Interagisce con un database PostgreSQL nel quale vengono salvati i dati del sistema (ad es. domande, statistiche, utenti). Fornisce quindi adeguate funzionalità di salvataggio e caricamento da database di tali dati.
		\item Al momento del salvataggio di una nuova domanda esegue il \emph{parsing} del codice QML tramite un componente chiamato \emph{Parser}. Se la domanda è sintatticamente corretta può essere salvata nel database.
		\item Offre un'interfaccia logica di accesso al Presenter attraverso la quale richiedere dati e operazioni su di essi.
	\end{itemize}
	\subsection{Descrizione del componente View}
	La componente View rappresenta l'interfaccia grafica che visualizza i dati del Model e inoltra i comandi dell'utente (o gli eventi da esso generati) al Presenter che si occuperà di gestire tali richieste sui dati interagendo col Model; la View si occupa quindi solamente della rappresentazione grafica dei dati e non ha alcun contatto diretto con essi. Essendo un'interfaccia web verrà realizzata tramite HTML5 e CSS3 per le parti statiche e con l'utilizzo di Javascript per le parti dinamiche. Per assicurare uno stile coerente tra le pagine web e migliorare l'adattabilità a piattaforme mobile verrà utilizzato il framework \emph{Materialize}.
	\subsection{Descrizione del componente Presenter}
	Il Presenter ricopre tre ruoli fondamentali: recepire ed elaborare gli input dell'utente,
comunicare col Model, ed aggiornare il View con i dati ottenuti. Per poterlo fare possiede le seguenti caratteristiche:
	\begin{itemize}
		\item Conosce i riferimenti alle altre due componenti. Il Presenter è l'unica componente che conosce entrambe le altre e facendo da singolo tramite tra Model e View permette il loro totale disaccoppiamento;
		\item E' in grado di elaborare gli input della View e tradurli in azioni sul Model. Viceversa ad ogni modifica del Model si preoccupa di aggiornare di conseguenza la View.  
		\item E' responsabile della traduzione delle domande dal formato QML a formato HTML visualizzabile da browser, tramite un componente \emph{Interpreter}.  Tale funzionalità viene richiesta ogni qualvolta il Presenter richiede e riceve dal Model una domanda in formato QML.
		\item Possiede dei gestori che possano modificare l'aspetto della View in reazione all'interazione dell'utente o ai dati ricevuti dal Model. Al suo interno il Presenter contiene delle classi che in risposta ad un evento, quale l'interazione dell'utente con la View o il ricevimento di una risposta dal Model, modificano l'aspetto della GUI presentata.
		\item Gestisce la somministrazione di un questionario ad un utente, domanda dopo domanda, fino alla consegna e valutazione.
	\end{itemize}
	\newpage

	\section{Tecnologie e strumenti utilizzati}
	\subsection{HTML5}
	Linguaggio di markup per la progettazione di pagine Web. Richiesto espressamente nel capitolato per la creazione dell'interfaccia utente (mi pare, devo controllare).
	\begin{itemize}
		\item\textbf{Utilizzo}: viene utilizzato per creare la GUI che permette all'utente di accedere al sistema mediante browser.
		\item\textbf{Vantaggi}: favorisce la portabilità su diversi dispositivi (desktop, smartphone, tablet...) e browser.
Sono punti a favore anche l'elevata compatibilità con tecnologie quali CSS3 e Javascript.
		\item\textbf{Svantaggi}: essendo HTML5 un linguaggio non ancora standard il rischio nel suo utilizzo è l'instabilità dei tag utilizzati. Un tag oggi accettato potrebbe essere modificato in un futuro prossimo, rendendo la visualizzazione dell'interfaccia utente dipendente dal stabilità dei tag utilizzati. 
	\end{itemize}
	\subsection{CSS3}
	Principale linguaggio usato per la formattazione di pagine HTML.
	\begin{itemize}
		\item\textbf{Utilizzo}: Viene utilizzato per formattare il codice HTML,
ovvero creare fogli di stile che permettono allo sviluppatore di modificare alcuni aspetti grafici della pagina Web.
		\item\textbf{Vantaggi}: Richiede un minor sforzo di interpretazione da parte del browser ed è leggero da scaricare.
		\item\textbf{Svantaggi}: Può presentare problemi di compatibilità con browser meno recenti.
	\end{itemize}
	\subsection{Materialize}
	Materialize è un framework CSS basato sull'idea di design "Material Design" di Google.
	\begin{itemize}
		\item\textbf{Utilizzo}: verrà utilizzato per la formattazione delle pagine web.
		\item\textbf{Vantaggi}: un tale framework aiuta l'applicazione a mantenere uno stile coerente in tutte le sue parti, e mette a disposizione una vasta scelta ci classi CSS preconfezionate, esonerando lo sviluppatore dalla necessita di definirne di proprie. Aumenta inoltra l'adattabilità (responsiveness) delle pagine HTML a diversi supporti (desktop, mobile, etc.).
		\item\textbf{Svantaggi}: necessità di un breve tempo di studio per impararne le classi e il corretto utilizzo.
	\end{itemize}
	\subsection{Javascript}
	E' un linguaggio di scripting debolmente orientato agli oggetti, utilizzato nelle applicazioni
Web. Viene interpretato all'interno del browser. Permette di definire funzionalità simili a
quelle offerte da C++ e Java, quali cicli e strutture di controllo.
Viene solitamente affiancato a pagine statiche HTML per poter gestire i contenuti dinamicamente, offrendo funzionalità che il linguaggio di markup non può offrire.
	\begin{itemize}
		\item\textbf{Utilizzo}: nel sistema Quizzipedia Javascript rivestirà un ruolo importante, verrà infatti utilizzato nella realizzazione del parser/interprete/boh andrea dimmi te.
		\item\textbf{Vantaggi}: eseguito lato Client (sul browser) non sovraccarica il Server per l'esecuzione di richieste, anche se lo script è complesso.
		\item\textbf{Svantaggi}: per script sorgenti molto corposi, può risultare oneroso in termini di
tempo lo scaricamento dei contenuti. Deve, inoltre, fare affidamento ad un linguaggio
che possa fisicamente effettuare transazioni di dati quando lo script esegue operazioni
su oggetti remoti (eg: database).
E' altresì un linguaggio non tipizzato, quindi occorre porre attenzione ai tipi delle
variabili che non sono dichiarati, ma variano dinamicamente.
		\item\textbf{Variabili e oggetti}: le variabili se sono dichiarate all'interno di una funzione sono
visibili solo all'interno di essa; se sono invece esterne sono globali. Vengono dichiarate
con la keyword \emph{var} o semplicemente assegnando loro un valore.
Ogni elemento in JavaScript è un tipo primitivo o un oggetto.
Gli oggetti sono entità dotate di unicità (sono uguali solo a sé stessi) e identificabili
con vettori associativi, che associano nomi di proprietà a valori.
	\end{itemize}
	\subsection{Node.js}
	\begin{itemize}
		\item\textbf{Utilizzo}: Tecnologia imposta dal proponente per la gestione della base di dati su cui poggia il sistema Quizzipedia.
		\item\textbf{Vantaggi}: Tecnologia simil javascript dotata di robusta sistema di packaging e di una velocità superiore rispetto alla concorrenza. Secondariamente, questa tecnologia cross-platform ha suscitato molto di più l'interesse del team che se ne intende avvalere rispetto a Tomcat, l'altra tecnologia proposta dal proponente.
		\item\textbf{Svantaggi}: Il design del linguaggio impedisce alcune ottimizzazioni delle operazioni e la gestione dei tipi non è confortevole come in altri concorrenti.
	\end{itemize}
	\subsection{PostgreSQL}
	DBMS ad oggetti open-source.
	\begin{itemize}
		\item\textbf{Utilizzo}: Base di dati con lo scopo di memorizzare domande e altri dati necessari al funzionamento del sistema.
		\item\textbf{Vantaggi}: Tecnologia più robusta, stabile e performante di MySQL (inizialmente preso in considerazione ma poi scartato in favore di PostgreSQL).
		\item\textbf{Svantaggi}: Complessità maggiore rispetto al classico MySQL.
	\end{itemize}	
	\newpage
	\section{Diagrammi dei packages}
	\subsection{Package della componente Model}
	\begin{figure}[h!]
	\begin{center}
		\includegraphics[scale=0.65]{../images/ModelPackage.png}
		\caption{Package della componente Model}
	\end{center}
	\end{figure}
	Il package per il componente Model del pattern architetturale MVP contiene i seguenti sotto packages:
	\begin{itemize}
		\item\textbf{Database:} questo package si occuperà di gestire le richieste in arrivo dal Presenter e avviserà quest'ultimo ogni qualvolta si verifica un cambiamento dei dati salvati nel sistema tramite appositi eventi; Per fare ciò si avvale delle classi:
			\begin{itemize}
				\item\textit{Database}
				\item\textit{QuestionModifiedEvent}
				\item\textit{QuestionRemovedEvent}
				\item\textit{QuizModifiedEvent}
				\item\textit{QuizRemovedEvent}
				\item\textit{UserCreatedEvent}
				\item\textit{UserModifiedEvent}
				\item\textit{UserRemovedEvent}
			\end{itemize}
				e delle seguenti interfacce:
			\begin{itemize}
				\item\textit{ModelEvent}
			\end{itemize}
		\item\textbf{Parser:} questo package fornisce funzionalità per il controllo sintattico rispetto a QML; Per fare ciò si avvale delle classi:
			\begin{itemize}
				\item\textit{Parser}
			\end{itemize}
		\item\textbf{Statistic:} questo package fornisce classi per il raccoglimento delle statistiche relative ai quesiti, questionari ed utenti; Per fare ciò si avvale delle classi:
			\begin{itemize}
				\item\textit{Statistics}
			\end{itemize}
		\end{itemize}
		\newpage
	
	\subsection{Package della componente View}
	\begin{figure}[h!]
	\begin{center}
		\includegraphics[scale=0.6]{../images/ViewPackage.png}
		\caption{Package della componente View}
	\end{center}
	\end{figure}
	Il package per il componente View del pattern architetturale MVP contiene i seguenti sotto packages:
	\begin{itemize}
		\item\textbf{CurrentViewManager:} questo sotto package ha lo scopo di definire lo stato attuale del sistema Quizzipedia; per fare ciò si avvale delle classi:
			\begin{itemize}
				\item\textit{CurrentView}%mettere link alla sezione in cui è spiegata
				\item\textit{Sender}%mettere link alla sezione in cui è spiegata
			\end{itemize}
		\item\textbf{Pages:} contiene la classe astratta \textit{Page}, che rappresenta una specifica situazione del sistema (ovvero una pagina web del sito), più tutte le sue derivazioni concrete:
			\begin{itemize}
				\item\textit{MainPage}%mettere link alla sezione in cui è spiegata
				\item\textit{CategoryListPage}%mettere link alla sezione in cui è spiegata
				\item\textit{QuizListPage}%mettere link alla sezione in cui è spiegata
				\item\textit{QuizExecutionPage}%mettere link alla sezione in cui è spiegata
				\item\textit{QuizManagementPage}%mettere link alla sezione in cui è spiegata
				\item\textit{ViewTutorialPage}%mettere link alla sezione in cui è spiegata
			\end{itemize}
		\item\textbf{UserAuthentication:} contiene la classe \textit{User} e la sua derivata \textit{Admin} che raggruppano le funzionalità di autenticazione di un utente generico sulla piattaforma Quizzipedia; come \textit{Pages} anche \textit{UserAuthentication} è un package d'appoggio per la classe \textit{CurrentViewManager}::\textit{CurrentView}.\\
		\\
	\end{itemize}
	\subsection{Package della componente Presenter}
	\begin{center}
		\includegraphics[scale=0.6]{../images/PresenterPackage.png}
	\end{center}
	Gli elementi del package collaborano e interagiscono con l'obiettivo comune di trattare il flusso di dati tra View e Model e mantenere aggiornato lo stato di entrambi. \\
	Il package del Presenter contiene i seguenti sub-packages:
	\begin{itemize}
	\item \textbf{UserInputManager}:
	Il package \emph{UserInputManager} gestisce gli input degli utenti ricevuti dalla View, realizzando la logica dell'applicazione web. Per svolgere il proprio compito si interfaccia con il package \emph{Updaters} per aggiornare e interagire con View e Model. \\
	Il package contiene al suo interno le classi:
	\begin{itemize}
		\item \textit{InputManager}
		\item \textit{Input}
		\item \textit{sottoclassi di Input}
	\end{itemize}
	
	\item \textbf{Updaters}:
	Il package \emph{Updaters} ha la responsabilità di aggiornare View e Model in risposta agli input utente. per svolgere i suoi compiti collabora con i packages \emph{QuizManger} e \emph{QuestionManager}. Il package contiene al suo interno le classi:
	\begin{itemize}
		\item \textit{Updater}
		\item \textit{ModelUpdater}
		\item \textit{ViewUpdater}
	\end{itemize}
	
	\item \textbf{QuestionManager}:
	Il package \emph{QuestionManager} espone le funzionalità necessarie alla creazione e gestione di domande. Per permettere al package di essere estendibile in futuro con domande in diversi formati, le classi al suo interno sono organizzate seguendo il pattern \emph{Abstract Factory}. \\
	Il package contiene al suo interno le classi:
	\begin{itemize}
		\item \textit{QuestionFactory}
		\item \textit{Question}
		\item \textit{QML2HTMLFactory}
		\item \textit{HTMLQuestion}
		\item \textit{Translator}
	\end{itemize}
	
	\item\textit{QuizManager}:
	Il Package \emph{QuizManager} espone le funzionalità necessarie alla creazione e gestione di quiz, collaborando con il package \emph{QuestionManager}. Per disaccoppiare la creazione di un quiz dalle domande che lo compongono, la classi all'interno del package seguono la struttura del design pattern \emph{Builder}.\\ 
	Il package contiene al suo interno le classi:
	\begin{itemize}
		\item \textit{QuizDirector}
		\item \textit{QuizBuilder}
		\item \textit{Quiz}
	\end{itemize}
	
	\item \textbf{Interpreter}:
	Il package \emph{Interpreter} è responsabile della traduzione di testo QML in codice HTML5 visualizzabile da browser. Per permettere al package di essere estendibile in futuro con nuovi tipi di "Interpreter", le classi al suo interno sono organizzate seguendo il pattern \emph{Abstract Factory} . \\
	Il package contiene al suo interno le seguenti classi:
	\begin{itemize}
	\item \textit{Interpreter}
	\item \textit{InterpreterFactory}
	\item \textit{QMLInterpreterFactory}
	\item \textit{QMLInterpreter}
	\item \textit{QML2HTMLInterpreter}
	\end{itemize}
	
	\end{itemize}
	\section{Diagrammi delle classi}
		\subsection{Diagrammi delle classi del Model}
			\subsubsection{Package Database}
			\begin{center}
				\includegraphics[scale=0.6]{../images/Database.jpg}
			\end{center}
 			\subsubsubsection{Classe Database}
 			\begin{itemize}
		    	\item\textbf{Funzione del componente:} la classe permettera' l'inserimento, la lettura, la modifica e la rimozione dei dati all'interno del database
			\item\textbf{Relazioni d'uso di altri componenti:} interagisce con il Presenter, inviando o ricevendo dati sulla base delle richieste di quest'ultimo
			\item\textbf{Attivita' svolte e dati trattati:} ogni metodo della classe consente l'inserimento, la lettura, la modifica e la rimozione di dati dal database, in seguito alla quale si possono verificare eventi per notificare al Presenter
			\end{itemize}
			\subsubsubsection{Interfaccia ModelEvent}
			\begin{itemize}
		    	\item\textbf{Funzione del componente:} l'interfaccia identifichera' dei particolari eventi che richiedono l'intervento del Presenter
			\item\textbf{Relazioni d'uso di altri componenti:} questa interfaccia verra implementata da classi che andranno a specificare un particolare tipo di evento
			\end{itemize}
			\subsubsubsection{Classe QuestionModifiedEvent}
			\begin{itemize}
		    	\item\textbf{Funzione del componente:} la classe specifica la modifica di un quesito nel database
				\item\textbf{Relazioni d'uso di altri componenti:} implementa l'interfaccia ModelEvent
				\item\textbf{Attivita' svolte e dati trattati:} segnala al Presenter il verificarsi della modifica di un quesito
			\end{itemize}
			\subsubsubsection{Classe QuestionRemovedEvent}
			\begin{itemize}
		    	\item\textbf{Funzione del componente:} la classe specifica la rimozione di un quesito nel database
				\item\textbf{Relazioni d'uso di altri componenti:} implementa l'interfaccia ModelEvent
				\item\textbf{Attivita' svolte e dati trattati:} segnala al Presenter il verificarsi della rimozione di un quesito
			\end{itemize}
			\subsubsubsection{Classe QuizModifiedEvent}
			\begin{itemize}
		    	\item\textbf{Funzione del componente:} la classe specifica la modifica di un quiz nel database
				\item\textbf{Relazioni d'uso di altri componenti:} implementa l'interfaccia ModelEvent
				\item\textbf{Attivita' svolte e dati trattati:} segnala al Presenter il verificarsi della modifica di un quiz
			\end{itemize}
			\subsubsubsection{Classe QuizRemovedEvent}
			\begin{itemize}
		    	\item\textbf{Funzione del componente:} la classe specifica la rimozione di un quiz nel database
				\item\textbf{Relazioni d'uso di altri componenti:} implementa l'interfaccia ModelEvent
				\item\textbf{Attivita' svolte e dati trattati:} segnala al Presenter il verificarsi della rimozione di un quiz
			\end{itemize}
			\subsubsubsection{Classe UserCreatedEvent}
			\begin{itemize}
		    	\item\textbf{Funzione del componente:} la classe specifica l'aggiunta di un nuovo utente nel database
				\item\textbf{Relazioni d'uso di altri componenti:} implementa l'interfaccia ModelEvent
				\item\textbf{Attivita' svolte e dati trattati:} segnala al Presenter il verificarsi dell'aggiunta di un nuovo utente
			\end{itemize}
			\subsubsubsection{Classe UserModifiedEvent}
			\begin{itemize}
		    	\item\textbf{Funzione del componente:} la classe specifica la modifica di un utente nel database
				\item\textbf{Relazioni d'uso di altri componenti:} implementa l'interfaccia ModelEvent
				\item\textbf{Attivita' svolte e dati trattati:} segnala al Presenter il verificarsi della modifica di un utente
			\end{itemize}
			\subsubsubsection{Classe UserRemovedEvent}
			\begin{itemize}
		    	\item\textbf{Funzione del componente:} la classe specifica la rimozione di un utente nel database
				\item\textbf{Relazioni d'uso di altri componenti:} implementa l'interfaccia ModelEvent
				\item\textbf{Attivita' svolte e dati trattati:} segnala al Presenter il verificarsi della rimozione di un utente
			\end{itemize}
			
			\subsubsection{Package Parser}
			\begin{center}
				\includegraphics[scale=0.6]{../images/Parser.jpg}
			\end{center}
 			\subsubsubsection{Classe Parser}
 			\begin{itemize}
		    	\item\textbf{Funzione del componente:} controlla che il testo fornito risulti corretto secondo la sintassi QML
			\item\textbf{Attivita' svolte e dati trattati:} il Parser controlla che il testo fornito in input rispetta la sintassi QML e fornisce in caso di errore un messaggio avvertendo l'utente di dove si trova l'errore e la tipologia
			\end{itemize}
			
			\subsubsection{Package Statistiche}
			\begin{center}
				\includegraphics[scale=0.6]{../images/Statistics.jpg}
			\end{center}
 			\subsubsubsection{Classe Statistic}
 			\begin{itemize}
		    	\item\textbf{Funzione del componente:} questa classe fornisce funzionalita' per il raccoglimento delle statistiche all'interno di Quizzipedia
			\item\textbf{Attivita' svolte e dati trattati:} la classe aggiornera' le statistiche relative ai quesiti, questionari ed utenti.
			Per i quesiti verranno indicati il numero di volte che e' stato proposto e il numero di risposte corrette.
			Per i questionari verranno indicati le valutazioni medie ottenute dagli utenti e il numero di volte che e' stato proposto.
			Per gli utenti verranno indicati la valutazione migliore e la media dei tentativi eseguiti su singolo quiz
			\end{itemize}
			
			\subsection{Diagrammi delle classi della View}
		     \subsubsection{Package CurrentViewManager}
			\begin{center}
				\includegraphics[scale=0.6]{"../images/CurrentViewManager".png}
			\end{center}		    
		    	\subsubsubsection{Classe CurrentView}
		    	\begin{itemize}
		    \item\textbf{Funzione del componente:} lo scopo di questa classe è quello di rappresentare la vista del sistema attualmente disponibile all'utente e di gestirne eventuali cambiamenti 
			\item\textbf{Relazioni d'uso di altri componenti:} (da dire bene) toPresenter, fromPresenter e currentUser sono parametri di tipi appartenenti ad altri package percui c'è una 
relazione di dipendenza (?) verso i rispettivi packages Sender, Receiver ed UserAuthentication
			\item\textbf{Attività svolte e dati trattati:} sono presenti dei parametri per la memorizzazione dell'utente corrente, della pagina corrente e per l'invio e la ricezione di informazioni dal Presenter. La funzionalità principale offerta è enterLink (nome da rivedere) che, come suggerisce il nome, permette all'utente di spostarsi nell'applicazione cambiando pagina. Sono presenti altre funzionalità minori che gestiscono l'avvio e la chiusura della sessione e il ritorno alla homepage.
			\end{itemize}
			\subsubsubsection{Classe Sender}
			\begin{itemize}
			\item\textbf{Funzione del componente:} inoltra le richieste dell'utente al presenter
			\item\textbf{Relazioni d'uso di altri componenti:}  viene utilizzata all'interno di CurrentViewManager::CurrentView in risposta agli eventi che accadono sull'interfaccia web in seguito alle azioni compiute dall'utente; l'output dei metodi di questa classe viene poi processato dalle classi del package Presenter::UserInputManager.
			\item\textbf{Attività svolte e dati trattati:} ogni metodo di questa classe corrisponde ad una situazione che si verifica, attraverso un rispettivo metodo, in una classe derivata da Pages::Page; la classe conosce il tipo Presenter::UserInputManager::Input, e genera messaggi per ogni evento passando oggetti di tipo Input al Presenter.
			\end{itemize}
			
			\subsubsection{Package Pages}
			\begin{center}
				\centerline{\includegraphics[scale=0.6]{"../images/Pages".png}}
			\end{center}
			\subsubsubsection{Classe Page}
			\begin{itemize}
		    \item\textbf{Funzione del componente:} rappresenta una pagina web //meglio renderla interfaccia?
			\item\textbf{Relazioni d'uso di altri componenti:} classe astratta che viene concretizzata dalle sue classi derivate (MainPage, CategoryListPage, QuizListPage,QuizTutorialPage, QuizExecutionPage, QuizManagementPage)
			\item\textbf{Attività svolte e dati trattati:} le funzionalità offerte dalla classe sono compiti fondamentali che una pagina web deve svolgere ovvero ricevere gli input dell'utente (quali click su link o dati inseriti in un form che poi andranno passati al presenter) e aggiornarsi quando nuovi dati sono disponibili
			\end{itemize}
			\subsubsubsection{Classe MainPage}
			\begin{itemize}
		    \item\textbf{Funzione del componente:} mostra la pagina principale a cui l'utente arriva entrando nella piattaforma Quizzipedia 
			\item\textbf{Relazioni d'uso di altri componenti:} concretizza la classe astratta Page da cui è diretta discendente (e viene usata da CurrentView come currentPage)
			\item\textbf{Attività svolte e dati trattati:} permette l'autenticazione/registrazione dell'utente nel sistema, da una panoramica del sistema generale all'utente
			\end{itemize}
			\subsubsubsection{Classe CategoryListPage}
			\begin{itemize}
		    \item\textbf{Funzione del componente:} pagina che elenca gli argomenti tra i quali l'utente può scegliere e fornisce operazioni di ordinamento sulla vista 
			\item\textbf{Relazioni d'uso di altri componenti:} concretizza la classe astratta Page da cui è diretta discendente (e viene usata da CurrentView come currentPage)
			\item\textbf{Attività svolte e dati trattati:} le funzionalità di questa classe permettono la visualizzazione delle varie categorie, l'ordinamento per diversi criteri (alfabetico, categorie più visitate, ultimi questionari disponibili...) e la scelta di una tra di esse
			\end{itemize}
			\subsubsubsection{Classe QuizListPage}
			\begin{itemize}
		    \item\textbf{Funzione del componente:} pagina che elenca i questionari (su uno stesso argomento) tra i quali l'utente può scegliere e fornisce operazioni di ordinamento sulla vista
			\item\textbf{Relazioni d'uso di altri componenti:} concretizza la classe astratta Page da cui è diretta discendente (e viene usata da CurrentView come currentPage). L'argomento visualizzato è quello scelto nella precedente pagina CategoryListPage
			\item\textbf{Attività svolte e dati trattati:} le funzionalità di questa classe permettono la visualizzazione di informazioni sui vari questionari, l'ordinamento per diversi criteri (alfabetico, più visitato, novità...) e la scelta di uno tra essi
			\end{itemize}
			\subsubsubsection{Classe QuizExecutionPage}
			\begin{itemize}
		    \item\textbf{Funzione del componente:} questa classe rappresenta il punto focale del sistema Quizzipedia ovvero la parte in cui l'utente svolge i questionari scelti
			\item\textbf{Relazioni d'uso di altri componenti:} concretizza la classe astratta Page da cui è diretta discendente (e viene usata da CurrentView come currentPage). Il questionario visualizzato è quello scelto nella precedente pagina QuizListPage.
			\item\textbf{Attività svolte e dati trattati:} l'utente può navigare tra le domande del questionario nell'ordine che preferisce, dare le proprie risposte e, al termine del questionario, visualizzarne il risultato; per ognuna di queste funzionalità è presente un metodo della classe
			\end{itemize}
			\subsubsubsection{Classe QuizManagementPage}
			\begin{itemize}
		    \item\textbf{Funzione del componente:} gestisce creazione, modifica ed eliminazione di singole domande e interi questionari
			\item\textbf{Relazioni d'uso di altri componenti:} concretizza la classe astratta Page da cui è diretta discendente (e viene usata da CurrentView come currentPage)
			\item\textbf{Attività svolte e dati trattati:} le funzionalità offerte dalla classe consentono creazione, modifca ed eliminazione di singoli quesiti e di interi questionari
			\end{itemize}
			\subsubsubsection{Classe QuizTutorialPage}
			\begin{itemize}
		    \item\textbf{Funzione del componente:} questa pagina visualizza un breve manuale che spiega l'uso e la sintassi del linguaggio QML (vedi analisi\_dei\_requisiti\_2.0.pdf)
			\item\textbf{Relazioni d'uso di altri componenti:} concretizza la classe astratta Page da cui è diretta discendente (e viene usata da CurrentView come currentPage)
			\item\textbf{Attività svolte e dati trattati:} questa classe svolge solamente una semplice attività di visualizzazione di informazioni
			\end{itemize}
			\subsubsection{Package UserAuthentication}
			\begin{center}
				\includegraphics[scale=0.6]{"../images/UserAuthentication".png}
			\end{center}	
			\subsubsubsection{Classe User}
			\begin{itemize}
\item\textbf{Funzione del componente:} classe che rappresenta il singolo utilizzatore attuale del sistema nella propria sessione
\item\textbf{Relazioni d'uso di altri componenti:} User viene utilizzato dalla classe CurrentViewManager::CurrentView
\item\textbf{Attività svolte e dati trattati:} questa classe prevede le funzionalità basilari per l'autenticazione dell'utente all'interno del sistema come registrazione, login e  logout ed altre operazioni secondarie come il cambio o il recupero della password smarrita.
			\end{itemize}
			\newpage
			
			\subsection{Diagrammi delle classi del Presenter}
			\subsubsection{Package UserInputManager}
			\begin{figure}[h!]
			\begin{center}
				\includegraphics[scale=0.65]{../images/UserInputManagerClass.png}
				\caption{Diagramma delle classi del package UserInputManager}
			\end{center}
			\end{figure}

			\subsubsubsection{Classe InputManager}
			La classe implementa, con riferimento al pattern \emph{Command}, il componente Invoker del Command. In questo modo il trattamento degli input viene disaccoppiato dalle componenti che ne eseguiranno le richieste.
			\begin{itemize}
				\item\textbf{Funzione del componente}: riceve gli input dalla View e delega alle componenti del package Updaters la loro risoluzione.
				\item\textbf{Relazione d'uso di altre componenti}: si interfaccia con il package Updaters (che sono i suoi \emph{Receivers}) e con il package View dal quale riceve gli input.
				\item\textbf{Attività svolte e dati trattati}: questa classe possiede operazioni per ricevere e indirizzare ai giusti componenti gli input che le arrivano dalla View.
			\end{itemize}
			
			\subsubsubsection{Interfaccia Input}
			L'interfaccia implementa l'interfaccia di esecuzione delle richieste; con riferimento al pattern \emph{Command}, essa corrisponde alla classe Command. Quest'implementazione permette di aggiungere facilmente nuovi tipi di Input (concretizzazioni del Command).
			\begin{itemize}
				\item\textbf{Funzione del componente}: interfaccia di base degli input.
				\item\textbf{Relazione d'uso di altre componenti}: viene concretizzata in più sottoclassi. Oggetti delle sue sottoclassi vengono creati da View::Sender. Le sottoclassi concrete s'interfacciano con Updaters::ViewUpdater e Updaters::ModelUpdater.
				\item\textbf{Attività svolte e dati trattati}: definisce il contratto di ogni classe di tipo Input. Le classi concrete s'interfacciano con le classi del package Updaters in modo specializzato per ogni input.
				\item\textbf{Concretizzazioni}: Attualmente l'interfaccia Input viene concretizzata nelle seguenti sottoclassi (paragonabili, con riferimento al pattern \emph{Command}, a classi di tipo\emph{ConcreteCommand}), così definite:
				\begin{itemize}
				\item\textit{CreateQuiz}: richiede l'interfaccia per la creazione di un nuovo questionario.
				\item\textit{AddQuestion}: richiede di aggiungere una domanda al questionario in creazione.
				\item\textit{RemoveQuestion}: richiede di rimuovere una domanda precedentemente aggiunta nel questionario in creazione.
				\item\textit{SaveQuiz}: richiede di salvare un nuovo questionario appena creato.
				\item\textit{CreateQuestion}: richiede l'interfaccia per la creazione di una nuova domanda.
				\item\textit{SaveQuestion}: richiede di salvare nel sistema una nuova domanda.
				\item\textit{DeleteQuestion}: richiede di cancellare dal sistema una domanda precedentemente creata.
				\item\textit{ChooseQuiz}: richiede di caricare e presentare all'utente il quiz scelto.
				\item\textit{StartQuiz}: richiede di iniziare lo svolgimento di un quiz scelto.
				\item\textit{NextQuestion}: richiede la domanda successiva in un quiz.
				\item\textit{PreviousQuestion}: richiede la domanda precedente in un quiz.
				\item\textit{EndQuiz}: richiede di consegnare un quiz svolto completamente o in parte.
				\item\textit{ViewProfile}: richiede la pagina di visualizzazione del profilo utente.
				\item\textit{UpdateProfile}: richiede di modificare le informazioni nel profilo utente.
				\item\textit{ViewQMLTutorial}: richiede la visualizzazione del manuale utente QML.
				\item\textit{ViewQuizList}: richiede la visualizzazione di una lista di quiz.
				\end{itemize}
			\end{itemize}

			\subsubsection{Package Updaters}
			\begin{figure}[h!]
			\begin{center}
				\includegraphics[scale=0.65]{../images/UpdatersClass.png}
				\caption{Diagramma delle classi del package Updaters}
			\end{center}
			\end{figure}
			\subsubsubsection{Interfaccia Updater}
			\begin{itemize}
				\item\textbf{Funzione del componente}: definisce le operazioni di base che andranno implementate nelle sottoclassi.
				\item\textbf{Relazione d'uso di altre componenti}: viene concretizzata in ViewUpdater e ModelUpdater.
				\item\textbf{Attività svolte e dati trattati}: nessuno.
			\end{itemize}
			
			\subsubsubsection{Classe ModelUpdater}
			\begin{itemize}
				\item\textbf{Funzione del componente}: inoltra al Model i dati da salvare su database. Riceve eventi dal Database.
				\item\textbf{Relazione d'uso di altre componenti}: collabora con le classi ViewUpdater, QuizManager e QuestionManager e s'interfaccia con la classe Database del Model esterna al package Presenter.
				\item\textbf{Attività svolte e dati trattati}: possiede un riferimento al Model e ne riceve gli eventi; tali eventi possono comportare l'aggiornamento della View, in tal caso collabora con la classe ViewUpdater. E' il \emph{Receiver} degli input che modificano il Model. Alcuni input possono richiedere la sua collaborazione con le classi QuizManager e QuestionManager.
			\end{itemize}

			\subsubsubsection{Classe ViewUpdater}
			\begin{itemize}
				\item\textbf{Funzione del componente}: è responsabile dell'aggiornamento della View rispetto agli input utente e rispetto alle modifiche del Model. E' il \emph{Receiver} degli input che modificano la View.
				\item\textbf{Relazione d'uso di altre componenti}: s'interfaccia con le classi ViewUpdater, QuizManager e QuestionManager e con le classi del package View.
				\item\textbf{Attività svolte e dati trattati}: la componente può ricevere input che non necessitano di interazione col Model; in tal caso provvede direttamente all'aggiornamento della View. Se è richiesta interazione col Model essa interagisce con la classe ModelUpdater. Alcuni input possono richiedere la sua collaborazione con le classi QuizManager e QuestionManager.
			\end{itemize}
			\newpage
			
			\subsubsection{Package QuestionManager}	
			Il package fa uso del design pattern \emph{Abstract Factory}.
			\begin{figure}[h!]
			\begin{center}
				\includegraphics[scale=0.65]{../images/QuestionManagerClass.png}
				\caption{Diagramma delle classi del package QuestionManager}
			\end{center}
			\end{figure}
			\subsubsubsection{Interfaccia QuestionFactory}
			\begin{itemize}
				\item\textbf{Funzione del componente}: interfaccia di base delle Factory di tipi Question.
				\item\textbf{Relazione d'uso di altre componenti}: può essere concretizzata in diversi tipi di QuestionFactory.
				\item\textbf{Attività svolte e dati trattati}: definisce il contratto delle factory Question, cioè le operazioni di costruzione di Question che saranno definite in ogni concretizzazione.
			\end{itemize}
			
			\subsubsubsection{Classe astratta Question}
			\begin{itemize}
				\item\textbf{Funzione del componente}: classe di base del tipo Question.
				\item\textbf{Relazione d'uso di altre componenti}: può essere concretizzata in diversi tipi di Question.
				\item\textbf{Attività svolte e dati trattati}: definisce il contratto generale delle Question, cioè le loro operazioni e dati.
			\end{itemize}
			
			\subsubsubsection{Classe QML2HTMLQuestionFactory}
			La classe QML2HTMLQuestionFactory è un \emph{singleton}.
			\begin{itemize}
				\item\textbf{Funzione del componente}: crea oggetti di tipo HTMLQuestion.
				\item\textbf{Relazione d'uso di altre componenti}: è concretizzazione della classe QuestionFactory. Crea oggetti HTMLQuestion. Si interfaccia con la classe Translator.
				\item\textbf{Attività svolte e dati trattati}: crea su richiesta oggetti di tipo HTMLQuestion, delegando la traduzione del codice QML alla classe Translator.
			\end{itemize}
			
			\subsubsubsection{Classe HTMLQuestion}	
			\begin{itemize}
				\item\textbf{Funzione del componente}: rappresenta una domanda in formato HTML.
				\item\textbf{Relazione d'uso di altre componenti}: è concretizzazione di Question.
				\item\textbf{Attività svolte e dati trattati}: eredita e specializza le funzionalità dell'interfaccia Question.
			\end{itemize}
			
			\subsubsubsection{Classe Translator}
			\begin{itemize}
				\item\textbf{Funzione del componente}: s'interfaccia con le componenti del package Interpreter per la traduzione di quesiti QML in HTML.
				\item\textbf{Relazione d'uso di altre componenti}: collabora con le interfacce Interpreter e InterpreterFactory.
				\item\textbf{Attività svolte e dati trattati}: riceve dalla classe ViewUpdater le richieste di traduzione e il codice QML da tradurre. Attraverso la factory InterpreterFactory (una sua concretizzazione) costruisce un Interpreter concreto e lo utilizza per la traduzione del codice. L'esito della traduzione viene reso disponibile a ViewUpdater.
			\end{itemize}
			\newpage
			
			\subsubsection{Package QuizManager}
			Il package fa uso del design pattern \emph{Builder}.
			\begin{figure}[h!]
			\begin{center}
				\includegraphics[scale=0.65]{../images/QuizManagerClass.png}
				\caption{Diagramma delle classi del package QuizManager}
			\end{center}
			\end{figure}
			\subsubsubsection{Classe QuizDirector}
			\begin{itemize}
			\item\textbf{Funzione del componente}: è responsabile della costruzione di un oggetto di tipo Quiz.
				\item\textbf{Relazione d'uso di altre componenti}: utilizza la classe QuizBuilder.
				\item\textbf{Attività svolte e dati trattati}: la classe, a partire da un set di domande (in questo caso in QML), utilizza il Builder per costruire un Quiz.ì
			\end{itemize}
			
			\subsubsubsection{Classe QuizBuilder}
			\begin{itemize}
			\item\textbf{Funzione del componente}: Costruisce domanda per domanda un questionario.
				\item\textbf{Relazione d'uso di altre componenti}: interagisce con il package QuestionManager.
				\item\textbf{Attività svolte e dati trattati}: Costruisce passo pe r passo un oggetto di tipo Quiz. Si appoggia al package QuestionManager per la creazione delle singole domande.
			\end{itemize}
			
			\subsubsubsection{Classe Quiz}
			\begin{itemize}
			\item\textbf{Funzione del componente}: la classe rappresenta un Quiz, una raccolta di domande.
				\item\textbf{Relazione d'uso di altre componenti}: nessuna.
				\item\textbf{Attività svolte e dati trattati}: la classe possiede i dati e fornisce le operazioni necessaria alla fruizione di un Quiz.
			\end{itemize}
			\clearpage			
			
			\subsubsection{Package Interpreter}
			\begin{figure}[h!]
			\begin{center}
				\includegraphics[scale=0.65]{../images/InterpreterClass.png}
				\caption{Diagramma delle classi del package Interpreter}
			\end{center}
			\end{figure}
			\subsubsubsection{Interfaccia Interpreter}
			\begin{itemize}
				\item\textbf{Funzione del componente}: interfaccia di base del tipo Interpreter.
				\item\textbf{Relazione d'uso di altre componenti}: può essere concretizzata in diversi tipi di Interpreter. Viene riferita dalla classe Translator.
				\item\textbf{Attività svolte e dati trattati}: definisce il contratto degli Interpreter, cioè le operazioni di traduzione che saranno definite in ogni concretizzazione.
			\end{itemize}
			\subsubsubsection{Interfaccia InterpreterFactory}
			\begin{itemize}
				\item\textbf{Funzione del componente}: interfaccia di base delle Factory di tipi Interpreter.
				\item\textbf{Relazione d'uso di altre componenti}: può essere concretizzata in diversi tipi di InterpreterFactory. Viene riferita dalla classe Translator.
				\item\textbf{Attività svolte e dati trattati}: definisce il contratto delle factory, cioè le operazioni di costruzione di Interpreter che saranno definite in ogni concretizzazione.
			\end{itemize}
			\subsubsubsection{Classe QMLInterpreterFactory}
			La classe QMLInterpreterFactory è un \emph{singleton}.
			\begin{itemize}
				\item\textbf{Funzione del componente}: crea oggetti di tipo QMLInterpreter.
				\item\textbf{Relazione d'uso di altre componenti}: è concretizzazione della classe InterpreterFactory. Crea oggetti QMLInterpreter.
				\item\textbf{Attività svolte e dati trattati}: crea su richiesta oggetti di tipo QMLInterpreter.
			\end{itemize}
			\subsubsubsection{Classe QMLInterpreter}
			\begin{itemize}
				\item\textbf{Funzione del componente}: classe astratta che rappresenta gli Interpreter che traducono codice QML in un altro formato.
				\item\textbf{Relazione d'uso di altre componenti}: è sottotipo di Interpreter. Può essere concretizzata in più tipi di QMLInterpreter.
				\item\textbf{Attività svolte e dati trattati}: definisce il contratto dei QMLInterpreter, cioè le operazioni di traduzione da QML verso altri linguaggi.
			\end{itemize}
			\subsubsubsection{Classe QML2HTMLInterpreter}	
			\begin{itemize}
				\item\textbf{Funzione del componente}: traduce codice QML in codice HTML.
				\item\textbf{Relazione d'uso di altre componenti}: è concretizzazione di QMLInterpreter.
				\item\textbf{Attività svolte e dati trattati}: riceve in input domande in QML e le traduce in codice HTML visualizzabile da browser.
			\end{itemize}
			\newpage
	\section{Diagrammi di attività}
	
\subsection{Creazione Questionario}
\begin{figure}[h!]
\begin{center}
	\centerline{\includegraphics[scale=0.1]{"../images/diagramma 1".png}}
	\caption{Diagramma di attività della creazione di un questionario}
\end{center}
\end{figure}
\begin{itemize}
\item\textbf{Precondizioni:} l'utente è autenticato nel sistema Quizzipedia e accede all'area del sito dedicata alla gestione (creazione, eliminazione e modifica) dei propri questionari.
\item\textbf{Postcondizioni:} l'utente ha creato con successo (o ha annullato la creazione di) un questionario su un argomento a piacere contenente domande già disponibili o create precedentemente da lui stesso. L'utente viene reindirizzato alla pagina principale. Il questionario creato può ora essere svolto da altri utenti del sistema Quizzipedia. %da rivedere dove viene reindirizzato l'utente
\item\textbf{Descrizione:} l'utente inizialmente assegna un titolo al questionario, sceglie l'argomento trattato tra le categorie presenti nel sistema e assegna un tempo limite entro il quale il questionario deve essere completato. A questo punto l'utente può iniziare ad inserire domande nel questionario: di volta in volta può scegliere tra le domande già disponibili sull'argomento (domande precaricate dal team404 e domande create da altri utenti del sistema) o tra le domande create da lui stesso (vedi diagramma d'attività in sezione successiva). Quando l'utente ha terminato di inserire le domande nel proprio questionario ne verrà presentata un'anteprima. L'utente ha infine la possibilità di annullare la creazione del questionario (e quindi tornare alla schermata precedente perdendo però i dati immessi in precedenza) o di confermarla con conseguente pubblicazione nel sito dove sarà disponibile agli altri utenti.
\end{itemize}

\newpage
\subsection{Creazione Domanda}
\begin{figure}[h!]
\begin{center}
	\centerline{\includegraphics[scale=0.1]{"../images/diagramma 2".png}}
	\caption{Diagramma di attività della creazione di una domanda}
\end{center}
\end{figure}
\begin{itemize}
\item\textbf{Precondizioni:} l'utente è autenticato nel sistema Quizzipedia e accede all'area del sito dedicata alla gestione (creazione, eliminazione e modifica) dei propri singoli quesiti.
\item\textbf{Postcondizioni:} l'utente ha creato con successo una nuova domanda. La nuova domanda sarà disponibile a tutti gli utenti qualora essi decidano di creare un questionario appartenente alla stessa categoria. L'utente viene reindirizzato alla pagina di gestione questionari. %anche qua decidere redirect
\item\textbf{Descrizione:} l'utente inizialmente sceglie una categoria (argomento trattato) per la domanda, successivamente procede con la stesura del codice QML in un'area di testo apposita. Qualora l'utente abbia dubbi sull'uso della sintassi è presente nella pagina un link al tutorial interno al sistema. Quando l'utente ha terminato l'inserimento del codice può sottoporlo al sistema che lo valuterà. Dell'input viene eseguito il parsing per controllare anomalie nella sintassi, se sono presenti errori ciò viene segnalato all'utente che ha la possibilità di correggere il codice errato, altrimenti la domanda viene considerata valida e il sistema procede alla sua memorizzazione nel sistema (con conseguente notifica positiva all'utente). Al termine del procedimento la domanda sarà disponibile a tutti gli utenti registrati durante la fase di creazione di questionari.
\end{itemize}
\newpage
\subsection{Compilazione Questionario}
\begin{figure}[h!]
\begin{center}
	\includegraphics[scale=0.1]{"../images/diagramma 3".png}
	\caption{Diagramma di attività della compilazione di un questionario}
\end{center}
\end{figure}
\begin{itemize}
\item\textbf{Precondizioni:} durante le sue ultime interazioni con il sistema Quizzipedia, l'utente ha scelto un argomento e un questionario inerente a quell'argomento che è intenzionato a svolgere. Alternativamente l'utente arriva dall'esterno seguendo un link ad un preciso questionario.
\item\textbf{Postcondizioni:} L'utente ha terminato lo svolgimento del questionario (ha risposto a tutte le domande oppure è scaduto il tempo limite). L'utente viene reindirizzato ad una pagina contenente i risultati della sua prestazione cognitiva.
\item\textbf{Descrizione:} Al momento dell'inizio del questionario (coincidente con il momento in cui l'utente accede alla pagina del questionario) il tempo limite inizia a scorrere, l'utente dovrà rispondere al maggior numero di domande possibili entro lo scadere del tempo (prefissato dal creatore del questionario, vedi sezioni precedenti). Viene presentato un singolo quesito per volta all'utente. L'utente può dare una risposta oppure cambiare domanda. L'utente ha la libertà di scorrere le domande del questionario e di svolgerle in qualsiasi ordine. Se l'utente ha dato una risposta a tutte le domande può consegnare il questionario. Se il tempo limite scade il questionario verrà immediatamente consegnato anche se ancora incompleto. Al momento della consegna il questionario verrà valutato dal sistema che informerà poi l'utente sull'esito della sua performance mediante un redirect ad una apposita pagina contente statistiche e correzioni.
\end{itemize}

\newpage
\subsection{Scelta Questionario}
\begin{figure}[h!]
\begin{center}
	\includegraphics[scale=0.1]{"../images/diagramma 4".png}
	\caption{Diagramma di attività della scelta di un questionario}
\end{center}
\end{figure}
\begin{itemize}
\item\textbf{Precondizioni:} L'utente, intenzionato a svolgere un questionario, accede alla pagina adibita alla selezione dell'argomento.
\item\textbf{Postcondizioni:} L'utente ha scelto un questionario da svolgere e viene reindirizzato alla pagina contenente tale questionario (da qui parte il diagramma "Compilazione Questionario").
\item\textbf{Descrizione:} Inizialmente viene presentato all'utente un elenco di tutti gli argomenti in cui sono suddivisi i questionari del sistema Quizzipedia. L'utente ha la possibilità di ordinare le categorie secondo vari criteri (ordine alfabetico e categorie più recenti) prima di passare alla scelta effettiva di una tra di esse. All'utente viene ora proposta una lista, anch'essa ordinabile secondo vari criteri (alfabetico e cronologico), contenente tutti i questionari disponibili relativi all'argomento scelto in precedenza. Eventualmente l'utente può effettuare una ricerca per nome nella lista qualora stesse cercando uno specifico questionario. L'utente può quindi scegliere definitivamente un questionario da svolgere oppure tornare indietro alla pagina di selezione categorie.
\end{itemize}
\newpage

	\section{Diagrammi di sequenza}
	\subsection{Salvataggio di una Domanda}
	\begin{figure}[h!]
	\begin{center}
		\includegraphics[scale=0.5]{../images/QuestionSaving.png}
		\caption{Diagramma di sequenza del salvataggio di una domanda}
	\end{center}
	\end{figure}
	\begin{itemize}
	\item\textbf{Precondizioni:} l'utente è autenticato nel sistema Quizzipedia e accede all'area del sito dedicata alla gestione (creazione, eliminazione e modifica) dei propri singoli quesiti.\\
	\item\textbf{Postcondizioni:} l'utente ha creato con successo una nuova domanda. La nuova domanda sarà disponibile a tutti gli utenti qualora essi decidano di creare un questionario appartenente alla stessa categoria. L'utente viene reindirizzato alla pagina di gestione questionari.\\
	\item\textbf{Descrizione:} l'utente compila il form per la creazione della nuova domanda (al cui interno ricordiamo è situato un campo adibito alla stesura del codice QML) e sottomette i dati al sistema. L'evento viene notificato al Presenter che chiama il Parser situato nel Model passandogli l'input di testo QML. All'interno del Parser viene effettuata la valutazione del codice che può avere due differenti esiti e portare a diversi scenari di conseguenza. In caso il Parser dia un esito negativo, ciò è semplicemente notificato al Presenter e di conseguenza alla View che si aggiornerà rendendo noto all'utente il fallimento della procedura di creazione. In caso contrario, ovvero il testo QML inserito dall'utente viene considerato valido, il Model provvede a salvare la nuova domanda nel database (e anche di ciò verrà notificato l'utente con un aggiornamento nella View).\\
	\end{itemize}
	
	\subsection{Visualizzazione e scelta di un Questionario}
	\begin{figure}[h!]
	\begin{center}
		\includegraphics[scale=0.6]{../images/ChooseViewQuiz.png}
		\caption{Diagramma di sequenza della visualizzazione e scelta di un questionario}
	\end{center}
	\end{figure}
	\begin{itemize}
	\item\textbf{Precondizioni:} L'utente, intenzionato a svolgere un questionario, accede alla pagina adibita alla selezione dell'argomento.\\
	\item\textbf{Postcondizioni:} L'utente ha scelto un questionario da svolgere e viene reindirizzato alla pagina contenente tale questionario.\\ %è giusta solo questa post o c'è altro?
	\item\textbf{Descrizione:} La classe CurrentView della View interagisce con la classe InputManager
del Presenter richiedendo un elenco di questionari da visualizzare. La richiesta viene
inoltrata al Model che procede a recuperare i dati richiesti e a restituirli alla View. Opzionalmente
l’utente ha la possibilità di ordinare i questionari su schermo secondo vari criteri
(alfabetico o cronologico). L’utente ha poi la possibilità di scegliere un questionario e il flusso di eventi si collegherà con quello espresso nel diagramma della sezione "Caricamento
di un Questionario".
	\end{itemize}
	\newpage
	
	\subsection{Caricamento di un Questionario}
	\begin{figure}[h!]
	\begin{center}
		\includegraphics[scale=0.55]{../images/LoadQuiz.png}
		\caption{Diagramma di sequenza del caricamento di un questionario}
	\end{center}
	\end{figure}
	\begin{itemize}
	\item\textbf{Precondizioni:} L'utente sta entrando in una pagina contenente un questionario che verrà caricato on demand.\\
	\item\textbf{Postcondizioni:} Il questionario è stato correttamente caricato e viene visualizzato a schermo pronto per la risoluzione.\\
	\item\textbf{Descrizione:} L'input iniziale che parte dalla View al Presenter contiene l' identificativo del questionario scelto dall'utente. Il Presenter inoltra successivamente questa informazione al Model che si occuperà di estrarre dal database i dati necessari alla costruzione del questionario (effettuata poi dal QuizManager). Una volta assemblato il questionario è pronto per essere somministrato all'utente (il cui diagramma di sequenza è riportato nella sezione "Svolgimento di un Questionario").
	\end{itemize}
	\newpage
	\subsection{Traduzione di una domanda}
	\begin{figure}[h!]
	\begin{center}
		\includegraphics[scale=0.5]{../images/QuestionTranslation.png}
		\caption{Diagramma di sequenza della traduzione di una domanda QML}
	\end{center}
	\end{figure}
	\begin{itemize}
	\item\textbf{Precondizioni:} Viene richiesta la visualizzazione su schermo di una domanda.
	\item\textbf{Postcondizioni:} La domanda viene convertita dal precedente formato QML al formato HTML interpretabile dal browser.
	\item\textbf{Descrizione:} MANCA SOLO QUESTO
	\end{itemize}
	\newpage
	
	\subsection{Svolgimento di un Questionario}
	\begin{figure}[h!]
	\begin{center}
		\includegraphics[scale=0.5]{../images/QuizExecution.png}
		\caption{Diagramma di sequenza sullo svolgimento di un questionario}
	\end{center}
	\end{figure}
	\begin{itemize}
	\item\textbf{Precondizioni:} L'utente è intenzionato a svolgere un questionario ed è entrato nella pagina corrispondente a tale questionario. Inizialmente tutte le domande risultano senza risposta. \\ % e il tempo limite per la consegna comincia a scorrere
	\item\textbf{Postcondizioni:} L'utente ha svolto il questionario e il sistema, dopo aver compiuto la valutazione, comunica i risultati della prova.\\
	\item\textbf{Descrizione:} Come descritto nel diagramma (evento loop Question Change), l'utente ha la possibilità di scorrere più volte le domande del questionario in modo da risolverle nell'ordine in cui si sente più confortevole. Ogni volta che l'utente richiede la visualizzazione di un'altra domanda il sistema memorizza la risposta fornita dall'utente alla domanda precedente (se questa risposta era effettivamente stata data). Quando un utente ha risposto a tutte le domande può passare alla consegna del questionario, ciò porta al salvataggio delle statistiche della prestazione corrente nella base di dati e al successivo resoconto di valutazione finale che viene esposto all'utente.\\
	\end{itemize}
	\newpage
	\section{Design pattern utilizzati}
	\subsection{Abstract Factory}
	\begin{figure}[h!]
	\begin{center}
		\includegraphics[scale=1]{../images/AbsFactory.png}
		\caption{Design Pattern Abstract Factory}
	\end{center}
	\end{figure}
	\begin{itemize}
		\item\textbf{Scopo}: fornire al Client un'interfaccia per creare famiglie di prodotti, senza dover esplicitare il nome concreto delle classi a cui si riferisce. La factory, quindi, incapsula le responsabilità per la creazione degli oggetti prodotto.
		\item\textbf{Applicabilità}:
		\begin{itemize}
			\item quando si necessita di rendere indipendente un sistema dalla creazione dei suoi componenti, dalla loro rappresentazione e composizione.
			\item quando si necessita la configurazione di un sistema scegliendo tra più tipologie.
			\item quando si devono gestire famiglie di prodotti correlati tra loro e progettati per essere usati insieme.
			\item quando si vuole rivelare solamente l'interfaccia delle classi all'interno di una libreria.
		\end{itemize}
		\item\textbf{Conseguenze}:
		\begin{itemize}
			\item isola le classi concrete, controllandone la creazione. La creazione è delegata
alla classe astratta e i Client manipolano solamente interfacce non conoscendo
i nomi dei prodotti nascosti
			\item permette di modificare facilmente la famiglia di prodotti utilizzati poiché la
classe concreta appare una sola volta nel programma
			\item promuove la coerenza nell'utilizzo dei prodotti poiché sono stati progettati per
essere utilizzati insieme (si usano oggetti di una sola famiglia per volta)
			\item risulta invece difficile l'inserimento di nuove tipologie di prodotto poichè richiederebbe la modifica della classe di Abstract Factory e di conseguentemente il
cambiamento di tutte le sottoclassi (il problema può essere evitato mediante
una tecnica di implementazione della classe astratta, ma in modo meno flessibile e poco sicuro.
		\end{itemize}
		\item\textbf{Utilizzo}: nel sistema Quizzipedia il design pattern è stato utilizzato nei package Interpreter e QuestionManager. Entrambi si avvalgono del pattern per rendere il sistema indipendente dalla creazione delle rispettive classi concrete e rendere i package aperti all'estensione tramite la definizione di nuovi tipi "Interpreter" e "Question".
	\end{itemize}
	\subsection{Singleton}
	\begin{figure}[h!]
	\begin{center}
		\includegraphics[scale=0.7]{../images/Singleton.png}
		\caption{Design Pattern Singleton}
	\end{center}
	\end{figure}
	\begin{itemize}
		\item\textbf{Scopo}: assicurare l'esistenza di una sola istanza di un oggetto, offrendo un punto globale di accesso a tale istanza.
		\item\textbf{Applicabilità}:
		\begin{itemize}
			\item quando deve essere creata una sola istanza di una classe e deve esistere un punto di accesso ad essa noto a tutti Client
			\item quando l'istanza deve poter essere estesa e i Client devono essere in grado di
utilizzarne le istanze senza apportare modifiche al proprio codice
		\end{itemize}
		\item\textbf{Conseguenze}:
		\begin{itemize}
			\item l'accesso all'unica istanza è controllato
			\item permette di aver un miglior uso delle variabili globali, riducendo lo spazio dei nomi senza inquinarlo con variabili utilizzate per memorizzare riferimenti ad
istanze
			\item maggior  flessibilità per quanto riguarda le operazioni di classe, ma maggior
difficoltà nelle modifiche del progetto per utilizzare più istanze. Impossibilità
di sovrascrivere le funzioni di classe poiché statiche e non virtuali.
		\end{itemize}
		\item\textbf{Utilizzo}: nel sistema Quizzipedia il design pattern è stato utilizzato per le classi\\
		 QML2HTMLQuestionFactory e QMLInterpreterFactory. Usato in collaborazione con\\ l'\emph{Abstract Factory}, il pattern assicura che la costruzione degli oggetti Interpreter e Question abbia un unico punto d'accesso nelle classi sopracitate.
	\end{itemize}
	\newpage
	
	\subsection{Builder}
	\begin{figure}[h!]
	\begin{center}
		\includegraphics[scale=1]{../images/Builder.png}
		\caption{Design Pattern Builder}
	\end{center}
	\end{figure}
	\begin{itemize}
		\item\textbf{Scopo}: Separa la costruzione di un oggetto complesso dalla sua rappresentazione.
		\item\textbf{Applicabilità}:
		\begin{itemize}
			\item La procedura di creazione di un oggetto complesso
deve essere indipendente dalle parti che
compongono l'oggetto
			\item Il processo di costruzione deve permettere diverse
rappresentazioni per l'oggetto da costruire.
		\end{itemize}
		\item\textbf{Conseguenze}:
		\begin{itemize}
			\item Facilita le modifiche alla rappresentazione interna di
un prodotto
			\item Isola il codice dedicato alla costruzione di un prodotto
dalla sua rappresentazione
			\item Consente un controllo migliore del processo di
costruzione
		\end{itemize}
		\item\textbf{Utilizzo}: nel sistema Quizzipedia il design pattern è stato utilizzato nel package QuizManager per rendere la creazione di un Quiz indipendente dalle domande che lo compongono.
	\end{itemize}
	\newpage
	
	\subsection{Command}
	\begin{figure}[h!]
	\begin{center}
		\includegraphics[scale=1]{../images/Command.png}
		\caption{Design Pattern Command}
	\end{center}
	\end{figure}
	\begin{itemize}
		\item\textbf{Scopo}: Incapsulare una richiesta in un oggetto, cosicché i
client sia indipendenti dalle richieste
		\item\textbf{Applicabilità}:
		\begin{itemize}
			\item necessità di gestire richieste di cui non si conoscono i
particolari
			\item Parametrizzazione di oggetti sull'azione da eseguire
			\item Specificare, accodare ed eseguire richieste molteplici
volte
			\item Supporto ad operazioni di Undo e Redo e a transazioni.
		\end{itemize}
		\item\textbf{Conseguenze}:
		\begin{itemize}
			\item Accoppiamento "lasco" tra oggetto invocante e
quello che porta a termine l'operazione
			\item I command possono essere estesi
			\item I comandi possono essere composti e innestati
			\item È facile aggiungere nuovi comandi, le classi esistenti non devono essere modificate.
		\end{itemize}
		\item\textbf{Utilizzo}: nel sistema Quizzipedia il design pattern è stato utilizzato nel package Presenter, per disaccoppiare la ricezione degli input da parte della classe InputManager dalla loro risoluzione da parte di altre classi del Presenter e del Model. Il pattern permette inoltre di aggiungere facilmente nuovi \emph{Command} al sistema, estendendo l'interfaccia \emph{Input} del package UserInputManager. Sarà in questo modo più facile estendere le funzionalità offerte all'utente dal sistema Quizzipedia.
	\end{itemize}
	\newpage
	
	\section{Tracciamento}
	\newpage
	\section{Analisi di fattibilità}


\end{document}
