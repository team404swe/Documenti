\documentclass[a4paper,11pt]{article}
	%INCLUDE DEL TEMPLATE
	\input{../../template.tex}

	\title{\textbf{{\fontsize{8mm}{5mm}\selectfont QUIZZIPEDIA}}}
	\date{}
	\author{}	


\begin{document}
	%\title{Piano di Qualifica} comment by Martin
	%\author{Andrea Multineddu}
	\maketitle
	% da qui : add by Martin
	\thispagestyle{empty}
	\begin{center}	
	\includegraphics{../team_not_found.jpg}\\
	\fontsize{5mm}{3mm}\url{team404swe@gmail.com}\\
	
	\vspace{50mm}
	\textbf{Specifica Tecnica 0.1}
	%\'end{center}
	%\'begin{center}
	%\vspace{4mm}
	\end{center}
	\introtab{Specifica Tecnica}			%1 nome documento
			{0.1} 							%2 versione
			{Esterno} 						%3 Uso
			{12 aprile 2016} 				%4 Data cre
			{\today} 						%5 Data mod
			{Marco Crivellaro}	%6 Redazione
			{Alessandro Beccaro } 			%7 Verifica
			{Martin } 				%8 Approvazione
	%fino a qui : add by Martin
	\newpage
	\thispagestyle{empty}
	\null  % add by Martin

	%\null	comment by Martin
	\newpage
	\newpage
	\fancyhead[R]{REGISTRO DELLE MODIFICHE}
	\fancyfoot[R]{\thepage}
	
	\hspace{30 mm}
	\section*{Registro delle modifiche}
	
	\beginregistro
	\rigaregistro{\textbf{Versione}}{\textbf{Autore}}{\textbf{Data}}		 {\hspace{5 mm} \textbf{Descrizione}}
	\rigaregistro{0.1}{Marco Crivellaro (Progettista)}{12/04/2016}{Creazione documento.}
	
			
	\fineregistro
	\newpage
	\fancyhead[R]{\leftmark} % add by Martin 
	\tableofcontents
	\pagenumbering{Roman}

	%\newpage %Comment by Martin
	\listoffigures	
	
	\newpage
	\pagenumbering{arabic}
	
	\section*{Sommario}
	
	
	\newpage
	\section{Introduzione}
	\subsection{Scopo del documento}
	
	
	\subsection{Scopo del prodotto}
	Il progetto \textbf{Quizzipedia} ha come obiettivo lo sviluppo di un sistema software basato su tecnologie Web (Javascript\addglos, Node.js\addglos, HTML5\addglos, CSS3\addglos) che permetta la creazione, gestione e fruizione di questionari. Il sistema dovrà quindi poter archiviare i questionari suddivisi per argomento, le cui domande dovranno essere raccolte attraverso uno specifico linguaggio di markup (Quiz Markup Language) d'ora in poi denominato QML\addglos. In un caso d'uso a titolo esemplificativo, un "esaminatore" dovrà poter costruire il proprio questionario scegliendo tra le domande archiviate, ed il questionario così composto sarà presentato e fruibile all' "esaminando", traducendo l'oggetto QML in una pagina HTML\addglos, tramite un'apposita interfaccia web. Il sistema presentato dovrà inoltre poter proporre questionari preconfezionati e valutare le risposte fornite dall'utente finale.
	\\
	Per un'analisi più precisa ed approfondita del progetto si rimanda al documento\\ "\textit{analisi\_dei\_requisiti\_1.0.pdf}".
	\subsection{Glossario}
	Viene allegato un glossario nel file ``\textit{glossario\_1.0.pdf}'' nel quale viene data una definizione a tutti i termini che in questo documento appaiono con il simbolo '\addglos' a pedice.
	\newpage
	\subsection{Riferimenti}
		\subsubsection{Normativi}
		\begin{itemize}
			\item Capitolato d'appalto Quizzipedia:\\
			\url{http://www.math.unipd.it/~tullio/IS-1/2015/Progetto/C5.pdf}
			\item Norme di Progetto: "\textit{norme\_di\_progetto\_1.0.pdf}"
		\end{itemize}
		\subsubsection{Informativi}
		\begin{itemize}
			\item Corso di Ingegneria del Software anno 2015/2016:\\
			\url{http://www.math.unipd.it/~tullio/IS-1/2015/}
			\item Regole del progetto didattico:\\
			\url{http://www.math.unipd.it/~tullio/IS-1/2015/Dispense/PD01.pdf}
			\url{http://www.math.unipd.it/~tullio/IS-1/2015/Progetto/}\\
			\end{itemize}
	\pagebreak
	
\newpage





\end{document}