\section{Tecnologie e strumenti utilizzati}
	\subsection{HTML5}
	Linguaggio di markup per la progettazione di pagine Web. Richiesto espressamente nel capitolato per la creazione dell'interfaccia utente (mi pare, devo controllare).
	\begin{itemize}
		\item\textbf{Utilizzo}: viene utilizzato per creare la GUI che permette all'utente di accedere al sistema mediante browser.
		\item\textbf{Vantaggi}: favorisce la portabilità su diversi dispositivi (desktop, smartphone, tablet...) e browser.
Sono punti a favore anche l'elevata compatibilità con tecnologie quali CSS3 e Javascript.
		\item\textbf{Svantaggi}: essendo HTML5 un linguaggio non ancora standard il rischio nel suo utilizzo è l'instabilità dei tag utilizzati. Un tag oggi accettato potrebbe essere modificato in un futuro prossimo, rendendo la visualizzazione dell'interfaccia utente dipendente dal stabilità dei tag utilizzati. 
	\end{itemize}
	\subsection{CSS3}
	Principale linguaggio usato per la formattazione di pagine HTML.
	\begin{itemize}
		\item\textbf{Utilizzo}: Viene utilizzato per formattare il codice HTML,
ovvero creare fogli di stile che permettono allo sviluppatore di modificare alcuni aspetti grafici della pagina Web.
		\item\textbf{Vantaggi}: Richiede un minor sforzo di interpretazione da parte del browser ed è leggero da scaricare.
		\item\textbf{Svantaggi}: Può presentare problemi di compatibilità con browser meno recenti.
	\end{itemize}
	\subsection{Materialize}
	Materialize è un framework CSS basato sull'idea di design "Material Design" di Google.
	\begin{itemize}
		\item\textbf{Utilizzo}: verrà utilizzato per la formattazione delle pagine web.
		\item\textbf{Vantaggi}: un tale framework aiuta l'applicazione a mantenere uno stile coerente in tutte le sue parti, e mette a disposizione una vasta scelta ci classi CSS preconfezionate, esonerando lo sviluppatore dalla necessita di definirne di proprie. Aumenta inoltra l'adattabilità (responsiveness) delle pagine HTML a diversi supporti (desktop, mobile, etc.).
		\item\textbf{Svantaggi}: necessità di un breve tempo di studio per impararne le classi e il corretto utilizzo.
	\end{itemize}
	\subsection{Javascript}
	E' un linguaggio di scripting debolmente orientato agli oggetti, utilizzato nelle applicazioni
Web. Viene interpretato all'interno del browser. Permette di definire funzionalità simili a
quelle offerte da C++ e Java, quali cicli e strutture di controllo.
Viene solitamente affiancato a pagine statiche HTML per poter gestire i contenuti dinamicamente, offrendo funzionalità che il linguaggio di markup non può offrire.
	\begin{itemize}
		\item\textbf{Utilizzo}: nel sistema Quizzipedia Javascript sarà ampiamente utilizzato. La quasi totalità delle componenti del Presenter sarà realizzata in Javascript, per gestire la parte logica dell'applicazione gli elementi dinamici dell'interfaccia.
		\item\textbf{Vantaggi}: eseguito lato Client (sul browser) non sovraccarica il Server per l'esecuzione di richieste complesse. Permette un maggior livello d'interattività dell'interfaccia utente.
		\item\textbf{Svantaggi}: per script sorgenti molto corposi, può risultare oneroso in termini di
tempo lo scaricamento dei contenuti. Deve, inoltre, fare affidamento ad un linguaggio
che possa fisicamente effettuare transazioni di dati quando lo script esegue operazioni
su oggetti remoti (eg: database).
E' altresì un linguaggio non tipizzato, quindi occorre porre attenzione ai tipi delle
variabili che non sono dichiarati, ma variano dinamicamente.
		\item\textbf{Variabili e oggetti}: le variabili se sono dichiarate all'interno di una funzione sono
visibili solo all'interno di essa; se sono invece esterne sono globali. Vengono dichiarate
con la keyword \emph{var} o semplicemente assegnando loro un valore.
Ogni elemento in JavaScript è un tipo primitivo o un oggetto.
Gli oggetti sono entità dotate di unicità (sono uguali solo a sé stessi) e identificabili
con vettori associativi, che associano nomi di proprietà a valori.
	\end{itemize}
	\subsection{Node.js}
	 Node.js è un framework event-driven per il motore JavaScript V8, relativo all'utilizzo server-side di Javascript.
	\begin{itemize}
		\item\textbf{Utilizzo}: Tecnologia imposta dal proponente per la gestione della base di dati, e in generale della parte server, su cui poggia il sistema Quizzipedia.
		\item\textbf{Vantaggi}: Node.js offre prestazioni eccellenti per lo scripting server-side una velocità superiore rispetto alla concorrenza. Il modello event-driven su cui si basa si adatta inoltre molto bene agli scopi del progetto. Secondariamente, questa tecnologia cross-platform ha suscitato molto di più l'interesse del team che se ne intende avvalere rispetto a Tomcat, l'altra tecnologia proposta dal proponente.
		\item\textbf{Svantaggi}: Il design del linguaggio impedisce alcune ottimizzazioni delle operazioni e la gestione dei tipi non è confortevole come in altri concorrenti.
	\end{itemize}
	\subsection{PostgreSQL}
	DBMS ad oggetti open-source.
	\begin{itemize}
		\item\textbf{Utilizzo}: Base di dati con lo scopo di memorizzare domande e altri dati necessari al funzionamento del sistema.
		\item\textbf{Vantaggi}: Tecnologia più robusta, stabile e performante di MySQL (inizialmente preso in considerazione ma poi scartato in favore di PostgreSQL).
		\item\textbf{Svantaggi}: Complessità maggiore rispetto al classico MySQL.
	\end{itemize}
	\newpage