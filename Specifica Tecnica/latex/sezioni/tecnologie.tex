\section{Tecnologie e strumenti utilizzati}
	\subsection{HTML5}
	Linguaggio di markup per la progettazione di pagine Web. Richiesto espressamente nel capitolato per la creazione dell'interfaccia utente (mi pare, devo controllare).
	\begin{itemize}
		\item\textbf{Utilizzo}: viene utilizzato per creare la GUI che permette all'utente di accedere al sistema mediante browser.
		\item\textbf{Vantaggi}: favorisce la portabilità su diversi dispositivi (desktop, smartphone, tablet...) e browser.
Sono punti a favore anche l'elevata compatibilità con tecnologie quali CSS3 e Javascript.
		\item\textbf{Svantaggi}: essendo HTML5 un linguaggio non ancora standard il rischio nel suo utilizzo è l'instabilità dei tag utilizzati. Un tag oggi accettato potrebbe essere modificato in un futuro prossimo, rendendo la visualizzazione dell'interfaccia utente dipendente dal stabilità dei tag utilizzati. 
	\end{itemize}
	\subsection{CSS3}
	Principale linguaggio usato per la formattazione di pagine HTML.
	\begin{itemize}
		\item\textbf{Utilizzo}: Viene utilizzato per formattare il codice HTML,
ovvero creare fogli di stile che permettono allo sviluppatore di modificare alcuni aspetti grafici della pagina Web.
		\item\textbf{Vantaggi}: Richiede un minor sforzo di interpretazione da parte del browser ed è leggero da scaricare.
		\item\textbf{Svantaggi}: Può presentare problemi di compatibilità con browser meno recenti.
	\end{itemize}
	\subsection{Materialize}
	Materialize è un framework CSS basato sull'idea di design "Material Design" di Google.
	\begin{itemize}
		\item\textbf{Utilizzo}: verrà utilizzato per la formattazione delle pagine web.
		\item\textbf{Vantaggi}: un tale framework aiuta l'applicazione a mantenere uno stile coerente in tutte le sue parti, e mette a disposizione una vasta scelta ci classi CSS preconfezionate, esonerando lo sviluppatore dalla necessita di definirne di proprie. Aumenta inoltra l'adattabilità (responsiveness) delle pagine HTML a diversi supporti (desktop, mobile, etc.).
		\item\textbf{Svantaggi}: necessità di un breve tempo di studio per impararne le classi e il corretto utilizzo.
	\end{itemize}
	\subsection{Javascript}
	E' un linguaggio di scripting debolmente orientato agli oggetti, utilizzato nelle applicazioni
Web. Viene interpretato all'interno del browser. Permette di definire funzionalità simili a
quelle offerte da C++ e Java, quali cicli e strutture di controllo.
Viene solitamente affiancato a pagine statiche HTML per poter gestire i contenuti dinamicamente, offrendo funzionalità che il linguaggio di markup non può offrire.
	\begin{itemize}
		\item\textbf{Utilizzo}: nel sistema Quizzipedia Javascript sarà ampiamente utilizzato. La quasi totalità delle componenti della ViewModel sarà realizzata in Javascript, per gestire la parte logica dell'applicazione gli elementi dinamici dell'interfaccia.
		\item\textbf{Vantaggi}: eseguito lato Client (sul browser) non sovraccarica il Server per l'esecuzione di richieste complesse. Permette un maggior livello d'interattività dell'interfaccia utente.
		\item\textbf{Svantaggi}: per script sorgenti molto corposi, può risultare oneroso in termini di
tempo lo scaricamento dei contenuti. Deve, inoltre, fare affidamento ad un linguaggio
che possa fisicamente effettuare transazioni di dati quando lo script esegue operazioni
su oggetti remoti (eg: database).
E' altresì un linguaggio non tipizzato, quindi occorre porre attenzione ai tipi delle
variabili che non sono dichiarati, ma variano dinamicamente.
		\item\textbf{Variabili e oggetti}: le variabili se sono dichiarate all'interno di una funzione sono
visibili solo all'interno di essa; se sono invece esterne sono globali. Vengono dichiarate
con la keyword \emph{var} o semplicemente assegnando loro un valore.
Ogni elemento in JavaScript è un tipo primitivo o un oggetto.
Gli oggetti sono entità dotate di unicità (sono uguali solo a sé stessi) e identificabili
con vettori associativi, che associano nomi di proprietà a valori.
	\end{itemize}
	\subsection{Node.js}
	 Node.js è un framework event-driven per il motore JavaScript V8, relativo all'utilizzo server-side di Javascript.
	\begin{itemize}
		\item\textbf{Utilizzo}: Tecnologia imposta dal proponente per la gestione della base di dati, e in generale della parte server, su cui poggia il sistema Quizzipedia.
		\item\textbf{Vantaggi}: Node.js offre prestazioni eccellenti per lo scripting server-side una velocità superiore rispetto alla concorrenza. Il modello event-driven su cui si basa si adatta inoltre molto bene agli scopi del progetto. Secondariamente, questa tecnologia cross-platform ha suscitato molto di più l'interesse del team che se ne intende avvalere rispetto a Tomcat, l'altra tecnologia proposta dal proponente.
		\item\textbf{Svantaggi}: Il design del linguaggio impedisce alcune ottimizzazioni delle operazioni e la gestione dei tipi non è confortevole come in altri concorrenti.
	\end{itemize}
	\subsection{MongoDB}
	MongoDB è un DBMS non relazionale, orientato ai documenti. Classificato come un database di tipo NoSQL, MongoDB si allontana dalla struttura tradizionale basata su tabelle dei database relazionali in favore di documenti in stile JSON con schema dinamico rendendo l'integrazione di dati di alcuni tipi di applicazioni più facile e veloce.
	\begin{itemize}
		\item\textbf{Utilizzo}: Base di dati con lo scopo di memorizzare domande e altri dati necessari al funzionamento del sistema.
		\item\textbf{Vantaggi}: I dati non sono ristretti da alcun tipo di schema, il sistema è facilmente scalabile in caso di necessità, il livello di consistenza dei dati può essere definito a piacere, tecnologia open-source semplice da padroneggiare, in particolare il gruppo è già familiare con la sintassi JSON usata da MongoDB per la memorizzazione delle informazioni.
		\item\textbf{Svantaggi}: Minore flessibilità per quanto riguarda la formulazione delle query e elevata dimensione dei dati su server rispetto a tecnologie più tradizionali.
	\end{itemize}
	\subsection{AngularJS}
	AngularJS è un framework strutturale per la costruzione di applicazioni web. Permette di estendere la sintassi HTML con componenti della propria applicazione mantenendo comunque una stretta separazione tra i dati e la loro rappresentazione rispettando il pattern MVC.
	\begin{itemize}
		\item\textbf{Utilizzo}: Verrà utilizzato come renderer dei template della user interface di cui si occupa Meteor.
		\item\textbf{Vantaggi}: Fornisce la possibilità di creare Single Page Application in modo pulito e mantenibile facendo utilizzo di dependency injection e rispettando il principio di separazione degli interessi. Le componenti costruite con AngularJS sono facilmente riusabili e testabili. Elevata compatibilità con diversi tipi di browser.
		\item\textbf{Svantaggi}: Essendo un framework scritto interamente in Javascript, le applicazioni costruite in Angular non sono molto robuste sotto il punto di vista della sicurezza inoltre, in caso l'utente disabiliti Javascript tutto il funzionamento dell'applicazione verrebbe compromesso.
	\end{itemize}
	\subsection{Meteor}
	Meteor è una piattaforma full-stack Javascript per la costruzione di mobile e web apps. Semplifica la creazione di applicazioni real time offrendo un ecosistema completo atto al loro sviluppo e alla loro fruizione.
	\begin{itemize}
		\item\textbf{Utilizzo}: Questa tecnologia contribuirà alla realizzazione del cuore principale dell'applicazione Quizzipedia.
		\item\textbf{Vantaggi}: Framework abbastanza semplice da imparare ad utilizzare, ideale per lo sviluppo di applicazioni web real time, sia front-end che back-end vengono codificati principalmente usando solo Javascript, presenza di package che semplificano e velocizzano lo sviluppo dell'applicazione, alta scalabilità del progetto.
		\item\textbf{Svantaggi}: Il modo in cui alcuni componenti core della tecnologia si interfacciano potrebbe limitare la libertà degli sviluppatori. Esperienze precedenti con la tecnologia scarse o nulle dei componenti del Team 404, tempi abbastanza considerevoli dovranno essere impiegati nel conseguimento di una buona padronanza del framework.
	\end{itemize}
	\newpage