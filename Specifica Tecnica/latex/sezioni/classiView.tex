\subsection{Diagrammi delle classi della View}
		     \subsubsection{Package CurrentViewManager}
			\begin{center}
				\includegraphics[scale=0.6]{"../images/CurrentViewManager".png}
			\end{center}		    
		    	\subsubsubsection{Classe CurrentView}
		    	\begin{itemize}
		    \item\textbf{Funzione del componente:} lo scopo di questa classe è quello di rappresentare la vista del sistema attualmente disponibile all'utente e di gestirne eventuali cambiamenti 
		    \item\textbf{Relazioni d'uso di altri componenti:} toPresenter, currentPage e currentUser sono parametri di tipi appartenenti ad altri package percui c'è una relazione di dipendenza verso i rispettivi packages Sender, Pages ed UserAuthentication
			\item\textbf{Attività svolte e dati trattati:} sono presenti dei parametri per la memorizzazione dell'utente corrente, della pagina corrente e per l'invio e la ricezione di informazioni dal Presenter. La funzionalità principale offerta è enterLink (nome da rivedere) che, come suggerisce il nome, permette all'utente di spostarsi nell'applicazione cambiando pagina. Sono presenti altre funzionalità minori che gestiscono l'avvio e la chiusura della sessione e il ritorno alla homepage.
			\end{itemize}
			\subsubsubsection{Classe Sender}
			\begin{itemize}
			\item\textbf{Funzione del componente:} inoltra le richieste dell'utente al presenter
			\item\textbf{Relazioni d'uso di altri componenti:}  viene utilizzata all'interno di CurrentViewManager::CurrentView in risposta agli eventi che accadono sull'interfaccia web in seguito alle azioni compiute dall'utente; l'output dei metodi di questa classe viene poi processato dalle classi del package Presenter::UserInputManager.
			\item\textbf{Attività svolte e dati trattati:} ogni metodo di questa classe corrisponde ad una situazione che si verifica, attraverso un rispettivo metodo, in una classe derivata da Pages::Page; la classe conosce il tipo Presenter::UserInputManager::Input, e genera messaggi per ogni evento passando oggetti di tipo Input al Presenter.
			\end{itemize}
			
			\subsubsection{Package Pages}
			\begin{center}
				\centerline{\includegraphics[scale=0.6]{"../images/Pages".png}}
			\end{center}
			\subsubsubsection{Classe Page}
			\begin{itemize}
		    \item\textbf{Funzione del componente:} rappresenta una pagina web
			\item\textbf{Relazioni d'uso di altri componenti:} classe astratta che viene concretizzata dalle sue classi derivate (MainPage, CategoryListPage, QuizListPage,QuizTutorialPage, QuizExecutionPage, QuizManagementPage)
			\item\textbf{Attività svolte e dati trattati:} le funzionalità offerte dalla classe sono compiti fondamentali che una pagina web deve svolgere ovvero ricevere gli input dell'utente (quali click su link o dati inseriti in un form che poi andranno passati al presenter) e aggiornarsi quando nuovi dati sono disponibili
			\end{itemize}
			\subsubsubsection{Classe MainPage}
			\begin{itemize}
		    \item\textbf{Funzione del componente:} mostra la pagina principale a cui l'utente arriva entrando nella piattaforma Quizzipedia 
			\item\textbf{Relazioni d'uso di altri componenti:} concretizza la classe astratta Page da cui è diretta discendente (e viene usata da CurrentView come currentPage)
			\item\textbf{Attività svolte e dati trattati:} permette l'autenticazione/registrazione dell'utente nel sistema, da una panoramica del sistema generale all'utente
			\end{itemize}
			\subsubsubsection{Classe CategoryListPage}
			\begin{itemize}
		    \item\textbf{Funzione del componente:} pagina che elenca gli argomenti tra i quali l'utente può scegliere e fornisce operazioni di ordinamento sulla vista 
			\item\textbf{Relazioni d'uso di altri componenti:} concretizza la classe astratta Page da cui è diretta discendente (e viene usata da CurrentView come currentPage)
 +			\item\textbf{Attività svolte e dati trattati:} le funzionalità di questa classe permettono la visualizzazione delle varie categorie, l'ordinamento per diversi criteri (alfabetico, categorie più visitate, ultimi questionari disponibili...) e la scelta di una tra di esse
			\end{itemize}
			\subsubsubsection{Classe QuizListPage}
			\begin{itemize}
		    \item\textbf{Funzione del componente:} pagina che elenca i questionari (su uno stesso argomento) tra i quali l'utente può scegliere e fornisce operazioni di ordinamento sulla vista
		    \item\textbf{Attività svolte e dati trattati:} le funzionalità di questa classe permettono la visualizzazione di informazioni sui vari questionari, l'ordinamento per diversi criteri (alfabetico, più visitato, novità...) e la scelta di uno tra essi
			\item\textbf{Attività svolte e dati trattati:} le funzionalità di questa classe permettono la visualizzazione di informazioni sui vari questionari, l'ordinamento per diversi criteri (alfabetico, più visitato, novità...) e la scelta di uno tra essi
			\end{itemize}
			\subsubsubsection{Classe QuizExecutionPage}
			\begin{itemize}
		    \item\textbf{Funzione del componente:} questa classe rappresenta il punto focale del sistema Quizzipedia ovvero la parte in cui l'utente svolge i questionari scelti
			\item\textbf{Relazioni d'uso di altri componenti:} concretizza la classe astratta Page da cui è diretta discendente (e viene usata da CurrentView come currentPage). Il questionario visualizzato è quello scelto nella precedente pagina QuizListPage.
			\item\textbf{Attività svolte e dati trattati:} l'utente può navigare tra le domande del questionario nell'ordine che preferisce, dare le proprie risposte e, al termine del questionario, visualizzarne il risultato; per ognuna di queste funzionalità è presente un metodo della classe
			\end{itemize}
			\subsubsubsection{Classe QuizManagementPage}
			\begin{itemize}
		    \item\textbf{Funzione del componente:} gestisce creazione, modifica ed eliminazione di singole domande e interi questionari
			\item\textbf{Relazioni d'uso di altri componenti:} concretizza la classe astratta Page da cui è diretta discendente (e viene usata da CurrentView come currentPage)
			\item\textbf{Attività svolte e dati trattati:} le funzionalità offerte dalla classe consentono creazione, modifca ed eliminazione di singoli quesiti e di interi questionari
			\end{itemize}
			\subsubsubsection{Classe QuizTutorialPage}
			\begin{itemize}
		    \item\textbf{Funzione del componente:} questa pagina visualizza un breve manuale che spiega l'uso e la sintassi del linguaggio QML (vedi analisi\_dei\_requisiti\_2.0.pdf)
			\item\textbf{Relazioni d'uso di altri componenti:} concretizza la classe astratta Page da cui è diretta discendente (e viene usata da CurrentView come currentPage)
			\item\textbf{Attività svolte e dati trattati:} questa classe svolge solamente una semplice attività di visualizzazione di informazioni
			\end{itemize}
			\subsubsection{Package UserAuthentication}
			\begin{center}
				\includegraphics[scale=0.6]{"../images/UserAuthentication".png}
			\end{center}	
			\subsubsubsection{Classe User}
			\begin{itemize}
\item\textbf{Funzione del componente:} classe che rappresenta il singolo utilizzatore attuale del sistema nella propria sessione
\item\textbf{Relazioni d'uso di altri componenti:} User viene utilizzato dalla classe CurrentViewManager::CurrentView
\item\textbf{Attività svolte e dati trattati:} questa classe prevede le funzionalità basilari per l'autenticazione dell'utente all'interno del sistema come registrazione, login e  logout ed altre operazioni secondarie come il cambio o il recupero della password smarrita.
			\end{itemize}
			\newpage