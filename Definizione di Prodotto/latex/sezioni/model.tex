\section{Package Model}\begin{figure}[h!]
\begin{center}
	\includegraphics[scale=0.4]{../images/ModelPackage.jpg}
\end{center}
\end{figure}
\subsection{Model::Database}
\subsubsection{Model::Database::QuizManager}
\begin{itemize}
\item\textbf{Funzione del componente:} la classe permettera' l'inserimento, la lettura e la rimozione di questionari all'interno della collezione
\item\textbf{Relazioni d'uso con altre componenti:} ViewModel::Methods::QuestionMethods\\
\item\textbf{Metodi}:
\begin{itemize}
	\item\code{+ addQuiz()} : Metodo per aggiungere un nuovo quiz al database\\
	\textbf{Parametri}:
	\begin{itemize}
		\item\code{ID\_utente: int}\\
		\item\code{Titolo\_quiz: string}\\
		\item\code{Tempo: int}\\
		\item\code{Domande: Array}\\
		\item\code{Categorie: Array}\\
		\textbf{Precondizioni}: Vengono richiesti i dati necessari per la creazione e il salvataggio di un quiz\\
		\textbf{Postcondizioni}: Il quiz e' stato salvato correttamente nel database\\
	\end{itemize}
	\item\code{+ modifyQuiz()} :  Metodo per modificare un quiz nel database\\
	\textbf{Parametri}:
	\begin{itemize}
		\item\code{ID\_utente: int}\\
		\item\code{ID\_quiz: int}\\
		\item\code{Titolo\_quiz: string}\\
		\item\code{Tempo: int}\\
		\item\code{Domande: Array}\\
		\item\code{Categorie: Array}\\
		\textbf{Precondizioni}: Vengono richiesti i dati necessari per la modifica di un quiz\\
		\textbf{Postcondizioni}: Il quiz e' stato modificato nel database\\
	\end{itemize}
	\item\code{+ removeQuiz()} : Metodo per rimuovere un quiz dal database\\
	\textbf{Parametri}:
	\begin{itemize}
		\item\code{ID\_utente: int}\\
		\item\code{ID\_quiz: int}\\
		\textbf{Precondizioni}: Vengono richiesti i dati necessari per la rimozione di un quiz\\
		\textbf{Postcondizioni}: Il quiz e' stato rimosso dal database\\
	\end{itemize}
\end{itemize}
\end{itemize}

\subsubsection{Model::Database::QuestionManager}
\begin{itemize}
\item\textbf{Funzione del componente:} la classe permettera' l'inserimento, la lettura e la rimozione di singoli quesiti all'interno della collezione
\item\textbf{Relazioni d'uso con altre componenti:} ViewModel::Methods::QuizMethods\\
\item\textbf{Metodi}:
	\begin{itemize}
		\item\code{+ addQuestion()} : Metodo per aggiungere un nuovo quesito al database\\
		\textbf{Parametri}:
			\begin{itemize}
				\item\code{ID\_utente: int}\\
				\item\code{testo\_QML: string}\\
				\item\code{categoria: string}\\
				\textbf{Precondizioni}: Vengono richiesti i dati necessari per la creazione e il salvataggio di un nuovo quesito\\
				\textbf{Postcondizioni}: Il quesito e' stato salvato correttamente nel database se viene rispettata la sintassi QML altrimenti viene restituito un messaggio d'errore\\
			\end{itemize}
		\item\code{+ modifyQuestion()} : Metodo per modificare un quesito nel database\\
		\textbf{Parametri}:
			\begin{itemize}
				\item\code{ID\_utente: int}\\
				\item\code{ID\_quesito: int}\\
				\item\code{testo\_QML: string}\\
				\item\code{categoria: string}\\
				\textbf{Precondizioni}: Vengono richiesti i dati necessari per la modifica di un quesito\\
				\textbf{Postcondizioni}: Il quesito e' stato modificato nel database se viene rispettata la sintassi QML altrimenti viene restituito un messaggio d'errore\\
			\end{itemize}
		\item\code{+ removeQuestion()} : Metodo per rimuovere un quesito dal database\\
		\textbf{Parametri}:
			\begin{itemize}
				\item\code{ID\_utente: int}\\
				\item\code{ID\_quesito: int}\\
				\textbf{Precondizioni}: Vengono richiesti i dati necessari per la rimozione di un quesito\\
				\textbf{Postcondizioni}: Il quesito e' stato rimosso dal database\\
			\end{itemize}
	\end{itemize}
\end{itemize}

\subsection{Model::Parser}
\subsubsection{Model::Parser::Parser}
\begin{itemize}
\item\textbf{Funzione del componente:} controlla che il testo fornito risulti corretto secondo la sintassi QML
\item\textbf{Relazioni d'uso con altre componenti:} QuestionManager\\
\item\textbf{Metodi}:
	\begin{itemize}
		\item\code{+ check()} : Metodo per controllare che un nuovo quesito rispetti la sintassi QML prima di essere aggiunto al database\\
		\textbf{Parametri}:
			\begin{itemize}
				\item\code{testo\_da\_controllare: string}\\
				\textbf{Precondizioni}: Viene richiesta una string che contiene la domanda\\
				\textbf{Postcondizioni}: Viene restituito un messaggio contenente il risultato dell'operazione di controllo sintattico\\
			\end{itemize}
	\end{itemize}
\end{itemize}

\subsection{Model::Statistics}
\subsubsection{Model::Statistics::Statistics}
\begin{itemize}
\item\textbf{Funzione del componente:} questa classe fornisce funzionalita' per il raccoglimento delle statistiche sulle prestazioni degli utenti del sistema
\item\textbf{Relazioni d'uso con altre componenti:} Nessuna\\
\item\textbf{Metodi}:
\begin{itemize}
	\item\code{+ updateQuestionStatitics()} : Metodo per aggiornare le statistiche di un quesito nel database\\
	\textbf{Parametri}:
	\begin{itemize}
		\item\code{ID\_quesito: int}\\
		\item\code{Corretta: bool}\\
		\textbf{Precondizioni}: Vengono richiesti i risultati conseguiti dall'utente per lo specifico quesito\\
		\textbf{Postcondizioni}: Le statistiche relative al quesito risultano aggiornate\\
	\end{itemize}
	\item\code{+ updateQuizStatitics()} : Metodo per aggiornare le statistiche di un quiz nel database\\
	\textbf{Parametri}:
	\begin{itemize}
		\item\code{ID\_quiz: int}\\
		\item\code{ID\_utente: int}\\
		\item\code{Risposte: Array}\\
		\textbf{Precondizioni}: Vengono richiesti i risultati conseguiti dall'utente per lo specifico quiz\\
		\textbf{Postcondizioni}: Viene salvato un record relativo ad una singola esecuzione del quiz\\
	\end{itemize}
	\item\code{+ updateUserStatitics()} : Metodo per aggiornare le statistiche di un utente nel database\\
	\textbf{Parametri}:
	\begin{itemize}
		\item\code{ID\_utente: int}\\
		\item\code{quesiti\_corretti: int}\\
		\item\code{quesiti\_totali\_quiz: int}\\
		\textbf{Precondizioni}: Vengono richiesti i risultati conseguiti dall'utente in un quiz\\
		\textbf{Postcondizioni}: Viene aggiornate le statistiche globali dell'utente\\
	\end{itemize}
\end{itemize}
\end{itemize}

\subsection{Model::Publishers}
\subsubsection{Model::Publishers::QuizPublisher}
\begin{itemize}
\item\textbf{Funzione del componente:} questa classe fornisce funzionalita' per rendere pubblica la collezione dei quiz all'avvio del Sistema
\item\textbf{Relazioni d'uso con altre componenti:} Nessuna \\
\item\textbf{Metodi}:
	\begin{itemize}
		\item\code{+ publishQuiz()} : Metodo per rendere accessibili la collezione dei quiz all'avvio del Sistema\\
		\textbf{Precondizioni}: Il Sistema non ha ancora reso pubblica la collezione dei quiz\\
		\textbf{Postcondizioni}: Il Sistema ha reso pubblica la collezione dei quiz\\
	\end{itemize}
\end{itemize}

\subsubsection{Model::Publishers::QuestionPublisher}
\begin{itemize}
\item\textbf{Funzione del componente:} questa classe fornisce funzionalita' per rendere pubblica la collezione dei quesiti all'avvio del Sistema
\item\textbf{Relazioni d'uso con altre componenti:} Nessuna \\
\item\textbf{Metodi}:
	\begin{itemize}
		\item\code{+ publishQuestion()} : Metodo per rendere accessibili la collezione dei quesiti all'avvio del Sistema\\
		\textbf{Precondizioni}: Il Sistema non ha ancora reso pubblica la collezione dei quesiti\\
		\textbf{Postcondizioni}: Il Sistema ha reso pubblica la collezione dei quesiti\\
	\end{itemize}
\end{itemize}