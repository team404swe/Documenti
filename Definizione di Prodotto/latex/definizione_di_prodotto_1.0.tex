\documentclass[a4paper,11pt]{article}
	\input{../../template.tex}
	\newcommand{\code}[1]{\texttt{#1}}

	\title{\textbf{{\fontsize{8mm}{5mm}\selectfont QUIZZIPEDIA}}}
	\date{}
	\author{}	


\begin{document}
	\maketitle
	\thispagestyle{empty}
	\begin{center}	
	\includegraphics{../team_not_found.jpg}\\
	\fontsize{5mm}{3mm}\url{team404swe@gmail.com}\\
	
	\vspace{50mm}
	\textbf{Definizione di Prodotto 1.0}
	\end{center}
	\introtab{Definizione di Prodotto}	%1 nome documento
			{1.0} 							%2 versione
			{Esterno} 						%3 Uso
			{20 maggio 2016} 				%4 Data cre
			{\today} 						%5 Data mod
			{D. Bortot}		%6 Redazione
			{Alex Beccaro } 			%7 Verifica
			{Martin V. Mbouenda} 				%8 Approvazione
	\newpage
	\thispagestyle{empty}
	\null  

	\newpage
	\newpage
	\fancyhead[R]{REGISTRO DELLE MODIFICHE}
	\fancyfoot[R]{\thepage}
	
	\hspace{30 mm}
	\section*{Registro delle modifiche}
	
	\beginregistro
	
	\rigaregistro{\textbf{Versione}}{\textbf{Autore}}{\textbf{Data}}		 {\hspace{5 mm}}
	\rigaregistro{0.0.2}{Davide Bortot (Progettista)}{26/05/2016}{Impostata la struttura della sezione "Presenter" e della sezione "Tracciamento".} 
	\rigaregistro{0.0.1}{Davide Bortot (Progettista)}{20/05/2016}{Creazione e impostazione documento. Definite sezioni "Sommario" e "Introduzione".}
	
	\fineregistro

	\newpage
	\fancyhead[R]{\leftmark}
	\tableofcontents
	\pagenumbering{Roman}
	\newpage
	\listoffigures
	\listoftables
	
	\newpage
	\pagenumbering{arabic}
	
	\section*{Sommario}
	Questo documento, redatto dal gruppo \textbf{Team404}, contiene la descrizione di dettaglio dell'architettura software sulla quale verrà sviluppato il progetto Quizzipedia, commissionato da Zucchetti S.p.A..
Le specifiche del documento si basano sull'architettura generale descritta nel documento "\textit{specifica\_tecnica\_2.0.pdf}".
	
	\newpage
	\section{Introduzione}
	\subsection{Scopo del documento}
	Il documento ha lo scopo di definire nel dettaglio la struttura del sistema Quizzipedia, approfondendone la descrizione già definita nel documento di Specifica Tecnica. Per ogni package del sistema verrà data una descrizione estensiva delle sue classiPer poter sviluppare al meglio il prodotto, i programmatori dovranno attenersi alle specifiche definite in questo documento.
	
	\subsection{Scopo del prodotto}
	Il progetto \textbf{Quizzipedia} ha come obiettivo lo sviluppo di un sistema software basato su tecnologie Web (Javascript\addglos, Node.js\addglos, HTML5\addglos, CSS3\addglos) che permetta la creazione, gestione e fruizione di questionari. Il sistema dovrà quindi poter archiviare i questionari suddivisi per argomento, le cui domande dovranno essere raccolte attraverso uno specifico linguaggio di markup (Quiz Markup Language) d'ora in poi denominato QML\addglos. In un caso d'uso a titolo esemplificativo, un "esaminatore" dovrà poter costruire il proprio questionario scegliendo tra le domande archiviate, ed il questionario così composto sarà presentato e fruibile all' "esaminando", traducendo l'oggetto QML in una pagina HTML\addglos, tramite un'apposita interfaccia web. Il sistema presentato dovrà inoltre poter proporre questionari preconfezionati e valutare le risposte fornite dall'utente finale.
	\\
	Per un'analisi più precisa ed approfondita del progetto si rimanda al documento\\ "\textit{analisi\_dei\_requisiti\_2.0.pdf}".
	\subsection{Glossario}
	Viene allegato un glossario nel file ``\textit{glossario\_2.0.pdf}'' nel quale viene data una definizione a tutti i termini che in questo documento appaiono con il simbolo '\addglos' a pedice.
	\subsection{Riferimenti}
		\subsubsection{Normativi}

		\begin{itemize}
			\item Capitolato d'appalto Quizzipedia:\\
			\url{http://www.math.unipd.it/~tullio/IS-1/2015/Progetto/C5.pdf}
			\item Norme di Progetto: "\textit{norme\_di\_progetto\_2.0.pdf}"
		\end{itemize}
		\subsubsection{Informativi}
		\begin{itemize}
			\item Corso di Ingegneria del Software anno 2015/2016:\\
			\url{http://www.math.unipd.it/~tullio/IS-1/2015/}
			\item Regole del progetto didattico:\\
			\url{http://www.math.unipd.it/~tullio/IS-1/2015/Dispense/PD01.pdf}
			\url{http://www.math.unipd.it/~tullio/IS-1/2015/Progetto/}\\
		\end{itemize}
	\pagebreak
	\newpage
	\section{Architettura generale}
Ciao sono una sezione di prova
	\newpage
	\section{Package View}
\begin{figure}[h!]
\begin{center}
	\includegraphics[scale=0.7]{../images/ViewPackage.png}
\end{center}
\end{figure}

\subsection{View::Pages}
\subsubsection{View::Pages::Page}
\begin{itemize}
\item\textbf{Funzione del componente:} rappresenta una pagina web
				\item\textbf{Relazioni d'uso con altre componenti:} L'interfaccia Page viene concretizzata dalle sue classi derivate, una rappresentativa per ogni pagina dell'applicazione\\ \\
La classe utilizza:
	\begin{itemize}
		\item
	\end{itemize}
\item\textbf{Attributi}:
	\begin{itemize}
		\item\code{}\\
		\item\code{}\\
		\item\code{}\\
		\item\code{}\\
	\end{itemize}
\item\textbf{Metodi}:
	\begin{itemize}
		\item\code{}\\
		\textbf{Parametri}:
			\begin{itemize}
				\item\code{}\\
			\end{itemize}
		\item\code{}\\
		\textbf{Parametri}:
			\begin{itemize}
				\item\code{}\\
			\end{itemize}
		\item\code{}\\
		\textbf{Parametri}:
			\begin{itemize}
				\item\code{}\\
			\end{itemize}
		\item\code{}\\
		\textbf{Parametri}:
			\begin{itemize}
				\item\code{}\\
			\end{itemize}
	\end{itemize}
\end{itemize}

\subsubsection{View::Pages::LoginPage}
\begin{itemize}
\item\textbf{Funzione del componente:} visualizza il form di autenticazione e permette il login dell'utente. Fornisce inoltre un link alla pagina di registrazione e uno alla pagina di recupero della password 
				\item\textbf{Relazioni d'uso con altre componenti:} concretizza l'interfaccia Page da cui è diretta discendente e utilizza il template LoginForm\\ \\
La classe utilizza:
	\begin{itemize}
		\item
	\end{itemize}
\item\textbf{Attributi}:
	\begin{itemize}
		\item\code{}\\
		\item\code{}\\
		\item\code{}\\
		\item\code{}\\
	\end{itemize}
\item\textbf{Metodi}:
	\begin{itemize}
		\item\code{}\\
		\textbf{Parametri}:
			\begin{itemize}
				\item\code{}\\
			\end{itemize}
		\item\code{}\\
		\textbf{Parametri}:
			\begin{itemize}
				\item\code{}\\
			\end{itemize}
		\item\code{}\\
		\textbf{Parametri}:
			\begin{itemize}
				\item\code{}\\
			\end{itemize}
		\item\code{}\\
		\textbf{Parametri}:
			\begin{itemize}
				\item\code{}\\
			\end{itemize}
	\end{itemize}
\end{itemize}

\subsubsection{View::Pages::RegistrationPage}
\begin{itemize}
\item\textbf{Funzione del componente:} visualizza il form di registrazione. Fornisce inoltre un link alla pagina di login
				\item\textbf{Relazioni d'uso con altre componenti:} concretizza l'interfaccia Page da cui è diretta discendente e utilizza il template RegistrationForm\\ \\
La classe utilizza:
	\begin{itemize}
		\item
	\end{itemize}
\item\textbf{Attributi}:
	\begin{itemize}
		\item\code{}\\
		\item\code{}\\
		\item\code{}\\
		\item\code{}\\
	\end{itemize}
\item\textbf{Metodi}:
	\begin{itemize}
		\item\code{}\\
		\textbf{Parametri}:
			\begin{itemize}
				\item\code{}\\
			\end{itemize}
		\item\code{}\\
		\textbf{Parametri}:
			\begin{itemize}
				\item\code{}\\
			\end{itemize}
		\item\code{}\\
		\textbf{Parametri}:
			\begin{itemize}
				\item\code{}\\
			\end{itemize}
		\item\code{}\\
		\textbf{Parametri}:
			\begin{itemize}
				\item\code{}\\
			\end{itemize}
	\end{itemize}
\end{itemize}

\subsubsection{View::Pages::PasswordRecoveryPage}
\begin{itemize}
\item\textbf{Funzione del componente:} visualizza il form per il recupero della password
				\item\textbf{Relazioni d'uso con altre componenti:} concretizza l'interfaccia Page da cui è diretta discendente e utilizza il template PasswordRecoveryForm\\ \\
La classe utilizza:
	\begin{itemize}
		\item
	\end{itemize}
\item\textbf{Attributi}:
	\begin{itemize}
		\item\code{}\\
		\item\code{}\\
		\item\code{}\\
		\item\code{}\\
	\end{itemize}
\item\textbf{Metodi}:
	\begin{itemize}
		\item\code{}\\
		\textbf{Parametri}:
			\begin{itemize}
				\item\code{}\\
			\end{itemize}
		\item\code{}\\
		\textbf{Parametri}:
			\begin{itemize}
				\item\code{}\\
			\end{itemize}
		\item\code{}\\
		\textbf{Parametri}:
			\begin{itemize}
				\item\code{}\\
			\end{itemize}
		\item\code{}\\
		\textbf{Parametri}:
			\begin{itemize}
				\item\code{}\\
			\end{itemize}
	\end{itemize}
\end{itemize}				
				
\subsubsection{View::Pages::ViewTutorialPage}
\begin{itemize}
\item\textbf{Funzione}:
\item\textbf{Relazioni con altre componenti}\\
La classe utilizza:
	\begin{itemize}
		\item
	\end{itemize}
\item\textbf{Attributi}:
	\begin{itemize}
		\item\code{}\\
		\item\code{}\\
		\item\code{}\\
		\item\code{}\\
	\end{itemize}
\item\textbf{Metodi}:
	\begin{itemize}
		\item\code{}\\
		\textbf{Parametri}:
			\begin{itemize}
				\item\code{}\\
			\end{itemize}
		\item\code{}\\
		\textbf{Parametri}:
			\begin{itemize}
				\item\code{}\\
			\end{itemize}
		\item\code{}\\
		\textbf{Parametri}:
			\begin{itemize}
				\item\code{}\\
			\end{itemize}
		\item\code{}\\
		\textbf{Parametri}:
			\begin{itemize}
				\item\code{}\\
			\end{itemize}
	\end{itemize}
\end{itemize}

\subsubsection{View::Pages::QuizCreationPage}
\begin{itemize}
\item\textbf{Funzione del componente:} visualizza il form di creazione di un nuovo questionario
				\item\textbf{Relazioni d'uso con altre componenti:} concretizza l'interfaccia Page da cui è diretta discendente e utilizza il template QuizCreationForm\\ \\
La classe utilizza:
	\begin{itemize}
		\item
	\end{itemize}
\item\textbf{Attributi}:
	\begin{itemize}
		\item\code{}\\
		\item\code{}\\
		\item\code{}\\
		\item\code{}\\
	\end{itemize}
\item\textbf{Metodi}:
	\begin{itemize}
		\item\code{}\\
		\textbf{Parametri}:
			\begin{itemize}
				\item\code{}\\
			\end{itemize}
		\item\code{}\\
		\textbf{Parametri}:
			\begin{itemize}
				\item\code{}\\
			\end{itemize}
		\item\code{}\\
		\textbf{Parametri}:
			\begin{itemize}
				\item\code{}\\
			\end{itemize}
		\item\code{}\\
		\textbf{Parametri}:
			\begin{itemize}
				\item\code{}\\
			\end{itemize}
	\end{itemize}
\end{itemize}

\subsubsection{View::Pages::QuestionUpdatePage}
\begin{itemize}
\item\textbf{Funzione del componente:} visualizza il form di modifica di una domanda
				\item\textbf{Relazioni d'uso con altre componenti:} concretizza l'interfaccia Page da cui è diretta discendente e utilizza il template QuestionForm\\ \\
La classe utilizza:
	\begin{itemize}
		\item
	\end{itemize}
\item\textbf{Attributi}:
	\begin{itemize}
		\item\code{}\\
		\item\code{}\\
		\item\code{}\\
		\item\code{}\\
	\end{itemize}
\item\textbf{Metodi}:
	\begin{itemize}
		\item\code{}\\
		\textbf{Parametri}:
			\begin{itemize}
				\item\code{}\\
			\end{itemize}
		\item\code{}\\
		\textbf{Parametri}:
			\begin{itemize}
				\item\code{}\\
			\end{itemize}
		\item\code{}\\
		\textbf{Parametri}:
			\begin{itemize}
				\item\code{}\\
			\end{itemize}
		\item\code{}\\
		\textbf{Parametri}:
			\begin{itemize}
				\item\code{}\\
			\end{itemize}
	\end{itemize}
\end{itemize}

\subsubsection{View::Pages::QuestionCreationPage}
\begin{itemize}
\item\textbf{Funzione del componente:} visualizza il form di creazione di una nuova domanda
				\item\textbf{Relazioni d'uso con altre componenti:} concretizza l'interfaccia Page da cui è diretta discendente e utilizza il template QuestionForm\\ \\
La classe utilizza:
	\begin{itemize}
		\item
	\end{itemize}
\item\textbf{Attributi}:
	\begin{itemize}
		\item\code{}\\
		\item\code{}\\
		\item\code{}\\
		\item\code{}\\
	\end{itemize}
\item\textbf{Metodi}:
	\begin{itemize}
		\item\code{}\\
		\textbf{Parametri}:
			\begin{itemize}
				\item\code{}\\
			\end{itemize}
		\item\code{}\\
		\textbf{Parametri}:
			\begin{itemize}
				\item\code{}\\
			\end{itemize}
		\item\code{}\\
		\textbf{Parametri}:
			\begin{itemize}
				\item\code{}\\
			\end{itemize}
		\item\code{}\\
		\textbf{Parametri}:
			\begin{itemize}
				\item\code{}\\
			\end{itemize}
	\end{itemize}
\end{itemize}

\subsubsection{View::Pages::QuestionManagementPage}
\begin{itemize}
\item\textbf{Funzione del componente:} visualizza la lista delle domande create dall'utente
				\item\textbf{Relazioni d'uso con altre componenti:} concretizza l'interfaccia Page da cui è diretta discendente e utilizza il template QuestionList\\ \\
La classe utilizza:
	\begin{itemize}
		\item
	\end{itemize}
\item\textbf{Attributi}:
	\begin{itemize}
		\item\code{}\\
		\item\code{}\\
		\item\code{}\\
		\item\code{}\\
	\end{itemize}
\item\textbf{Metodi}:
	\begin{itemize}
		\item\code{}\\
		\textbf{Parametri}:
			\begin{itemize}
				\item\code{}\\
			\end{itemize}
		\item\code{}\\
		\textbf{Parametri}:
			\begin{itemize}
				\item\code{}\\
			\end{itemize}
		\item\code{}\\
		\textbf{Parametri}:
			\begin{itemize}
				\item\code{}\\
			\end{itemize}
		\item\code{}\\
		\textbf{Parametri}:
			\begin{itemize}
				\item\code{}\\
			\end{itemize}
	\end{itemize}
\end{itemize}

\subsubsection{View::Pages::QuizResultsPage}
\begin{itemize}
\item\textbf{Funzione del componente:} visualizza i risultati del questionario appena compilato
				\item\textbf{Relazioni d'uso con altre componenti:} concretizza l'interfaccia Page da cui è diretta discendente e utilizza il template QuizResults
La classe utilizza:
	\begin{itemize}
		\item
	\end{itemize}
\item\textbf{Attributi}:
	\begin{itemize}
		\item\code{}\\
		\item\code{}\\
		\item\code{}\\
		\item\code{}\\
	\end{itemize}
\item\textbf{Metodi}:
	\begin{itemize}
		\item\code{}\\
		\textbf{Parametri}:
			\begin{itemize}
				\item\code{}\\
			\end{itemize}
		\item\code{}\\
		\textbf{Parametri}:
			\begin{itemize}
				\item\code{}\\
			\end{itemize}
		\item\code{}\\
		\textbf{Parametri}:
			\begin{itemize}
				\item\code{}\\
			\end{itemize}
		\item\code{}\\
		\textbf{Parametri}:
			\begin{itemize}
				\item\code{}\\
			\end{itemize}
	\end{itemize}
\end{itemize}

\subsubsection{View::Pages::QuizExecutionPage}
\begin{itemize}
\item\textbf{Funzione del componente:} visualizza un questionario (una domanda alla volta) e tutti i dati relativi (tempo rimasto, numero domande, ecc...)
				\item\textbf{Relazioni d'uso con altre componenti:} concretizza l'interfaccia Page da cui è diretta discendente e utilizza il template QuestionCompilation\\ \\
La classe utilizza:
	\begin{itemize}
		\item
	\end{itemize}
\item\textbf{Attributi}:
	\begin{itemize}
		\item\code{}\\
		\item\code{}\\
		\item\code{}\\
		\item\code{}\\
	\end{itemize}
\item\textbf{Metodi}:
	\begin{itemize}
		\item\code{}\\
		\textbf{Parametri}:
			\begin{itemize}
				\item\code{}\\
			\end{itemize}
		\item\code{}\\
		\textbf{Parametri}:
			\begin{itemize}
				\item\code{}\\
			\end{itemize}
		\item\code{}\\
		\textbf{Parametri}:
			\begin{itemize}
				\item\code{}\\
			\end{itemize}
		\item\code{}\\
		\textbf{Parametri}:
			\begin{itemize}
				\item\code{}\\
			\end{itemize}
	\end{itemize}
\end{itemize}

\subsubsection{View::Pages::QuizListPage}
\begin{itemize}
\item\textbf{Funzione del componente:} visualizza una lista di questionari
				\item\textbf{Relazioni d'uso con altre componenti:} concretizza l'interfaccia Page da cui è diretta discendente e utilizza il template QuizList\\ \\
La classe utilizza:
	\begin{itemize}
		\item
	\end{itemize}
\item\textbf{Attributi}:
	\begin{itemize}
		\item\code{}\\
		\item\code{}\\
		\item\code{}\\
		\item\code{}\\
	\end{itemize}
\item\textbf{Metodi}:
	\begin{itemize}
		\item\code{}\\
		\textbf{Parametri}:
			\begin{itemize}
				\item\code{}\\
			\end{itemize}
		\item\code{}\\
		\textbf{Parametri}:
			\begin{itemize}
				\item\code{}\\
			\end{itemize}
		\item\code{}\\
		\textbf{Parametri}:
			\begin{itemize}
				\item\code{}\\
			\end{itemize}
		\item\code{}\\
		\textbf{Parametri}:
			\begin{itemize}
				\item\code{}\\
			\end{itemize}
	\end{itemize}
\end{itemize}

\subsubsection{View::Pages::CategoryListPage}
\begin{itemize}
\item\textbf{Funzione del componente:} visualizza la lista delle categorie
				\item\textbf{Relazioni d'uso con altre componenti:} concretizza l'interfaccia Page da cui è diretta discendente\\ \\
La classe utilizza:
	\begin{itemize}
		\item
	\end{itemize}
\item\textbf{Attributi}:
	\begin{itemize}
		\item\code{}\\
		\item\code{}\\
		\item\code{}\\
		\item\code{}\\
	\end{itemize}
\item\textbf{Metodi}:
	\begin{itemize}
		\item\code{}\\
		\textbf{Parametri}:
			\begin{itemize}
				\item\code{}\\
			\end{itemize}
		\item\code{}\\
		\textbf{Parametri}:
			\begin{itemize}
				\item\code{}\\
			\end{itemize}
		\item\code{}\\
		\textbf{Parametri}:
			\begin{itemize}
				\item\code{}\\
			\end{itemize}
		\item\code{}\\
		\textbf{Parametri}:
			\begin{itemize}
				\item\code{}\\
			\end{itemize}
	\end{itemize}
\end{itemize}

\subsection{View::Templates}
\subsubsection{View::Templates::QuestionList}
\begin{itemize}
\item\textbf{Funzione del componente:} visualizza una lista di domande
				\item\textbf{Relazioni d'uso con altre componenti:} composta da Question\\ \\
La classe utilizza:
	\begin{itemize}
		\item
	\end{itemize}
\item\textbf{Attributi}:
	\begin{itemize}
		\item\code{}\\
		\item\code{}\\
		\item\code{}\\
		\item\code{}\\
	\end{itemize}
\item\textbf{Metodi}:
	\begin{itemize}
		\item\code{}\\
		\textbf{Parametri}:
			\begin{itemize}
				\item\code{}\\
			\end{itemize}
		\item\code{}\\
		\textbf{Parametri}:
			\begin{itemize}
				\item\code{}\\
			\end{itemize}
		\item\code{}\\
		\textbf{Parametri}:
			\begin{itemize}
				\item\code{}\\
			\end{itemize}
		\item\code{}\\
		\textbf{Parametri}:
			\begin{itemize}
				\item\code{}\\
			\end{itemize}
	\end{itemize}
\end{itemize}

\subsubsection{View::Templates::Question}
\begin{itemize}
\item\textbf{Funzione del componente:} visualizza una domanda inserita in una lista
				\item\textbf{Relazioni d'uso con altre componenti:} usata solo da QuestionList per creare la lista\\ \\
La classe utilizza:
	\begin{itemize}
		\item
	\end{itemize}
\item\textbf{Attributi}:
	\begin{itemize}
		\item\code{}\\
		\item\code{}\\
		\item\code{}\\
		\item\code{}\\
	\end{itemize}
\item\textbf{Metodi}:
	\begin{itemize}
		\item\code{}\\
		\textbf{Parametri}:
			\begin{itemize}
				\item\code{}\\
			\end{itemize}
		\item\code{}\\
		\textbf{Parametri}:
			\begin{itemize}
				\item\code{}\\
			\end{itemize}
		\item\code{}\\
		\textbf{Parametri}:
			\begin{itemize}
				\item\code{}\\
			\end{itemize}
		\item\code{}\\
		\textbf{Parametri}:
			\begin{itemize}
				\item\code{}\\
			\end{itemize}
	\end{itemize}
\end{itemize}

\subsubsection{View::Templates::QuizList}
\begin{itemize}
\item\textbf{Funzione del componente:} visualizza una lista di quiz
				\item\textbf{Relazioni d'uso con altre componenti:} composta da Quiz\\ \\
La classe utilizza:
	\begin{itemize}
		\item
	\end{itemize}
\item\textbf{Attributi}:
	\begin{itemize}
		\item\code{}\\
		\item\code{}\\
		\item\code{}\\
		\item\code{}\\
	\end{itemize}
\item\textbf{Metodi}:
	\begin{itemize}
		\item\code{}\\
		\textbf{Parametri}:
			\begin{itemize}
				\item\code{}\\
			\end{itemize}
		\item\code{}\\
		\textbf{Parametri}:
			\begin{itemize}
				\item\code{}\\
			\end{itemize}
		\item\code{}\\
		\textbf{Parametri}:
			\begin{itemize}
				\item\code{}\\
			\end{itemize}
		\item\code{}\\
		\textbf{Parametri}:
			\begin{itemize}
				\item\code{}\\
			\end{itemize}
	\end{itemize}
\end{itemize}

\subsubsection{View::Templates::Quiz}
\begin{itemize}
\item\textbf{Funzione del componente:} visualizza un quiz inserito in una lista
				\item\textbf{Relazioni d'uso con altre componenti:} usata solo da QuizList per creare la lista\\ \\
La classe utilizza:
	\begin{itemize}
		\item
	\end{itemize}
\item\textbf{Attributi}:
	\begin{itemize}
		\item\code{}\\
		\item\code{}\\
		\item\code{}\\
		\item\code{}\\
	\end{itemize}
\item\textbf{Metodi}:
	\begin{itemize}
		\item\code{}\\
		\textbf{Parametri}:
			\begin{itemize}
				\item\code{}\\
			\end{itemize}
		\item\code{}\\
		\textbf{Parametri}:
			\begin{itemize}
				\item\code{}\\
			\end{itemize}
		\item\code{}\\
		\textbf{Parametri}:
			\begin{itemize}
				\item\code{}\\
			\end{itemize}
		\item\code{}\\
		\textbf{Parametri}:
			\begin{itemize}
				\item\code{}\\
			\end{itemize}
	\end{itemize}
\end{itemize}

\subsubsection{View::Templates::QuestionForm}
\begin{itemize}
\item\textbf{Funzione del componente:} visualizza il form per i dati di una domanda. Utilizzabile sia per la creazione che per la modifica della domanda (se viene modificata una domanda già esistente nei campi vengono inseriti i valori attuali)
\item\textbf{Relazioni con altre componenti}\\
La classe utilizza:
	\begin{itemize}
		\item
	\end{itemize}
\item\textbf{Attributi}:
	\begin{itemize}
		\item\code{}\\
		\item\code{}\\
		\item\code{}\\
		\item\code{}\\
	\end{itemize}
\item\textbf{Metodi}:
	\begin{itemize}
		\item\code{}\\
		\textbf{Parametri}:
			\begin{itemize}
				\item\code{}\\
			\end{itemize}
		\item\code{}\\
		\textbf{Parametri}:
			\begin{itemize}
				\item\code{}\\
			\end{itemize}
		\item\code{}\\
		\textbf{Parametri}:
			\begin{itemize}
				\item\code{}\\
			\end{itemize}
		\item\code{}\\
		\textbf{Parametri}:
			\begin{itemize}
				\item\code{}\\
			\end{itemize}
	\end{itemize}
\end{itemize}

\subsubsection{View::Templates::QuizCreationForm}
\begin{itemize}
\item\textbf{Funzione del componente:} visualizza il form di creazione di un questionario
\item\textbf{Relazioni con altre componenti}\\
La classe utilizza:
	\begin{itemize}
		\item
	\end{itemize}
\item\textbf{Attributi}:
	\begin{itemize}
		\item\code{}\\
		\item\code{}\\
		\item\code{}\\
		\item\code{}\\
	\end{itemize}
\item\textbf{Metodi}:
	\begin{itemize}
		\item\code{}\\
		\textbf{Parametri}:
			\begin{itemize}
				\item\code{}\\
			\end{itemize}
		\item\code{}\\
		\textbf{Parametri}:
			\begin{itemize}
				\item\code{}\\
			\end{itemize}
		\item\code{}\\
		\textbf{Parametri}:
			\begin{itemize}
				\item\code{}\\
			\end{itemize}
		\item\code{}\\
		\textbf{Parametri}:
			\begin{itemize}
				\item\code{}\\
			\end{itemize}
	\end{itemize}
\end{itemize}

\subsubsection{View::Templates::QuestionCompilation}
\begin{itemize}
\item\textbf{Funzione del componente:} visualizza una domanda e ne permette la compilazione
\item\textbf{Relazioni con altre componenti}\\
La classe utilizza:
	\begin{itemize}
		\item
	\end{itemize}
\item\textbf{Attributi}:
	\begin{itemize}
		\item\code{}\\
		\item\code{}\\
		\item\code{}\\
		\item\code{}\\
	\end{itemize}
\item\textbf{Metodi}:
	\begin{itemize}
		\item\code{}\\
		\textbf{Parametri}:
			\begin{itemize}
				\item\code{}\\
			\end{itemize}
		\item\code{}\\
		\textbf{Parametri}:
			\begin{itemize}
				\item\code{}\\
			\end{itemize}
		\item\code{}\\
		\textbf{Parametri}:
			\begin{itemize}
				\item\code{}\\
			\end{itemize}
		\item\code{}\\
		\textbf{Parametri}:
			\begin{itemize}
				\item\code{}\\
			\end{itemize}
	\end{itemize}
\end{itemize}

\subsubsection{View::Templates::QuizResults}
\begin{itemize}
\item\textbf{Funzione del componente:} visualizza i risultati ottenuti in seguito alla compilazione di un quiz
\item\textbf{Relazioni con altre componenti}\\
La classe utilizza:
	\begin{itemize}
		\item
	\end{itemize}
\item\textbf{Attributi}:
	\begin{itemize}
		\item\code{}\\
		\item\code{}\\
		\item\code{}\\
		\item\code{}\\
	\end{itemize}
\item\textbf{Metodi}:
	\begin{itemize}
		\item\code{}\\
		\textbf{Parametri}:
			\begin{itemize}
				\item\code{}\\
			\end{itemize}
		\item\code{}\\
		\textbf{Parametri}:
			\begin{itemize}
				\item\code{}\\
			\end{itemize}
		\item\code{}\\
		\textbf{Parametri}:
			\begin{itemize}
				\item\code{}\\
			\end{itemize}
		\item\code{}\\
		\textbf{Parametri}:
			\begin{itemize}
				\item\code{}\\
			\end{itemize}
	\end{itemize}
\end{itemize}

\subsubsection{View::Templates::RegistrationForm}
\begin{itemize}
\item\textbf{Funzione del componente:} visualizza un form per la registrazione di un nuovo utente
\item\textbf{Relazioni con altre componenti}\\
La classe utilizza:
	\begin{itemize}
		\item
	\end{itemize}
\item\textbf{Attributi}:
	\begin{itemize}
		\item\code{}\\
		\item\code{}\\
		\item\code{}\\
		\item\code{}\\
	\end{itemize}
\item\textbf{Metodi}:
	\begin{itemize}
		\item\code{}\\
		\textbf{Parametri}:
			\begin{itemize}
				\item\code{}\\
			\end{itemize}
		\item\code{}\\
		\textbf{Parametri}:
			\begin{itemize}
				\item\code{}\\
			\end{itemize}
		\item\code{}\\
		\textbf{Parametri}:
			\begin{itemize}
				\item\code{}\\
			\end{itemize}
		\item\code{}\\
		\textbf{Parametri}:
			\begin{itemize}
				\item\code{}\\
			\end{itemize}
	\end{itemize}
\end{itemize}

\subsubsection{View::Templates::LoginForm}
\begin{itemize}
\item\textbf{Funzione del componente:} visualizza un form per l'autenticazione di un utente
\item\textbf{Relazioni con altre componenti}\\
La classe utilizza:
	\begin{itemize}
		\item
	\end{itemize}
\item\textbf{Attributi}:
	\begin{itemize}
		\item\code{}\\
		\item\code{}\\
		\item\code{}\\
		\item\code{}\\
	\end{itemize}
\item\textbf{Metodi}:
	\begin{itemize}
		\item\code{}\\
		\textbf{Parametri}:
			\begin{itemize}
				\item\code{}\\
			\end{itemize}
		\item\code{}\\
		\textbf{Parametri}:
			\begin{itemize}
				\item\code{}\\
			\end{itemize}
		\item\code{}\\
		\textbf{Parametri}:
			\begin{itemize}
				\item\code{}\\
			\end{itemize}
		\item\code{}\\
		\textbf{Parametri}:
			\begin{itemize}
				\item\code{}\\
			\end{itemize}
	\end{itemize}
\end{itemize}

\subsubsection{View::Templates::PasswordRecoveryForm}
\begin{itemize}
\item\textbf{Funzione del componente:} visualizza un form per il recupero della password dimenticata
\item\textbf{Relazioni con altre componenti}\\
La classe utilizza:
	\begin{itemize}
		\item
	\end{itemize}
\item\textbf{Attributi}:
	\begin{itemize}
		\item\code{}\\
		\item\code{}\\
		\item\code{}\\
		\item\code{}\\
	\end{itemize}
\item\textbf{Metodi}:
	\begin{itemize}
		\item\code{}\\
		\textbf{Parametri}:
			\begin{itemize}
				\item\code{}\\
			\end{itemize}
		\item\code{}\\
		\textbf{Parametri}:
			\begin{itemize}
				\item\code{}\\
			\end{itemize}
		\item\code{}\\
		\textbf{Parametri}:
			\begin{itemize}
				\item\code{}\\
			\end{itemize}
		\item\code{}\\
		\textbf{Parametri}:
			\begin{itemize}
				\item\code{}\\
			\end{itemize}
	\end{itemize}
\end{itemize}

\subsubsection{View::Templates::SearchForm}
\begin{itemize}
\item\textbf{Funzione del componente:} visualizza un form per l'inserimento dei dati desiderati e l'avvio della ricerca
\item\textbf{Relazioni con altre componenti}\\
La classe utilizza:
	\begin{itemize}
		\item
	\end{itemize}
\item\textbf{Attributi}:
	\begin{itemize}
		\item\code{}\\
		\item\code{}\\
		\item\code{}\\
		\item\code{}\\
	\end{itemize}
\item\textbf{Metodi}:
	\begin{itemize}
		\item\code{}\\
		\textbf{Parametri}:
			\begin{itemize}
				\item\code{}\\
			\end{itemize}
		\item\code{}\\
		\textbf{Parametri}:
			\begin{itemize}
				\item\code{}\\
			\end{itemize}
		\item\code{}\\
		\textbf{Parametri}:
			\begin{itemize}
				\item\code{}\\
			\end{itemize}
		\item\code{}\\
		\textbf{Parametri}:
			\begin{itemize}
				\item\code{}\\
			\end{itemize}
	\end{itemize}
\end{itemize}
	\newpage
	\section{Package Model}\begin{figure}[h!]
\begin{center}
	\includegraphics[scale=0.7]{../images/ModelPackage.png}
\end{center}
\end{figure}
\subsection{Model::Database}
\subsubsection{Model::Database::UserManager}
\begin{itemize}
\item\textbf{Funzione del componente:} la classe permettera' l'inserimento, la lettura e la rimozione di utenti all'interno della collezione
			\item\textbf{Relazioni d'uso con altre componenti:} interagisce con la ViewModel, gestendone le richieste di interazione sugli utenti\\ \\
La classe utilizza:
	\begin{itemize}
		\item
	\end{itemize}
\item\textbf{Attributi}:
	\begin{itemize}
		\item\code{}\\
		\item\code{}\\
		\item\code{}\\
		\item\code{}\\
	\end{itemize}
\item\textbf{Metodi}:
	\begin{itemize}
		\item\code{}\\
		\textbf{Parametri}:
			\begin{itemize}
				\item\code{}\\
			\end{itemize}
		\item\code{}\\
		\textbf{Parametri}:
			\begin{itemize}
				\item\code{}\\
			\end{itemize}
		\item\code{}\\
		\textbf{Parametri}:
			\begin{itemize}
				\item\code{}\\
			\end{itemize}
		\item\code{}\\
		\textbf{Parametri}:
			\begin{itemize}
				\item\code{}\\
			\end{itemize}
	\end{itemize}
\end{itemize}

\subsubsection{Model::Database::QuizManager}
\begin{itemize}
\item\textbf{Funzione del componente:} la classe permettera' l'inserimento, la lettura e la rimozione di questionari all'interno della collezione
			\item\textbf{Relazioni d'uso con altre componenti:} interagisce con la ViewModel, gestendo le richieste per inserire, modificare o eliminare quiz\\ \\
La classe utilizza:
	\begin{itemize}
		\item
	\end{itemize}
\item\textbf{Attributi}:
	\begin{itemize}
		\item\code{}\\
		\item\code{}\\
		\item\code{}\\
		\item\code{}\\
	\end{itemize}
\item\textbf{Metodi}:
	\begin{itemize}
		\item\code{}\\
		\textbf{Parametri}:
			\begin{itemize}
				\item\code{}\\
			\end{itemize}
		\item\code{}\\
		\textbf{Parametri}:
			\begin{itemize}
				\item\code{}\\
			\end{itemize}
		\item\code{}\\
		\textbf{Parametri}:
			\begin{itemize}
				\item\code{}\\
			\end{itemize}
		\item\code{}\\
		\textbf{Parametri}:
			\begin{itemize}
				\item\code{}\\
			\end{itemize}
	\end{itemize}
\end{itemize}

\subsubsection{Model::Database::QuestionManager}
\begin{itemize}
\item\textbf{Funzione del componente:} la classe permettera' l'inserimento, la lettura e la rimozione di singoli quesiti all'interno della collezione
			\item\textbf{Relazioni d'uso con altre componenti:} interagisce con la ViewModel, gestendo le richieste per inserire, modificare o eliminare quesiti\\ \\
La classe utilizza:
	\begin{itemize}
		\item
	\end{itemize}
\item\textbf{Attributi}:
	\begin{itemize}
		\item\code{}\\
		\item\code{}\\
		\item\code{}\\
		\item\code{}\\
	\end{itemize}
\item\textbf{Metodi}:
	\begin{itemize}
		\item\code{}\\
		\textbf{Parametri}:
			\begin{itemize}
				\item\code{}\\
			\end{itemize}
		\item\code{}\\
		\textbf{Parametri}:
			\begin{itemize}
				\item\code{}\\
			\end{itemize}
		\item\code{}\\
		\textbf{Parametri}:
			\begin{itemize}
				\item\code{}\\
			\end{itemize}
		\item\code{}\\
		\textbf{Parametri}:
			\begin{itemize}
				\item\code{}\\
			\end{itemize}
	\end{itemize}
\end{itemize}

\subsection{Model::Parser}
\subsubsection{Model::Parser::Parser}
\begin{itemize}
\item\textbf{Funzione del componente:} controlla che il testo fornito risulti corretto secondo la sintassi QML
			\item\textbf{Relazioni d'uso con altre componenti:} il Parser controlla che il testo fornito in input rispetta la sintassi QML e fornisce in caso di errore un messaggio avvertendo l'utente di dove si trova l'errore e la tipologia\\ \\
La classe utilizza:
	\begin{itemize}
		\item
	\end{itemize}
\item\textbf{Attributi}:
	\begin{itemize}
		\item\code{}\\
		\item\code{}\\
		\item\code{}\\
		\item\code{}\\
	\end{itemize}
\item\textbf{Metodi}:
	\begin{itemize}
		\item\code{}\\
		\textbf{Parametri}:
			\begin{itemize}
				\item\code{}\\
			\end{itemize}
		\item\code{}\\
		\textbf{Parametri}:
			\begin{itemize}
				\item\code{}\\
			\end{itemize}
		\item\code{}\\
		\textbf{Parametri}:
			\begin{itemize}
				\item\code{}\\
			\end{itemize}
		\item\code{}\\
		\textbf{Parametri}:
			\begin{itemize}
				\item\code{}\\
			\end{itemize}
	\end{itemize}
\end{itemize}

\subsection{Model::Statistics}
\subsubsection{Model::Statistics::Statistics}
\begin{itemize}
\item\textbf{Funzione del componente:} questa classe fornisce funzionalita' per il raccoglimento delle statistiche sulle prestazioni degli utenti del sistema
			\item\textbf{Relazioni d'uso con altre componenti:} \\ \\
La classe utilizza:
	\begin{itemize}
		\item
	\end{itemize}
\item\textbf{Attributi}:
	\begin{itemize}
		\item\code{}\\
		\item\code{}\\
		\item\code{}\\
		\item\code{}\\
	\end{itemize}
\item\textbf{Metodi}:
	\begin{itemize}
		\item\code{}\\
		\textbf{Parametri}:
			\begin{itemize}
				\item\code{}\\
			\end{itemize}
		\item\code{}\\
		\textbf{Parametri}:
			\begin{itemize}
				\item\code{}\\
			\end{itemize}
		\item\code{}\\
		\textbf{Parametri}:
			\begin{itemize}
				\item\code{}\\
			\end{itemize}
		\item\code{}\\
		\textbf{Parametri}:
			\begin{itemize}
				\item\code{}\\
			\end{itemize}
	\end{itemize}
\end{itemize}

\subsection{Model::Publishers}
\subsubsection{Model::Publishers::UserPublishers}
\begin{itemize}
\item\textbf{Funzione del componente:} questa classe fornisce funzionalita' per la pubblicazione degli utenti
				\item\textbf{Relazioni d'uso con altre componenti:} \\ \\
La classe utilizza:
	\begin{itemize}
		\item
	\end{itemize}
\item\textbf{Attributi}:
	\begin{itemize}
		\item\code{}\\
		\item\code{}\\
		\item\code{}\\
		\item\code{}\\
	\end{itemize}
\item\textbf{Metodi}:
	\begin{itemize}
		\item\code{}\\
		\textbf{Parametri}:
			\begin{itemize}
				\item\code{}\\
			\end{itemize}
		\item\code{}\\
		\textbf{Parametri}:
			\begin{itemize}
				\item\code{}\\
			\end{itemize}
		\item\code{}\\
		\textbf{Parametri}:
			\begin{itemize}
				\item\code{}\\
			\end{itemize}
		\item\code{}\\
		\textbf{Parametri}:
			\begin{itemize}
				\item\code{}\\
			\end{itemize}
	\end{itemize}
\end{itemize}

\subsubsection{Model::Publishers::QuizPublishers}
\begin{itemize}
\item\textbf{Funzione del componente:} questa classe fornisce funzionalita' per la pubblicazione dei quiz
\item\textbf{Relazioni con altre componenti}\\
La classe utilizza:
	\begin{itemize}
		\item
	\end{itemize}
\item\textbf{Attributi}:
	\begin{itemize}
		\item\code{}\\
		\item\code{}\\
		\item\code{}\\
		\item\code{}\\
	\end{itemize}
\item\textbf{Metodi}:
	\begin{itemize}
		\item\code{}\\
		\textbf{Parametri}:
			\begin{itemize}
				\item\code{}\\
			\end{itemize}
		\item\code{}\\
		\textbf{Parametri}:
			\begin{itemize}
				\item\code{}\\
			\end{itemize}
		\item\code{}\\
		\textbf{Parametri}:
			\begin{itemize}
				\item\code{}\\
			\end{itemize}
		\item\code{}\\
		\textbf{Parametri}:
			\begin{itemize}
				\item\code{}\\
			\end{itemize}
	\end{itemize}
\end{itemize}

\subsubsection{Model::Publishers::QuestionPublishers}
\begin{itemize}
\item\textbf{Funzione del componente:} questa classe fornisce funzionalita' per la pubblicazione dei quesiti
\item\textbf{Relazioni con altre componenti}\\
La classe utilizza:
	\begin{itemize}
		\item
	\end{itemize}
\item\textbf{Attributi}:
	\begin{itemize}
		\item\code{}\\
		\item\code{}\\
		\item\code{}\\
		\item\code{}\\
	\end{itemize}
\item\textbf{Metodi}:
	\begin{itemize}
		\item\code{}\\
		\textbf{Parametri}:
			\begin{itemize}
				\item\code{}\\
			\end{itemize}
		\item\code{}\\
		\textbf{Parametri}:
			\begin{itemize}
				\item\code{}\\
			\end{itemize}
		\item\code{}\\
		\textbf{Parametri}:
			\begin{itemize}
				\item\code{}\\
			\end{itemize}
		\item\code{}\\
		\textbf{Parametri}:
			\begin{itemize}
				\item\code{}\\
			\end{itemize}
	\end{itemize}
\end{itemize}
	\newpage
	\section{Package Presenter}
\begin{figure}[h!]
\begin{center}
	\includegraphics[scale=0.7]{../images/PresenterPackage.png}
\end{center}
\end{figure}
\subsection{Presenter::UserInputManager}
\subsubsection{Presenter::UserInputManager::InputManager}
\begin{itemize}
\item\textbf{Funzione}:
\item\textbf{Relazioni con altre componenti}\\
La classe utilizza:
	\begin{itemize}
		\item
	\end{itemize}
\item\textbf{Attributi}:
	\begin{itemize}
		\item\code{}\\
		\item\code{}\\
		\item\code{}\\
		\item\code{}\\
	\end{itemize}
\item\textbf{Metodi}:
	\begin{itemize}
		\item\code{}\\
		\textbf{Parametri}:
			\begin{itemize}
				\item\code{}\\
			\end{itemize}
		\item\code{}\\
		\textbf{Parametri}:
			\begin{itemize}
				\item\code{}\\
			\end{itemize}
		\item\code{}\\
		\textbf{Parametri}:
			\begin{itemize}
				\item\code{}\\
			\end{itemize}
		\item\code{}\\
		\textbf{Parametri}:
			\begin{itemize}
				\item\code{}\\
			\end{itemize}
	\end{itemize}
\end{itemize}

\subsubsection{Presenter::UserInputManager::Input}
\begin{itemize}
\item\textbf{Funzione}:
\item\textbf{Relazioni con altre componenti}\\
La classe utilizza:
	\begin{itemize}
		\item
	\end{itemize}
\item\textbf{Attributi}:
	\begin{itemize}
		\item\code{}\\
		\item\code{}\\
		\item\code{}\\
		\item\code{}\\
	\end{itemize}
\item\textbf{Metodi}:
	\begin{itemize}
		\item\code{}\\
		\textbf{Parametri}:
			\begin{itemize}
				\item\code{}\\
			\end{itemize}
		\item\code{}\\
		\textbf{Parametri}:
			\begin{itemize}
				\item\code{}\\
			\end{itemize}
		\item\code{}\\
		\textbf{Parametri}:
			\begin{itemize}
				\item\code{}\\
			\end{itemize}
		\item\code{}\\
		\textbf{Parametri}:
			\begin{itemize}
				\item\code{}\\
			\end{itemize}
	\end{itemize}
\end{itemize}

\subsubsection{Presenter::UserInputManager::CreateQuiz}
\begin{itemize}
\item\textbf{Funzione}:
\item\textbf{Relazioni con altre componenti}\\
La classe utilizza:
	\begin{itemize}
		\item
	\end{itemize}
\item\textbf{Attributi}:
	\begin{itemize}
		\item\code{}\\
		\item\code{}\\
		\item\code{}\\
		\item\code{}\\
	\end{itemize}
\item\textbf{Metodi}:
	\begin{itemize}
		\item\code{}\\
		\textbf{Parametri}:
			\begin{itemize}
				\item\code{}\\
			\end{itemize}
		\item\code{}\\
		\textbf{Parametri}:
			\begin{itemize}
				\item\code{}\\
			\end{itemize}
		\item\code{}\\
		\textbf{Parametri}:
			\begin{itemize}
				\item\code{}\\
			\end{itemize}
		\item\code{}\\
		\textbf{Parametri}:
			\begin{itemize}
				\item\code{}\\
			\end{itemize}
	\end{itemize}
\end{itemize}

\subsubsection{Presenter::UserInputManager::AddQuestion}\begin{itemize}
\item\textbf{Funzione}:
\item\textbf{Relazioni con altre componenti}\\
La classe utilizza:
	\begin{itemize}
		\item
	\end{itemize}
\item\textbf{Attributi}:
	\begin{itemize}
		\item\code{}\\
		\item\code{}\\
		\item\code{}\\
		\item\code{}\\
	\end{itemize}
\item\textbf{Metodi}:
	\begin{itemize}
		\item\code{}\\
		\textbf{Parametri}:
			\begin{itemize}
				\item\code{}\\
			\end{itemize}
		\item\code{}\\
		\textbf{Parametri}:
			\begin{itemize}
				\item\code{}\\
			\end{itemize}
		\item\code{}\\
		\textbf{Parametri}:
			\begin{itemize}
				\item\code{}\\
			\end{itemize}
		\item\code{}\\
		\textbf{Parametri}:
			\begin{itemize}
				\item\code{}\\
			\end{itemize}
	\end{itemize}
\end{itemize}

\subsubsection{Presenter::UserInputManager::RemoveQuestion}
\begin{itemize}
\item\textbf{Funzione}:
\item\textbf{Relazioni con altre componenti}\\
La classe utilizza:
	\begin{itemize}
		\item
	\end{itemize}
\item\textbf{Attributi}:
	\begin{itemize}
		\item\code{}\\
		\item\code{}\\
		\item\code{}\\
		\item\code{}\\
	\end{itemize}
\item\textbf{Metodi}:
	\begin{itemize}
		\item\code{}\\
		\textbf{Parametri}:
			\begin{itemize}
				\item\code{}\\
			\end{itemize}
		\item\code{}\\
		\textbf{Parametri}:
			\begin{itemize}
				\item\code{}\\
			\end{itemize}
		\item\code{}\\
		\textbf{Parametri}:
			\begin{itemize}
				\item\code{}\\
			\end{itemize}
		\item\code{}\\
		\textbf{Parametri}:
			\begin{itemize}
				\item\code{}\\
			\end{itemize}
	\end{itemize}
\end{itemize}

\subsubsection{Presenter::UserInputManager::SaveQuiz}
\begin{itemize}
\item\textbf{Funzione}:
\item\textbf{Relazioni con altre componenti}\\
La classe utilizza:
	\begin{itemize}
		\item
	\end{itemize}
\item\textbf{Attributi}:
	\begin{itemize}
		\item\code{}\\
		\item\code{}\\
		\item\code{}\\
		\item\code{}\\
	\end{itemize}
\item\textbf{Metodi}:
	\begin{itemize}
		\item\code{}\\
		\textbf{Parametri}:
			\begin{itemize}
				\item\code{}\\
			\end{itemize}
		\item\code{}\\
		\textbf{Parametri}:
			\begin{itemize}
				\item\code{}\\
			\end{itemize}
		\item\code{}\\
		\textbf{Parametri}:
			\begin{itemize}
				\item\code{}\\
			\end{itemize}
		\item\code{}\\
		\textbf{Parametri}:
			\begin{itemize}
				\item\code{}\\
			\end{itemize}
	\end{itemize}
\end{itemize}

\subsubsection{Presenter::UserInputManager::CreateQuestion}
\begin{itemize}
\item\textbf{Funzione}:
\item\textbf{Relazioni con altre componenti}\\
La classe utilizza:
	\begin{itemize}
		\item
	\end{itemize}
\item\textbf{Attributi}:
	\begin{itemize}
		\item\code{}\\
		\item\code{}\\
		\item\code{}\\
		\item\code{}\\
	\end{itemize}
\item\textbf{Metodi}:
	\begin{itemize}
		\item\code{}\\
		\textbf{Parametri}:
			\begin{itemize}
				\item\code{}\\
			\end{itemize}
		\item\code{}\\
		\textbf{Parametri}:
			\begin{itemize}
				\item\code{}\\
			\end{itemize}
		\item\code{}\\
		\textbf{Parametri}:
			\begin{itemize}
				\item\code{}\\
			\end{itemize}
		\item\code{}\\
		\textbf{Parametri}:
			\begin{itemize}
				\item\code{}\\
			\end{itemize}
	\end{itemize}
\end{itemize}

\subsubsection{Presenter::UserInputManager::SaveQuestion}
\begin{itemize}
\item\textbf{Funzione}:
\item\textbf{Relazioni con altre componenti}\\
La classe utilizza:
	\begin{itemize}
		\item
	\end{itemize}
\item\textbf{Attributi}:
	\begin{itemize}
		\item\code{}\\
		\item\code{}\\
		\item\code{}\\
		\item\code{}\\
	\end{itemize}
\item\textbf{Metodi}:
	\begin{itemize}
		\item\code{}\\
		\textbf{Parametri}:
			\begin{itemize}
				\item\code{}\\
			\end{itemize}
		\item\code{}\\
		\textbf{Parametri}:
			\begin{itemize}
				\item\code{}\\
			\end{itemize}
		\item\code{}\\
		\textbf{Parametri}:
			\begin{itemize}
				\item\code{}\\
			\end{itemize}
		\item\code{}\\
		\textbf{Parametri}:
			\begin{itemize}
				\item\code{}\\
			\end{itemize}
	\end{itemize}
\end{itemize}

\subsubsection{Presenter::UserInputManager::DeleteQuestion}
\begin{itemize}
\item\textbf{Funzione}:
\item\textbf{Relazioni con altre componenti}\\
La classe utilizza:
	\begin{itemize}
		\item
	\end{itemize}
\item\textbf{Attributi}:
	\begin{itemize}
		\item\code{}\\
		\item\code{}\\
		\item\code{}\\
		\item\code{}\\
	\end{itemize}
\item\textbf{Metodi}:
	\begin{itemize}
		\item\code{}\\
		\textbf{Parametri}:
			\begin{itemize}
				\item\code{}\\
			\end{itemize}
		\item\code{}\\
		\textbf{Parametri}:
			\begin{itemize}
				\item\code{}\\
			\end{itemize}
		\item\code{}\\
		\textbf{Parametri}:
			\begin{itemize}
				\item\code{}\\
			\end{itemize}
		\item\code{}\\
		\textbf{Parametri}:
			\begin{itemize}
				\item\code{}\\
			\end{itemize}
	\end{itemize}
\end{itemize}

\subsubsection{Presenter::UserInputManager::ChooseQuiz}
\begin{itemize}
\item\textbf{Funzione}:
\item\textbf{Relazioni con altre componenti}\\
La classe utilizza:
	\begin{itemize}
		\item
	\end{itemize}
\item\textbf{Attributi}:
	\begin{itemize}
		\item\code{}\\
		\item\code{}\\
		\item\code{}\\
		\item\code{}\\
	\end{itemize}
\item\textbf{Metodi}:
	\begin{itemize}
		\item\code{}\\
		\textbf{Parametri}:
			\begin{itemize}
				\item\code{}\\
			\end{itemize}
		\item\code{}\\
		\textbf{Parametri}:
			\begin{itemize}
				\item\code{}\\
			\end{itemize}
		\item\code{}\\
		\textbf{Parametri}:
			\begin{itemize}
				\item\code{}\\
			\end{itemize}
		\item\code{}\\
		\textbf{Parametri}:
			\begin{itemize}
				\item\code{}\\
			\end{itemize}
	\end{itemize}
\end{itemize}

\subsubsection{Presenter::UserInputManager::StartQuiz}
\begin{itemize}
\item\textbf{Funzione}:
\item\textbf{Relazioni con altre componenti}\\
La classe utilizza:
	\begin{itemize}
		\item
	\end{itemize}
\item\textbf{Attributi}:
	\begin{itemize}
		\item\code{}\\
		\item\code{}\\
		\item\code{}\\
		\item\code{}\\
	\end{itemize}
\item\textbf{Metodi}:
	\begin{itemize}
		\item\code{}\\
		\textbf{Parametri}:
			\begin{itemize}
				\item\code{}\\
			\end{itemize}
		\item\code{}\\
		\textbf{Parametri}:
			\begin{itemize}
				\item\code{}\\
			\end{itemize}
		\item\code{}\\
		\textbf{Parametri}:
			\begin{itemize}
				\item\code{}\\
			\end{itemize}
		\item\code{}\\
		\textbf{Parametri}:
			\begin{itemize}
				\item\code{}\\
			\end{itemize}
	\end{itemize}
\end{itemize}

\subsubsection{Presenter::UserInputManager::NextQuestion}
\begin{itemize}
\item\textbf{Funzione}:
\item\textbf{Relazioni con altre componenti}\\
La classe utilizza:
	\begin{itemize}
		\item
	\end{itemize}
\item\textbf{Attributi}:
	\begin{itemize}
		\item\code{}\\
		\item\code{}\\
		\item\code{}\\
		\item\code{}\\
	\end{itemize}
\item\textbf{Metodi}:
	\begin{itemize}
		\item\code{}\\
		\textbf{Parametri}:
			\begin{itemize}
				\item\code{}\\
			\end{itemize}
		\item\code{}\\
		\textbf{Parametri}:
			\begin{itemize}
				\item\code{}\\
			\end{itemize}
		\item\code{}\\
		\textbf{Parametri}:
			\begin{itemize}
				\item\code{}\\
			\end{itemize}
		\item\code{}\\
		\textbf{Parametri}:
			\begin{itemize}
				\item\code{}\\
			\end{itemize}
	\end{itemize}
\end{itemize}

\subsubsection{Presenter::UserInputManager::PreviousQuestion}
\begin{itemize}
\item\textbf{Funzione}:
\item\textbf{Relazioni con altre componenti}\\
La classe utilizza:
	\begin{itemize}
		\item
	\end{itemize}
\item\textbf{Attributi}:
	\begin{itemize}
		\item\code{}\\
		\item\code{}\\
		\item\code{}\\
		\item\code{}\\
	\end{itemize}
\item\textbf{Metodi}:
	\begin{itemize}
		\item\code{}\\
		\textbf{Parametri}:
			\begin{itemize}
				\item\code{}\\
			\end{itemize}
		\item\code{}\\
		\textbf{Parametri}:
			\begin{itemize}
				\item\code{}\\
			\end{itemize}
		\item\code{}\\
		\textbf{Parametri}:
			\begin{itemize}
				\item\code{}\\
			\end{itemize}
		\item\code{}\\
		\textbf{Parametri}:
			\begin{itemize}
				\item\code{}\\
			\end{itemize}
	\end{itemize}
\end{itemize}

\subsubsection{Presenter::UserInputManager::EndQuiz}
\begin{itemize}
\item\textbf{Funzione}:
\item\textbf{Relazioni con altre componenti}\\
La classe utilizza:
	\begin{itemize}
		\item
	\end{itemize}
\item\textbf{Attributi}:
	\begin{itemize}
		\item\code{}\\
		\item\code{}\\
		\item\code{}\\
		\item\code{}\\
	\end{itemize}
\item\textbf{Metodi}:
	\begin{itemize}
		\item\code{}\\
		\textbf{Parametri}:
			\begin{itemize}
				\item\code{}\\
			\end{itemize}
		\item\code{}\\
		\textbf{Parametri}:
			\begin{itemize}
				\item\code{}\\
			\end{itemize}
		\item\code{}\\
		\textbf{Parametri}:
			\begin{itemize}
				\item\code{}\\
			\end{itemize}
		\item\code{}\\
		\textbf{Parametri}:
			\begin{itemize}
				\item\code{}\\
			\end{itemize}
	\end{itemize}
\end{itemize}

\subsubsection{Presenter::UserInputManager::ViewProfile}
\begin{itemize}
\item\textbf{Funzione}:
\item\textbf{Relazioni con altre componenti}\\
La classe utilizza:
	\begin{itemize}
		\item
	\end{itemize}
\item\textbf{Attributi}:
	\begin{itemize}
		\item\code{}\\
		\item\code{}\\
		\item\code{}\\
		\item\code{}\\
	\end{itemize}
\item\textbf{Metodi}:
	\begin{itemize}
		\item\code{}\\
		\textbf{Parametri}:
			\begin{itemize}
				\item\code{}\\
			\end{itemize}
		\item\code{}\\
		\textbf{Parametri}:
			\begin{itemize}
				\item\code{}\\
			\end{itemize}
		\item\code{}\\
		\textbf{Parametri}:
			\begin{itemize}
				\item\code{}\\
			\end{itemize}
		\item\code{}\\
		\textbf{Parametri}:
			\begin{itemize}
				\item\code{}\\
			\end{itemize}
	\end{itemize}
\end{itemize}

\subsubsection{Presenter::UserInputManager::UpdateProfile}
\begin{itemize}
\item\textbf{Funzione}:
\item\textbf{Relazioni con altre componenti}\\
La classe utilizza:
	\begin{itemize}
		\item
	\end{itemize}
\item\textbf{Attributi}:
	\begin{itemize}
		\item\code{}\\
		\item\code{}\\
		\item\code{}\\
		\item\code{}\\
	\end{itemize}
\item\textbf{Metodi}:
	\begin{itemize}
		\item\code{}\\
		\textbf{Parametri}:
			\begin{itemize}
				\item\code{}\\
			\end{itemize}
		\item\code{}\\
		\textbf{Parametri}:
			\begin{itemize}
				\item\code{}\\
			\end{itemize}
		\item\code{}\\
		\textbf{Parametri}:
			\begin{itemize}
				\item\code{}\\
			\end{itemize}
		\item\code{}\\
		\textbf{Parametri}:
			\begin{itemize}
				\item\code{}\\
			\end{itemize}
	\end{itemize}
\end{itemize}

\subsubsection{Presenter::UserInputManager::ViewQMLTutorial}
\begin{itemize}
\item\textbf{Funzione}:
\item\textbf{Relazioni con altre componenti}\\
La classe utilizza:
	\begin{itemize}
		\item
	\end{itemize}
\item\textbf{Attributi}:
	\begin{itemize}
		\item\code{}\\
		\item\code{}\\
		\item\code{}\\
		\item\code{}\\
	\end{itemize}
\item\textbf{Metodi}:
	\begin{itemize}
		\item\code{}\\
		\textbf{Parametri}:
			\begin{itemize}
				\item\code{}\\
			\end{itemize}
		\item\code{}\\
		\textbf{Parametri}:
			\begin{itemize}
				\item\code{}\\
			\end{itemize}
		\item\code{}\\
		\textbf{Parametri}:
			\begin{itemize}
				\item\code{}\\
			\end{itemize}
		\item\code{}\\
		\textbf{Parametri}:
			\begin{itemize}
				\item\code{}\\
			\end{itemize}
	\end{itemize}
\end{itemize}

\subsubsection{Presenter::UserInputManager::ViewQuizList}\begin{itemize}
\item\textbf{Funzione}:
\item\textbf{Relazioni con altre componenti}\\
La classe utilizza:
	\begin{itemize}
		\item
	\end{itemize}
\item\textbf{Attributi}:
	\begin{itemize}
		\item\code{}\\
		\item\code{}\\
		\item\code{}\\
		\item\code{}\\
	\end{itemize}
\item\textbf{Metodi}:
	\begin{itemize}
		\item\code{}\\
		\textbf{Parametri}:
			\begin{itemize}
				\item\code{}\\
			\end{itemize}
		\item\code{}\\
		\textbf{Parametri}:
			\begin{itemize}
				\item\code{}\\
			\end{itemize}
		\item\code{}\\
		\textbf{Parametri}:
			\begin{itemize}
				\item\code{}\\
			\end{itemize}
		\item\code{}\\
		\textbf{Parametri}:
			\begin{itemize}
				\item\code{}\\
			\end{itemize}
	\end{itemize}
\end{itemize}


\subsection{Presenter::Updaters}
\subsubsection{Presenter::Updaters::Updater}
\begin{itemize}
\item\textbf{Funzione}:
\item\textbf{Relazioni con altre componenti}\\
La classe utilizza:
	\begin{itemize}
		\item
	\end{itemize}
\item\textbf{Attributi}:
	\begin{itemize}
		\item\code{}\\
		\item\code{}\\
		\item\code{}\\
		\item\code{}\\
	\end{itemize}
\item\textbf{Metodi}:
	\begin{itemize}
		\item\code{}\\
		\textbf{Parametri}:
			\begin{itemize}
				\item\code{}\\
			\end{itemize}
		\item\code{}\\
		\textbf{Parametri}:
			\begin{itemize}
				\item\code{}\\
			\end{itemize}
		\item\code{}\\
		\textbf{Parametri}:
			\begin{itemize}
				\item\code{}\\
			\end{itemize}
		\item\code{}\\
		\textbf{Parametri}:
			\begin{itemize}
				\item\code{}\\
			\end{itemize}
	\end{itemize}
\end{itemize}

\subsubsection{Presenter::Updaters::ModelUpdater}
\begin{itemize}
\item\textbf{Funzione}:
\item\textbf{Relazioni con altre componenti}\\
La classe utilizza:
	\begin{itemize}
		\item
	\end{itemize}
\item\textbf{Attributi}:
	\begin{itemize}
		\item\code{}\\
		\item\code{}\\
		\item\code{}\\
		\item\code{}\\
	\end{itemize}
\item\textbf{Metodi}:
	\begin{itemize}
		\item\code{}\\
		\textbf{Parametri}:
			\begin{itemize}
				\item\code{}\\
			\end{itemize}
		\item\code{}\\
		\textbf{Parametri}:
			\begin{itemize}
				\item\code{}\\
			\end{itemize}
		\item\code{}\\
		\textbf{Parametri}:
			\begin{itemize}
				\item\code{}\\
			\end{itemize}
		\item\code{}\\
		\textbf{Parametri}:
			\begin{itemize}
				\item\code{}\\
			\end{itemize}
	\end{itemize}
\end{itemize}

\subsubsection{Presenter::Updaters::ViewUpdater}
\begin{itemize}
\item\textbf{Funzione}:
\item\textbf{Relazioni con altre componenti}\\
La classe utilizza:
	\begin{itemize}
		\item
	\end{itemize}
\item\textbf{Attributi}:
	\begin{itemize}
		\item\code{}\\
		\item\code{}\\
		\item\code{}\\
		\item\code{}\\
	\end{itemize}
\item\textbf{Metodi}:
	\begin{itemize}
		\item\code{}\\
		\textbf{Parametri}:
			\begin{itemize}
				\item\code{}\\
			\end{itemize}
		\item\code{}\\
		\textbf{Parametri}:
			\begin{itemize}
				\item\code{}\\
			\end{itemize}
		\item\code{}\\
		\textbf{Parametri}:
			\begin{itemize}
				\item\code{}\\
			\end{itemize}
		\item\code{}\\
		\textbf{Parametri}:
			\begin{itemize}
				\item\code{}\\
			\end{itemize}
	\end{itemize}
\end{itemize}


\subsection{Presenter::QuestionManager}
\subsubsection{Presenter::QuestionManager::QuestionFactory}
\begin{itemize}
\item\textbf{Funzione}:
\item\textbf{Relazioni con altre componenti}\\
La classe utilizza:
	\begin{itemize}
		\item
	\end{itemize}
\item\textbf{Attributi}:
	\begin{itemize}
		\item\code{}\\
		\item\code{}\\
		\item\code{}\\
		\item\code{}\\
	\end{itemize}
\item\textbf{Metodi}:
	\begin{itemize}
		\item\code{}\\
		\textbf{Parametri}:
			\begin{itemize}
				\item\code{}\\
			\end{itemize}
		\item\code{}\\
		\textbf{Parametri}:
			\begin{itemize}
				\item\code{}\\
			\end{itemize}
		\item\code{}\\
		\textbf{Parametri}:
			\begin{itemize}
				\item\code{}\\
			\end{itemize}
		\item\code{}\\
		\textbf{Parametri}:
			\begin{itemize}
				\item\code{}\\
			\end{itemize}
	\end{itemize}
\end{itemize}

\subsubsection{Presenter::QuestionManager::Question}
\begin{itemize}
\item\textbf{Funzione}:
\item\textbf{Relazioni con altre componenti}\\
La classe utilizza:
	\begin{itemize}
		\item
	\end{itemize}
\item\textbf{Attributi}:
	\begin{itemize}
		\item\code{}\\
		\item\code{}\\
		\item\code{}\\
		\item\code{}\\
	\end{itemize}
\item\textbf{Metodi}:
	\begin{itemize}
		\item\code{}\\
		\textbf{Parametri}:
			\begin{itemize}
				\item\code{}\\
			\end{itemize}
		\item\code{}\\
		\textbf{Parametri}:
			\begin{itemize}
				\item\code{}\\
			\end{itemize}
		\item\code{}\\
		\textbf{Parametri}:
			\begin{itemize}
				\item\code{}\\
			\end{itemize}
		\item\code{}\\
		\textbf{Parametri}:
			\begin{itemize}
				\item\code{}\\
			\end{itemize}
	\end{itemize}
\end{itemize}

\subsubsection{Presenter::QuestionManager::QML2HTMLFactory}
\begin{itemize}
\item\textbf{Funzione}:
\item\textbf{Relazioni con altre componenti}\\
La classe utilizza:
	\begin{itemize}
		\item
	\end{itemize}
\item\textbf{Attributi}:
	\begin{itemize}
		\item\code{}\\
		\item\code{}\\
		\item\code{}\\
		\item\code{}\\
	\end{itemize}
\item\textbf{Metodi}:
	\begin{itemize}
		\item\code{}\\
		\textbf{Parametri}:
			\begin{itemize}
				\item\code{}\\
			\end{itemize}
		\item\code{}\\
		\textbf{Parametri}:
			\begin{itemize}
				\item\code{}\\
			\end{itemize}
		\item\code{}\\
		\textbf{Parametri}:
			\begin{itemize}
				\item\code{}\\
			\end{itemize}
		\item\code{}\\
		\textbf{Parametri}:
			\begin{itemize}
				\item\code{}\\
			\end{itemize}
	\end{itemize}
\end{itemize}

\subsubsection{Presenter::QuestionManager::HTMLQuestion}
\begin{itemize}
\item\textbf{Funzione}:
\item\textbf{Relazioni con altre componenti}\\
La classe utilizza:
	\begin{itemize}
		\item
	\end{itemize}
\item\textbf{Attributi}:
	\begin{itemize}
		\item\code{}\\
		\item\code{}\\
		\item\code{}\\
		\item\code{}\\
	\end{itemize}
\item\textbf{Metodi}:
	\begin{itemize}
		\item\code{}\\
		\textbf{Parametri}:
			\begin{itemize}
				\item\code{}\\
			\end{itemize}
		\item\code{}\\
		\textbf{Parametri}:
			\begin{itemize}
				\item\code{}\\
			\end{itemize}
		\item\code{}\\
		\textbf{Parametri}:
			\begin{itemize}
				\item\code{}\\
			\end{itemize}
		\item\code{}\\
		\textbf{Parametri}:
			\begin{itemize}
				\item\code{}\\
			\end{itemize}
	\end{itemize}
\end{itemize}

\subsubsection{Presenter::QuestionManager::Translator}
\begin{itemize}
\item\textbf{Funzione}:
\item\textbf{Relazioni con altre componenti}\\
La classe utilizza:
	\begin{itemize}
		\item
	\end{itemize}
\item\textbf{Attributi}:
	\begin{itemize}
		\item\code{}\\
		\item\code{}\\
		\item\code{}\\
		\item\code{}\\
	\end{itemize}
\item\textbf{Metodi}:
	\begin{itemize}
		\item\code{}\\
		\textbf{Parametri}:
			\begin{itemize}
				\item\code{}\\
			\end{itemize}
		\item\code{}\\
		\textbf{Parametri}:
			\begin{itemize}
				\item\code{}\\
			\end{itemize}
		\item\code{}\\
		\textbf{Parametri}:
			\begin{itemize}
				\item\code{}\\
			\end{itemize}
		\item\code{}\\
		\textbf{Parametri}:
			\begin{itemize}
				\item\code{}\\
			\end{itemize}
	\end{itemize}
\end{itemize}


\subsection{Presenter::Interpreter}
\subsubsection{Presenter::Interpreter::Interpreter}
\begin{itemize}
\item\textbf{Funzione}:
\item\textbf{Relazioni con altre componenti}\\
La classe utilizza:
	\begin{itemize}
		\item
	\end{itemize}
\item\textbf{Attributi}:
	\begin{itemize}
		\item\code{}\\
		\item\code{}\\
		\item\code{}\\
		\item\code{}\\
	\end{itemize}
\item\textbf{Metodi}:
	\begin{itemize}
		\item\code{}\\
		\textbf{Parametri}:
			\begin{itemize}
				\item\code{}\\
			\end{itemize}
		\item\code{}\\
		\textbf{Parametri}:
			\begin{itemize}
				\item\code{}\\
			\end{itemize}
		\item\code{}\\
		\textbf{Parametri}:
			\begin{itemize}
				\item\code{}\\
			\end{itemize}
		\item\code{}\\
		\textbf{Parametri}:
			\begin{itemize}
				\item\code{}\\
			\end{itemize}
	\end{itemize}
\end{itemize}

\subsubsection{Presenter::Interpreter::InterpreterFactory}
\begin{itemize}
\item\textbf{Funzione}:
\item\textbf{Relazioni con altre componenti}\\
La classe utilizza:
	\begin{itemize}
		\item
	\end{itemize}
\item\textbf{Attributi}:
	\begin{itemize}
		\item\code{}\\
		\item\code{}\\
		\item\code{}\\
		\item\code{}\\
	\end{itemize}
\item\textbf{Metodi}:
	\begin{itemize}
		\item\code{}\\
		\textbf{Parametri}:
			\begin{itemize}
				\item\code{}\\
			\end{itemize}
		\item\code{}\\
		\textbf{Parametri}:
			\begin{itemize}
				\item\code{}\\
			\end{itemize}
		\item\code{}\\
		\textbf{Parametri}:
			\begin{itemize}
				\item\code{}\\
			\end{itemize}
		\item\code{}\\
		\textbf{Parametri}:
			\begin{itemize}
				\item\code{}\\
			\end{itemize}
	\end{itemize}
\end{itemize}

\subsubsection{Presenter::Interpreter::QMLInterpreterFactory}
\begin{itemize}
\item\textbf{Funzione}:
\item\textbf{Relazioni con altre componenti}\\
La classe utilizza:
	\begin{itemize}
		\item
	\end{itemize}
\item\textbf{Attributi}:
	\begin{itemize}
		\item\code{}\\
		\item\code{}\\
		\item\code{}\\
		\item\code{}\\
	\end{itemize}
\item\textbf{Metodi}:
	\begin{itemize}
		\item\code{}\\
		\textbf{Parametri}:
			\begin{itemize}
				\item\code{}\\
			\end{itemize}
		\item\code{}\\
		\textbf{Parametri}:
			\begin{itemize}
				\item\code{}\\
			\end{itemize}
		\item\code{}\\
		\textbf{Parametri}:
			\begin{itemize}
				\item\code{}\\
			\end{itemize}
		\item\code{}\\
		\textbf{Parametri}:
			\begin{itemize}
				\item\code{}\\
			\end{itemize}
	\end{itemize}
\end{itemize}

\subsubsection{Presenter::Interpreter::QMLInterpreter}
\begin{itemize}
\item\textbf{Funzione}:
\item\textbf{Relazioni con altre componenti}\\
La classe utilizza:
	\begin{itemize}
		\item
	\end{itemize}
\item\textbf{Attributi}:
	\begin{itemize}
		\item\code{}\\
		\item\code{}\\
		\item\code{}\\
		\item\code{}\\
	\end{itemize}
\item\textbf{Metodi}:
	\begin{itemize}
		\item\code{}\\
		\textbf{Parametri}:
			\begin{itemize}
				\item\code{}\\
			\end{itemize}
		\item\code{}\\
		\textbf{Parametri}:
			\begin{itemize}
				\item\code{}\\
			\end{itemize}
		\item\code{}\\
		\textbf{Parametri}:
			\begin{itemize}
				\item\code{}\\
			\end{itemize}
		\item\code{}\\
		\textbf{Parametri}:
			\begin{itemize}
				\item\code{}\\
			\end{itemize}
	\end{itemize}
\end{itemize}

\subsubsection{Presenter::Interpreter::QML2HTMLInterpreterFactory}
\begin{itemize}
\item\textbf{Funzione}:
\item\textbf{Relazioni con altre componenti}\\
La classe utilizza:
	\begin{itemize}
		\item
	\end{itemize}
\item\textbf{Attributi}:
	\begin{itemize}
		\item\code{}\\
		\item\code{}\\
		\item\code{}\\
		\item\code{}\\
	\end{itemize}
\item\textbf{Metodi}:
	\begin{itemize}
		\item\code{}\\
		\textbf{Parametri}:
			\begin{itemize}
				\item\code{}\\
			\end{itemize}
		\item\code{}\\
		\textbf{Parametri}:
			\begin{itemize}
				\item\code{}\\
			\end{itemize}
		\item\code{}\\
		\textbf{Parametri}:
			\begin{itemize}
				\item\code{}\\
			\end{itemize}
		\item\code{}\\
		\textbf{Parametri}:
			\begin{itemize}
				\item\code{}\\
			\end{itemize}
	\end{itemize}
\end{itemize}


\subsection{Presenter::QuizManager}
\subsubsection{Presenter::QuizManager::Quiz}
\begin{itemize}
\item\textbf{Funzione}:
\item\textbf{Relazioni con altre componenti}\\
La classe utilizza:
	\begin{itemize}
		\item
	\end{itemize}
\item\textbf{Attributi}:
	\begin{itemize}
		\item\code{}\\
		\item\code{}\\
		\item\code{}\\
		\item\code{}\\
	\end{itemize}
\item\textbf{Metodi}:
	\begin{itemize}
		\item\code{}\\
		\textbf{Parametri}:
			\begin{itemize}
				\item\code{}\\
			\end{itemize}
		\item\code{}\\
		\textbf{Parametri}:
			\begin{itemize}
				\item\code{}\\
			\end{itemize}
		\item\code{}\\
		\textbf{Parametri}:
			\begin{itemize}
				\item\code{}\\
			\end{itemize}
		\item\code{}\\
		\textbf{Parametri}:
			\begin{itemize}
				\item\code{}\\
			\end{itemize}
	\end{itemize}
\end{itemize}

\subsubsection{Presenter::QuizManager::QuizDirector}
\begin{itemize}
\item\textbf{Funzione}:
\item\textbf{Relazioni con altre componenti}\\
La classe utilizza:
	\begin{itemize}
		\item
	\end{itemize}
\item\textbf{Attributi}:
	\begin{itemize}
		\item\code{}\\
		\item\code{}\\
		\item\code{}\\
		\item\code{}\\
	\end{itemize}
\item\textbf{Metodi}:
	\begin{itemize}
		\item\code{}\\
		\textbf{Parametri}:
			\begin{itemize}
				\item\code{}\\
			\end{itemize}
		\item\code{}\\
		\textbf{Parametri}:
			\begin{itemize}
				\item\code{}\\
			\end{itemize}
		\item\code{}\\
		\textbf{Parametri}:
			\begin{itemize}
				\item\code{}\\
			\end{itemize}
		\item\code{}\\
		\textbf{Parametri}:
			\begin{itemize}
				\item\code{}\\
			\end{itemize}
	\end{itemize}
\end{itemize}

\subsubsection{Presenter::QuizManager::QuizBuilder}
\begin{itemize}
\item\textbf{Funzione}:
\item\textbf{Relazioni con altre componenti}\\
La classe utilizza:
	\begin{itemize}
		\item
	\end{itemize}
\item\textbf{Attributi}:
	\begin{itemize}
		\item\code{}\\
		\item\code{}\\
		\item\code{}\\
		\item\code{}\\
	\end{itemize}
\item\textbf{Metodi}:
	\begin{itemize}
		\item\code{}\\
		\textbf{Parametri}:
			\begin{itemize}
				\item\code{}\\
			\end{itemize}
		\item\code{}\\
		\textbf{Parametri}:
			\begin{itemize}
				\item\code{}\\
			\end{itemize}
		\item\code{}\\
		\textbf{Parametri}:
			\begin{itemize}
				\item\code{}\\
			\end{itemize}
		\item\code{}\\
		\textbf{Parametri}:
			\begin{itemize}
				\item\code{}\\
			\end{itemize}
	\end{itemize}
\end{itemize}




	\newpage
	\section{Tracciamento}
\subsection{Mappatura classi - requisiti}
\begin{longtable}{p{0.8\textwidth}p{0.20\textwidth}}

\caption{Mappatura delle classi sui requisiti} \\

Nome classe & Codice requisito \\
\midrule
\endfirsthead

Nome classe & Codice requisito \\
\midrule
\endhead

\multicolumn{2}{c}{\footnotesize\itshape\tablename~\thetable: Mappatura delle classi sui requisiti}
\endfoot

\multicolumn{2}{c}{\footnotesize\itshape\tablename~\thetable: Mappatura delle classi sui requisiti}
\endlastfoot


Model::Database::UnaClasse 	& F 1.2.1.2\\
							& F 1.2.2\\
							& F 1.2.3\\
							& F 2.3.1.1\\
							& F 3\\
							& F 3.1.1\\							
							
							
							
							
\end{longtable}
\newpage

\subsection{Mappatura requisiti - classi}

\begin{longtable}{p{0.20\textwidth}p{0.80\textwidth}}
\caption{Mappatura dei requisiti sulle classi} \\

Codice Requisito & Nome Classe \\
\midrule
\endfirsthead

Codice Requisito & Nome Classe \\
\midrule
\endhead

\multicolumn{2}{c}{\footnotesize\itshape\tablename~\thetable: Mappatura dei requisiti sulle classi}
\endfoot

\multicolumn{2}{c}{\footnotesize\itshape\tablename~\thetable: Mappatura dei requisiti sulle classi}
\endlastfoot

F X & classe\\
	& altra::classe\\
\midrule





\end{longtable}
\end{document}